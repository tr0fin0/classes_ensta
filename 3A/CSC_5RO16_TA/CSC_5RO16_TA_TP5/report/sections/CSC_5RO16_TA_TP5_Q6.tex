\documentclass[../CSC_5RO16_TA_TP5.tex]{subfiles}

\begin{document}
\section{Question 6}
% 

\subsection{Algorithme}

\subsubsection{domain-monkey}
\begin{scriptsize}\mycode
    \lstinputlisting[
        caption={Algorithme \texttt{domain-monkey.pddl}},
        language={bash},
    ]{../../src/CSC_5RO16_TA_TP5_src_domain-monkey.pddl}
\end{scriptsize}

\subsubsection{problem-monkey}
\begin{scriptsize}\mycode
    \lstinputlisting[
        caption={Algorithme \texttt{problem-monkey.pddl}},
        language={bash},
    ]{../../src/CSC_5RO16_TA_TP5_src_problem-monkey.pddl}
\end{scriptsize}

\subsection{Analyse}

\noindent L'algorithme a été lancé avec la commande suivante sur Windows PowerShell:

\begin{scriptsize}\mycode
	\begin{lstlisting}[
        caption={Lancement \texttt{cpt.exe} pour \texttt{problem-monkey.pddl}},
        language={bash}
    ]
    .\cpt.exe -o .\domain-monkey.pddl -f .\problem-monkey.pddl
    \end{lstlisting}
\end{scriptsize}

\noindent Qui a retourné comme résultat le suivant:

\begin{scriptsize}\mycode
    \begin{lstlisting}[
        caption={R\'esultat \texttt{cpt.exe} pour \texttt{problem-monkey.pddl}},
        language={bash}
    ]
    domain file : .\CSC_5RO16_TA_TP5_src_domain-monkey.pddl
    problem file : .\CSC_5RO16_TA_TP5_src_problem-monkey.pddl
    
    Parsing domain.................... done : 0.00
    Parsing problem................... done : 0.00
    domain : monkey
    problem : monkey-banana
    Instantiating operators........... done : 0.00
    Creating initial structures....... done : 0.00
    Computing bound................... done : 0.00
    Computing e-deleters.............. done : 0.00
    Finalizing e-deleters............. done : 0.00
    Refreshing structures............. done : 0.00
    Computing distances............... done : 0.00
    Finalizing structures............. done : 0.00
    Variables creation................ done : 0.00
    Bad supporters.................... done : 0.00
    Distance boosting................. done : 0.00
    Initial propagations.............. done : 0.00
    
    Problem : 102 actions, 18 fluents, 275 causals
              4 init facts, 3 goals
    
    Bound : 4  ---  Nodes : 1  ---  Backtracks : 0  ---  Iteration time : 0.00
    
    0: (aller a b) [1]
    1: (pousser a b c) [1]
    2: (monter c) [1]
    3: (attraper bananes c) [1]
    
    Makespan : 4
    Length : 4
    Nodes : 1
    Backtracks : 0
    Support choices : 1
    Conflict choices : 0
    Mutex choices : 0
    Start time choices : 0
    World size : 100K
    Nodes/sec : 1000.00
    Search time : 0.00
    Total time : 0.02
    \end{lstlisting}
\end{scriptsize}

\subsubsection{Longueur}
\begin{resolution}
    La longueur du plan-solution est de 4 actions, correspondant à une séquence minimale permettant d'atteindre l'objectif

    \begin{scriptsize}\mycode
        \begin{lstlisting}[
            language={bash}
        ]
    Length : 4
        \end{lstlisting}
    \end{scriptsize}
\end{resolution}

\subsubsection{Temps d'Exécution}
\begin{resolution}
    Le plan-solution a été trouvé en 0,00 seconde de temps de recherche, avec un temps total d'exécution y compris le parsing et les calculs initiaux de 0,02 seconde.

    \begin{scriptsize}\mycode
        \begin{lstlisting}[
            language={bash}
        ]
    Search time : 0.00
    Total time : 0.02
        \end{lstlisting}
    \end{scriptsize}
\end{resolution}

\subsubsection{Itérations}
\begin{resolution}
    Le planificateur a effectué 1 itérations pour trouver la solution. Cela indique qu'aucune exploration ou backtracking n'a été nécessaire.

    \begin{scriptsize}\mycode
        \begin{lstlisting}[
            language={bash}
        ]
    Bound : 4  ---  Nodes : 1  ---  Backtracks : 0  ---  Iteration time : 0.00
        \end{lstlisting}
    \end{scriptsize}
\end{resolution}

\subsubsection{Ramification}
\noindent En effet, les préconditions du modèle ne spécifient pas toutes les conditions nécessaires pour que l'opérateur réussisse (comme le poids de l'objet ou la force du singe).\\

\noindent Ajouter de telles conditions rendrait le modèle très complexe, et certaines limitations doivent être implicitement comprises par l'utilisateur du modèle.

\begin{example}
    Si l'objet à pousser est trop lourd, l'opérateur "pousser" n'aurait pas d'effet, car la précondition n'est pas satisfaite.
\end{example}
\end{document}
