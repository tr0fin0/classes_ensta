\documentclass[../CSC_5RO16_TA_TP5.tex]{subfiles}

\begin{document}
\section{Question 3}
% 

\subsection{Algorithme}

\begin{scriptsize}\mycode
    \lstinputlisting[
    language={bash},
    caption={Algorithme \texttt{blocksaips02.pddl}},
    ]{../../src/CSC_5RO16_TA_TP5_src_blocksaips02.pddl}
\end{scriptsize}

\subsection{Analyse}

\noindent L'algorithme a été lancé avec la commande suivante sur Windows PowerShell:

\begin{scriptsize}\mycode
	\begin{lstlisting}[
        caption={Lancement \texttt{cpt.exe} pour \texttt{blocksaips02.pddl}},
        language={bash}
    ]
    .\cpt.exe -o .\domain-blocksaips.pddl -f .\blocksaips02.pddl
    \end{lstlisting}
\end{scriptsize}

\noindent Qui a retourné comme résultat le suivant:

\begin{scriptsize}\mycode
    \begin{lstlisting}[
        caption={R\'esultat \texttt{cpt.exe} pour \texttt{blocksaips02.pddl}},
        language={bash}
    ]
    domain file : .\domain-blocksaips.pddl
    problem file : .\blocksaips02.pddl
    
    Parsing domain.................... done : 0.00
    Parsing problem................... done : 0.00
    domain : blocks
    problem : blocks-4-1
    Instantiating operators........... done : 0.00
    Creating initial structures....... done : 0.00
    Computing bound................... done : 0.00
    Computing e-deleters.............. done : 0.00
    Finalizing e-deleters............. done : 0.00
    Refreshing structures............. done : 0.00
    Computing distances............... done : 0.00
    Finalizing structures............. done : 0.00
    Variables creation................ done : 0.00
    Bad supporters.................... done : 0.00
    Distance boosting................. done : 0.00
    Initial propagations.............. done : 0.00
    
    Problem : 34 actions, 25 fluents, 79 causals
              6 init facts, 3 goals
    
    Bound : 10  ---  Nodes : 0  ---  Backtracks : 0  ---  Iteration time : 0.00
    
    0: (unstack b c) [1]
    1: (put-down b) [1]
    2: (unstack c a) [1]
    3: (put-down c) [1]
    4: (unstack a d) [1]
    5: (stack a b) [1]
    6: (pick-up c) [1]
    7: (stack c a) [1]
    8: (pick-up d) [1]
    9: (stack d c) [1]
    
    Makespan : 10
    Length : 10
    Nodes : 0
    Backtracks : 0
    Support choices : 0
    Conflict choices : 0
    Mutex choices : 0
    Start time choices : 0
    World size : 100K
    Nodes/sec : 0.00
    Search time : 0.00
    Total time : 0.02
    \end{lstlisting}
\end{scriptsize}

\subsubsection{Longueur}
\begin{resolution}
    La longueur du plan-solution est de 10 actions, correspondant à une séquence minimale permettant d'atteindre l'objectif

    \begin{scriptsize}\mycode
        \begin{lstlisting}[
            language={bash}
        ]
    Length : 10
        \end{lstlisting}
    \end{scriptsize}
\end{resolution}

\subsubsection{Temps d'Exécution}
\begin{resolution}
    Le plan-solution a été trouvé en 0,00 seconde de temps de recherche, avec un temps total d'exécution y compris le parsing et les calculs initiaux de 0,02 seconde.

    \begin{scriptsize}\mycode
        \begin{lstlisting}[
            language={bash}
        ]
    Search time : 0.00
    Total time : 0.02
        \end{lstlisting}
    \end{scriptsize}
\end{resolution}

\subsubsection{Itérations}
\begin{resolution}
    Le planificateur a effectué 1 itérations pour trouver la solution. Cela indique qu'aucune exploration ou backtracking n'a été nécessaire.

    \begin{scriptsize}\mycode
        \begin{lstlisting}[
            language={bash}
        ]
    Bound : 10  ---  Nodes : 0  ---  Backtracks : 0  ---  Iteration time : 0.00
        \end{lstlisting}
    \end{scriptsize}
\end{resolution}
\end{document}
