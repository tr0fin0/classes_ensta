\documentclass[../CSC_5RO16_TA_TP5.tex]{subfiles}

\begin{document}
\section{Question 2}
% 
\noindent L'algorithme a été lancé avec la commande suivante sur Windows PowerShell:
\begin{scriptsize}\mycode
	\begin{lstlisting}[
        caption={Lancement \texttt{cpt.exe} pour \texttt{blocksaips01.pddl}},
        language={bash}
    ]
    .\cpt.exe -o .\domain-blocksaips.pddl -f .\blocksaips01.pddl
    \end{lstlisting}
\end{scriptsize}

\noindent Qui a retourné comme résultat le suivant:

\begin{scriptsize}\mycode
	\begin{lstlisting}[
        caption={R\'esultat \texttt{cpt.exe} pour \texttt{blocksaips01.pddl}},
        language={bash}
    ]
    domain file : .\domain-blocksaips.pddl
    problem file : .\blocksaips01.pddl
    
    Parsing domain.................... done : 0.00
    Parsing problem................... done : 0.00
    domain : blocks
    problem : blocks-4-0
    Instantiating operators........... done : 0.00
    Creating initial structures....... done : 0.00
    Computing bound................... done : 0.00
    Computing e-deleters.............. done : 0.00
    Finalizing e-deleters............. done : 0.00
    Refreshing structures............. done : 0.00
    Computing distances............... done : 0.00
    Finalizing structures............. done : 0.00
    Variables creation................ done : 0.00
    Bad supporters.................... done : 0.00
    Distance boosting................. done : 0.00
    Initial propagations.............. done : 0.00
    
    Problem : 34 actions, 25 fluents, 79 causals
              9 init facts, 3 goals
    
    Bound : 6  ---  Nodes : 0  ---  Backtracks : 0  ---  Iteration time : 0.00
    
    0: (pick-up b) [1]
    1: (stack b a) [1]
    2: (pick-up c) [1]
    3: (stack c b) [1]
    4: (pick-up d) [1]
    5: (stack d c) [1]
    
    Makespan : 6
    Length : 6
    Nodes : 0
    Backtracks : 0
    Support choices : 0
    Conflict choices : 0
    Mutex choices : 0
    Start time choices : 0
    World size : 100K
    Nodes/sec : 0.00
    Search time : 0.00
    Total time : 0.02
	\end{lstlisting}
\end{scriptsize}

\subsection{Plan-Solution}
\subsubsection{Longueur}
\begin{resolution}
    La longueur du plan-solution est de 6 actions, correspondant à une séquence minimale permettant d'atteindre l'objectif

    \begin{scriptsize}\mycode
        \begin{lstlisting}[
            language={bash}
        ]
    Length : 6
        \end{lstlisting}
    \end{scriptsize}
\end{resolution}

\subsubsection{Temps d'Exécution}
\begin{resolution}
    Le plan-solution a été trouvé en 0,00 seconde de temps de recherche, avec un temps total d'exécution y compris le parsing et les calculs initiaux de 0,02 seconde.

    \begin{scriptsize}\mycode
        \begin{lstlisting}[
            language={bash}
        ]
    Search time : 0.00
    Total time : 0.02
        \end{lstlisting}
    \end{scriptsize}
\end{resolution}

\subsubsection{Itérations}
\begin{resolution}
    Le planificateur a effectué 1 itérations pour trouver la solution. Cela indique qu'aucune exploration ou backtracking n'a été nécessaire.

    \begin{scriptsize}\mycode
        \begin{lstlisting}[
            language={bash}
        ]
    Bound : 6  ---  Nodes : 0  ---  Backtracks : 0  ---  Iteration time : 0.00
        \end{lstlisting}
    \end{scriptsize}
\end{resolution}

\subsubsection{Durée d'Action}
\begin{resolution}
    Dans ce cas, chaque action est considérée comme prenant 1 unité de temps, ce qui est arbitraire et dépend de la manière dont le domaine est modélisé.\\

    Le \texttt{Makespan} total est donc de 6 unités de temps, correspondant aux 6 actions du plan.

    \begin{scriptsize}\mycode
        \begin{lstlisting}[
            language={bash}
        ]
    Bound : 6  ---  Nodes : 0  ---  Backtracks : 0  ---  Iteration time : 0.00
    
    0: (pick-up b) [1]
    1: (stack b a) [1]
    2: (pick-up c) [1]
    3: (stack c b) [1]
    4: (pick-up d) [1]
    5: (stack d c) [1]
    
    Makespan : 6
        \end{lstlisting}
    \end{scriptsize}
\end{resolution}

\subsection{Longueur Solution}
\begin{resolution}
    Des plans plus longs pourraient exister si des actions supplémentaires ou inutiles étaient insérées.
\end{resolution}
\begin{example}
    Ramener un bloc sur la table avant de le replacer sur une autre position :

    \begin{scriptsize}\mycode
        \begin{lstlisting}[
            language={bash}
        ]
    (pick-up b) (put-down b) (pick-up b) (stack b a)
        \end{lstlisting}
    \end{scriptsize}

    Changer l'ordre des blocs temporairement avant de revenir à l'ordre final.
\end{example}
\noindent Ces plans ne sont pas proposés car le planificateur optimise pour trouver une solution minimale en termes de longueur ou de \texttt{Makespan}. Cela est dû à l'utilisation d'heuristiques qui privilégient les solutions les plus efficaces en respectant les contraintes du domaine. Fournir des plans plus longs irait à l'encontre de cette logique d'optimalité.\\

\noindent Le planificateur génère un plan optimal en minimisant la longueur et le temps. Les plans plus longs, bien que possibles, ne sont pas proposés car ils n'ajoutent aucune valeur à la résolution du problème et ne respectent pas les objectifs d'efficacité.
\end{document}
