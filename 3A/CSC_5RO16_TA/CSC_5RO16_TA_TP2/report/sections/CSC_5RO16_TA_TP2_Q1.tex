\documentclass[../CSC_5RO16_TA_TP2.tex]{subfiles}

\begin{document}
\section{Question 1}
% Test the two algorithms RRT and RRT* on this problem by varying the maximum number of iterations. What can you see on the average lengths of the paths ? On the computation times ? Remember to disable the display and to make several experiments to have significant results as these algorithms are stochastic.
% https://github.com/zhm-real/PathPlanning
\noindent Tout d'abbord, il est important de noter que les algorithmes \textcolor{graph_blue}{RRT} et \textcolor{graph_orange}{RRT*} sont des méthodes stochastiques. Cela implique que pour obtenir des résultats significatifs et fiables, plusieurs expérimentations doivent être réalisées. Ces répétitions permettent de lisser les variations inhérentes à la nature aléatoire de ces algorithmes. L'environment 1 est présenté ci-dessous pour chaque algorithme :
\begin{figure}[H]
    \centering
    \begin{subfigure}[b]{0.495\textwidth}
        \centering
        \includegraphics[width=\linewidth]{../src/images/rrt_env_2_0.1_0.0_1500.png}
        \caption{algorithme \textcolor{graph_blue}{RRT}}
        \label{}
    \end{subfigure}\hfill
    \begin{subfigure}[b]{0.495\textwidth}
        \centering
        \includegraphics[width=\linewidth]{../src/images/rrt_star_env_2_0.1_0.0_1500.png}
        \caption{algorithme \textcolor{graph_orange}{RRT*}}
        \label{}
    \end{subfigure}
    \caption{Environment 1}
    \label{}
\end{figure}
\noindent Les résultats obtenus après plusieurs expériences sont synthétisés ci-dessous :
\begin{figure}[H]
    \centering
	\includegraphics[width=1.00\linewidth]{../src/images/average-algorithm-performance_env_2_0.1_0.0_10.png}
	\caption{Performance Moyenne des Algorithmes avec \texttt{step}=2}
	\label{}
\end{figure}
\noindent L'image précédente illustre les résultats obtenus pour les algorithmes \textcolor{graph_blue}{RRT} et \textcolor{graph_orange}{RRT*} exécutes sur l'environmment 1 avec \texttt{step}=2, \texttt{goal\_sample\_rate}=0.1, \texttt{corner\_sample\_rate}=0.0 et \texttt{repetions}=10.\\
\begin{remark}
    Dans ce projet, chaque image aura sur son titre les informations sur l'exécution : \texttt{average-algorithm-performance\_i\_j\_k\_l\_m} :
    \begin{enumerate}[noitemsep]
        \item \texttt{i : environnement} : environnement utilisé;
        \item \texttt{j : step} : pas utilisé;
        \item \texttt{k : goal\_sample\_rate} : goal sample rate utilisé;
        \item \texttt{l : corner\_sample\_rate} : corner sample rate utilisé;
        \item \texttt{m : repetions} : repetions realisées.
    \end{enumerate}
\end{remark}
\noindent La ligne rouge représente le \textbf{chemin optimal théorique} entre le point de départ et le point d'arrivée, calculé en ignorant les obstacles. Cette ligne sert de référence pour évaluer la qualité des solutions générées par les algorithmes.\\

\noindent Même si les résultats pour \texttt{max\_iterations}=375 sont en dessous de cette ligne, ce n'est pas surprenant car, parmi les 10 expériences, seulement 6 chemins ont été trouvés par \textcolor{graph_blue}{RRT} et 3 par \textcolor{graph_orange}{RRT*}.\\

\noindent Les chifres autour de la ligne de 90 mètres représentent respectivement la quantité moyenne d'itérations effectuées, au dessus, et la quantité de chemins trouvés, en dessous.
\begin{remark}
    Les barres illustrent la longeur moyenne des chemins trouvés en mètres avec l'échelle à gauche et les lignes montrent le temps moyen de calcul en secondes avec l'échelle à droite.
\end{remark}
\noindent Au fur et à mesure que le nombre maximale d'itéractions augmente les algorithmes trouvent de chemins plus courts en générale et le temps de calcul augmente. Il faut noter que l'algorithme \textcolor{graph_orange}{RRT*} trouve souvent des chemin plus courts au coût d'un temps de calcul plus significatif.
\end{document}
