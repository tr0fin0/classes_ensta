\documentclass[../CSC_5RO16_TA_TP2.tex]{subfiles}

\begin{document}
\section{Question 4}
% To improve this, modify the rrt.py file to implement a simple variant of the OBRRT [3] algorithm. In this algorithm, the idea is to sample points taking into account the obstacles in order to increase the chances that the tree passes through difficult areas.
% Implement a very simple version in which you will sample a part of the points randomly in the obstacle free area around the corners of the obstacles. To do this, you must modify the function generate_random_node(self, goal_sample_rate). You will need to use the following variables and functions :
% — self.env.obs_rectangle : a list of tuples (x, y, w, h) describing the obstacles : x, y are the coordinates of the bottom left corners of the obstacles, w, h are the width and height of the obstacle
% — self.utils.is_inside_obs(node) : a function that checks if a node is in the obstacle free area
% — np.random.randint(n), np.random.random() and np.random.randn() : functions giving a random integer, random value between 0 and 1 with uniform probability and a random value following a unit gaussian.
% Show the performance variation as a function of the percentage of points sampled using this strategy (from 0% to 100%).
\end{document}
