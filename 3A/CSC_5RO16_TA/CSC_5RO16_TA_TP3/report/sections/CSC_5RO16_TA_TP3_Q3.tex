\documentclass[../CSC_5RO16_TA_TP3.tex]{subfiles}

\begin{document}
\section{Question 3 - Bicyclette, Contrôle Proportionnel Position}
% Vous devez implémenter cette méthode dans la fonction BicycleToPoseControl. Vous pouvez la tester à partir d’une position aléatoire à l’aide de la fonction BiclycleToPose. Réglez les gains du contrôleur pour atteindre rapidement le but en limitant les oscillations. Testez votre solution avec la fonction BiclycleToPoseBenchmark, vous pouvez obtenir un score inférieur à 1800.

\subsection{Description}
\noindent Pour cette question, un contrôleur \textbf{proportionnel} a été implémenté pour un bicycle: pour la position et l'orientation. Ces contrôleur suit les équations présentées ci-dessous:
\begin{enumerate}
    \item \textbf{Position}:
    \begin{equation}
        \upsilon = K_{\rho} \times \underbrace{\sqrt[2]{(x_{G} - x)^2 + (y_{G} - y)^2}}_{\rho}
    \end{equation}
    \item \textbf{Orientation}:
    \begin{equation}
        \phi = K_{\alpha} \times \underbrace{\left(\arctan\left(\frac{y_{G} - y}{x_{G} - x}\right) - \theta\right)}_{\alpha} + K_{\beta} \times \underbrace{\theta_{G} - \left(\arctan\left(\frac{y_{G} - y}{x_{G} - x}\right)\right)}_{\beta < 0}
    \end{equation}
\end{enumerate}

\subsection{Algorithme}
\noindent Ensuite, l'algorithme suivant a été implémenté pour intégrer ces contrôleurs:
\begin{scriptsize}\mycode
    \lstinputlisting[language=Octave, caption=BicycleToPoseControl.m]{../src/BicycleToPoseControl.m}
\end{scriptsize}

\subsection{Résultats}
\noindent Après une série d'essais statistiques, les meilleures performances observées étaient de \textbf{1735.8571}, obtenues avec les paramètres suivants: \texttt{K\_rho = 25}, \texttt{K\_alpha = 5} et \texttt{K\_beta = -3}.\\

\noindent Ces valeurs optimisées ont permis d'atteindre un compromis efficace entre récision et rapidité du contrôle.
\end{document}
