\documentclass[../CSC_5RO16_TA_TP3.tex]{subfiles}

\begin{document}
\section{Question 2 - Bicyclette, Contrôle Proportionnel Point}
% Vous devez implémenter cette méthode dans la fonction BicycleToPointControl. Vous pouvez la tester à partir d’une position aléatoire à l’aide de la fonction BiclycleToPoint. Réglez les gains du contrôleur pour atteindre rapidement le but en limitant les oscillations. Testez votre solution avec la fonction BiclycleToPointBenchmark, vous pouvez obtenir un score inférieur à 1400.

\subsection{Description}
\noindent Pour cette question, un contrôleur \textbf{proportionnels} a été implémenté pour un bicycle : pour la position. Ce contrôleur suit les équations présentées ci-dessous:
\begin{enumerate}
    \item \textbf{Position}:
    \begin{equation}
        \upsilon = K_{\rho} \times \underbrace{\sqrt[2]{(x_{G} - x)^2 + (y_{G} - y)^2}}_{\rho}
    \end{equation}
    \item \textbf{Orientation}:
    \begin{equation}
        \phi = K_{\alpha} \times \underbrace{\left(\arctan\left(\frac{y_{G} - y}{x_{G} - x}\right) - \theta\right)}_{\alpha}
    \end{equation}
\end{enumerate}

\subsection{Algorithme}
\noindent Ensuite l'algorithme suivant a été implémenté pour intégrer ce contrôleur:
\begin{scriptsize}\mycode
    \lstinputlisting[language=Octave, caption=BicycleToPointControl.m]{../src/BicycleToPointControl.m}
\end{scriptsize}

\subsection{Résultats}
\noindent Après une série d'essais statistiques, les meilleurs performances obsrvées étaient \textbf{1349.619}, obtenues avec les paramètres suivants: \texttt{K\_rho = 25} et \texttt{K\_alpha = 5}.\\

\noindent Ces valeurs optimisées ont permis d'atteindre un compromis efficace entre récision et rapidité du contrôle.
\end{document}
