\documentclass[../CSC_5RO12_TA_TP4.tex]{subfiles}

\begin{document}
\section{Question 4}
% Modify the code of the filter to use only the landmark direction in the Kalman correction. Implement a simple undelayed initialization for new landmarks by adding several landmarks along the perception direction with growing covariances. It is not asked to implement the full solution (e.g. the Federated filter and the map pruning described in [1]). Simply update the most likely hypothesis at each step, and prune the hypothesis whose detection probability is below a threshold. Test the approach in an environment with only one landmark for simplicity (i.e. reproduce the scenario of Fig 1).
\noindent Une implémentation préliminaire a été développée dans le code, mais elle n'a pas pu être finalisée. Pendant l'exécution, aucun moyen n'a été trouvé pour effectuer l'élagage (pruning) des différentes covariances au fil de l'exécution. De plus, les estimations des états et des covariances ne correspondaient pas à la référence, rendant le résultat imprécis et peu utile.\\

\noindent Néanmoins, on remarque que cette approche pourrait produire des résultats intéressants dans des situations où les informations sur les références ne sont pas disponibles, comme cela peut être le cas dans une application réelle, bien que cela implique une charge de calcul plus importante.
\end{document}
