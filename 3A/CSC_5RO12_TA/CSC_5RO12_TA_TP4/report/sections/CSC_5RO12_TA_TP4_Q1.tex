\documentclass[../CSC_5RO12_TA_TP4.tex]{subfiles}

\begin{document}
\section{Question 1}
\begin{remark}
    Des modifications ont été faites par rapport à l'algorithme original, notamment avec la création de différentes classes et la définition d'une fonction \texttt{main}.
\end{remark}

\subsection{Structure du Code}
\noindent L'algorithme utilisé dans ce projet suit la structure suivante:
\begin{enumerate}
    \item \texttt{EKF\_SLAM}: Extended Kalman Filter Simultaneous Localization and Mapping sous forme de dataclass, qui contient les méthodes suivantes:
    \begin{enumerate}[noitemsep]
        \item \texttt{F()};
        \item \texttt{G()};
        \item \texttt{H()};
        \item \texttt{calc\_innovation()};
        \item \texttt{compute\_iteration()};
        \item \texttt{extended\_jacobians()};
        \item \texttt{search\_landmarks\_id()};
    \end{enumerate}
\end{enumerate}
\begin{remark}
    Des commentaires et de la documentation ont été ajoutés à l'algorithme pour faciliter sa compréhension et ne seront pas répétés ici.
\end{remark}
\begin{remark}
    Entre chaque exécution, seule une variable a été variée tandis que les autres sont restées inchangées, garantissant ainsi que l'analyse se concentre uniquement sur la variable en question.
\end{remark}
\begin{remark}
    L'application de la fonction \texttt{Utils.convert\_angle()} a été appliquée à tous les angles calculés dans l'algorithme pour garantir que les résultats affichés sur le graphique soient contenus dans $[-\pi, +\pi]$.
\end{remark}
\begin{remark}
    Un vecteur $\widetilde{\mathbf{a}}$ contient des données bruitées.
\end{remark}
\begin{remark}
    Un vecteur $\hat{\mathbf{a}}$ est une prédiction.
\end{remark}
\begin{remark}
    Lors de cette exécution, chaque fois que le robot passait à moins d'une certaine distance, une fermeture de boucle était considérée, et une ligne jaune était ajoutée au moment où cela se produisait pour représenter cette fermeture de boucle.
\end{remark}

\newpage
\subsection{Execution Initiale}
\noindent Après l'implementation des fonctions et variables qui manque au code source, quelques modifications sur le graphique ont été fait et le résultat initiale, utilisé comme benchmark, est donnée par l'image ci-dessous:
\begin{figure}[H]
    \centering
	\includegraphics[width=\linewidth]{../outputs/EKF_SLAM_0.1_1.0_0.1_True_4_10_1_6_2.png}
	\caption{Résultats Initiales}
	\label{}
\end{figure}
\begin{remark}
    Dans ce projet, chaque image aura sur son titre les informations sur l'exécution: \texttt{EKF\_SLAM\_i\_j\_k\_l\_m\_n\_o\_p\_q.png} où:
    \begin{enumerate}[noitemsep]
        \item \texttt{i}: \texttt{dt}: intervalle entre deux prédictions consécutives en secondes;
        \item \texttt{j}: \texttt{v}: velocité tangecielle de mouvement;
        \item \texttt{k}: \texttt{w}: velocité angulaire du mouvement;
        \item \texttt{l}: \texttt{landmarks\_know}: les coordonnées des repères sont-elles connues?
        \item \texttt{m}: \texttt{landmarks\_count}: la quantité de repères dans le scénario;
        \item \texttt{n}: \texttt{observation\_range}: la distance d'observation;
        \item \texttt{o}: \texttt{P\_constant}: constant de la matrice de covariance d'état $\mathbf{P_{k}}$;
        \item \texttt{p}: \texttt{Q\_constant}: constant de la matrice de covariance du bruit du processus $\mathbf{Q_{k}}$;
        \item \texttt{q}: \texttt{R\_constant}: constant de la matrice de covariance du bruit de mesure $\mathbf{R_{k}}$;
    \end{enumerate}
\end{remark}

\subsubsection{Utilization}
\noindent Pour répondre aux questions, l'algorithme suivant a été utilisé:
\begin{scriptsize}\mycode
	\begin{lstlisting}[language=Python]
def execution(...) -> None:
    ...

    plt.suptitle(file_name)
    if save_result: plt.savefig(file_path, dpi=300)
    if show_result: plt.show()

def main():
    for scenario in [...]:
        execution(
            landmarks=scenario['landmarks'],
            v=scenario['v'],
            w=scenario['w'],
            save=True,
            show=True
        )

if __name__ == "__main__":
    main()
	\end{lstlisting}
\end{scriptsize}


\subsection{Scenario 1}
\begin{definition}
    a short loop and a dense map with many landmarks inside the robot perception radius
\end{definition}
\noindent Ci-dessous, sont présentées quelques interactions résultant de la configuration de \textbf{landmarks} du scenario:
\begin{figure}[H]
    \centering
	\includegraphics[width=0.65\linewidth]{../outputs/EKF_SLAM_0.1_1.5_0.2_True_25_10_1_6_2.png}
	\caption{Execution Scenario 1, with \texttt{landmarks\_know = True}}
	\label{}
\end{figure}
\noindent Dans ce cas, il est possible d'observer une oscillation périodique de la variance des coordonnées x et y, due à la rotation de l'ellipse de prédiction. En modifiant son angle, elle change la variation de la coordonnée correspondante. Malgré cela, il est notable que les erreurs restent faibles tout au long du trajet.\\

\noindent On remarque que, juste après les fermetures de boucle, représentées par les lignes jaunes verticales, les erreurs diminuent légèrement pour les coordonnées $x$ et $y$, mais plus nettement pour l'angle $\theta$. Ce phénomène s'explique par le fait que les fermetures de boucle provoquent une recalibration du filtre.\\

\noindent Cependant, il convient de souligner que chaque nouvel obstacle observé entraîne une mise à jour de l'estimation générale de tous les obstacles, améliorant ainsi l'estimation de la carte.
\subsection{Scenario 2}
\begin{definition}
    a long loop and a dense map with many landmarks all along the loop
\end{definition}
\noindent Ci-dessous, sont présentées quelques interactions résultant de la configuration de \textbf{landmarks} du scenario:
\begin{figure}[H]
    \centering
	\includegraphics[width=0.65\linewidth]{../outputs/EKF_SLAM_0.1_1.5_0.1_True_30_10_1_6_2.png}
	\caption{Exécution Scenario 2, with \texttt{landmarks\_know = True}}
	\label{}
\end{figure}
\noindent Dans ce cas, toutes les références ont été placées autour de la boucle, et on observe leur influence sur le résultat. Toutes les coordonnées ont présenté une augmentation significative des erreurs et de la variance, en particulier les coordonnées $x$ et $y$, qui montraient une croissance importante avant 550 secondes.\\

\noindent Pour les coordonnées $x$ et $y$, on note une atténuation du phénomène observé dans le scénario précédent. Les variances affichent une croissance plus marquée et une fréquence d'oscillation réduite. Cela pourrait indiquer que les obstacles plus proches de la trajectoire aident à préserver l'estimation.\\

\noindent Entre 550 et 600 secondes, toutes les coordonnées montrent une diminution substantielle des erreurs et des variances. Cela s'explique par une correction de la trajectoire, car une référence a été identifiée, permettant la mise à jour de la carte et de l'estimation des coordonnées du système.

\subsection{Scenario 3}
\begin{definition}
    long loop and a sparse map with only few landmarks near the start position
\end{definition}
\noindent Ci-dessous, sont présentées quelques interactions résultant de la configuration de \textbf{landmarks} du scenario:
\begin{figure}[H]
    \centering
	\includegraphics[width=0.65\linewidth]{../outputs/EKF_SLAM_0.1_1.5_0.1_True_10_10_1_6_2.png}
	\caption{Exécution Scenario 3, with \texttt{landmarks\_know = True}}
	\label{}
\end{figure}
\noindent Dans ce cas, on remarque l'influence de l'absence de références le long de la trajectoire, puisque, après avoir quitté la région initiale, la variance de toutes les coordonnées augmente continuellement. Cela se traduit par une incertitude croissante concernant la position, ce qui s'explique par l'absence de corrections et une dépendance exclusive à l'odométrie.\\

\noindent Aux alentours de 500 secondes, une référence est détectée pour la première fois depuis le départ. Cela permet une correction des estimations du filtre de Kalman, observable par une diminution brusque et significative des erreurs et des covariances à ce moment-là.
\end{document}
