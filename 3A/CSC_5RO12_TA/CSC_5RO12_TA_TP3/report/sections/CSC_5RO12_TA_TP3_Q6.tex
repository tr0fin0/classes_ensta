\documentclass[../CSC_5RO12_TA_TP3.tex]{subfiles}

\begin{document}
\section{Question 6}
% Simuler un trou de mesures entre t = 250 s et t = 350 s en utilisant la variable notValidCondition et expliquer les résultats.
\noindent Ci-dessous, un trou de mesures simulé entre 250 et 350 secondes:
\begin{figure}[H]
    \centering
    \begin{subfigure}[b]{0.475\textwidth}
        \centering
        \includegraphics[width=\linewidth]{../../outputs/PF_1_1_005_300_low-variance_0-1_2_2_False.png}
        \caption{\texttt{black\_out = False}}
        \label{}
    \end{subfigure}\hfill
    \begin{subfigure}[b]{0.475\textwidth}
        \centering
        \includegraphics[width=\linewidth]{../../outputs/PF_1_1_005_300_low-variance_0-1_2_2_True.png}
        \caption{\texttt{black\_out = True}}
        \label{}
    \end{subfigure}
    \caption{Simulation d'un trou \texttt{black\_out}}
    \label{}
\end{figure}
\noindent Il est à noter que lorsque il y a un trou de mesures entre 250 et 350 secondes peu de changements sur l'estimation peu être apperçu. Si, il y a une augmentation de l'erreur et de la covariance, plus notable à la covariance de $\theta$, mais ce manque de donnès n'a pas beaucoup impacté le résultat du filtre particulaire.\\

\begin{remark}
    Le filtre particulaire reste robuste face à une perte de mesures car il continue d'estimer l'état grâce au modèle prédictif et à la diversité des particules. Lorsque les mesures reviennent, le rééchantillonnage ajuste rapidement les particules, permettant une récupération efficace.
\end{remark}
\end{document}
