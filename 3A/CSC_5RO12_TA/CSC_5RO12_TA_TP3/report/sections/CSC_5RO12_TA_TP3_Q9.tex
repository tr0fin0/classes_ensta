\documentclass[../CSC_5RO12_TA_TP3.tex]{subfiles}

\begin{document}
\section{Question 9}
% Proposer une autre façon de ré-échantillonner les poids, en remplacement de la fonction re_sampling, la coder puis commenter les résultats. Vous pourrez vous aider de [12] (référence slide 36, pdf fourni avec le code).
\noindent Ci-dessous, la méthode de rééchantillonnage \texttt{multinomial} a été choisie pour la fonction \texttt{resample\_method}:
\begin{figure}[H]
    \centering
    \begin{subfigure}[b]{0.475\textwidth}
        \centering
        \includegraphics[width=\linewidth]{../../outputs/PF_1_1_005_300_low-variance_0-1_2_2_False.png}
        \caption{\texttt{resample\_method = 'low-variance'}}
        \label{}
    \end{subfigure}\hfill
    \begin{subfigure}[b]{0.475\textwidth}
        \centering
        \includegraphics[width=\linewidth]{../../outputs/PF_1_1_005_300_multinomial_0-1_2_2_False.png}
        \caption{\texttt{resample\_method = 'multinomial'}}
        \label{}
    \end{subfigure}
    \begin{subfigure}[b]{0.475\textwidth}
        \centering
        \includegraphics[width=\linewidth]{../../outputs/hist_1_1_005_300_low-variance_0-1_2_2_False.png}
        \caption{\texttt{resample\_method = 'low-variance'}}
        \label{}
    \end{subfigure}\hfill
    \begin{subfigure}[b]{0.475\textwidth}
        \centering
        \includegraphics[width=\linewidth]{../../outputs/hist_1_1_005_300_multinomial_0-1_2_2_False.png}
        \caption{\texttt{resample\_method = 'multinomial'}}
        \label{}
    \end{subfigure}
    \caption{Comparison des \texttt{resample\_method}}
    \label{}
\end{figure}
\noindent La méthode originale privilégie les particles à faible variance, ce qui réduit le facteur aléatoire dans la sélection des particules et rend le filtre plus efficace en permettant une convergence plus rapide. Pour une approche plus simple et plus aléatoire, la méthode multinomiale a été choisie, car son implémentation est plus facile à mettre en oeuvre et les particles sont sélectionnées de manière totalement aléatoire.\\

\noindent Ce changement de méthode améliore légèrement l'efficacité du filtr particulaire, car la dégénérescence est réduite, comme on peut le constater dans l'histogramme plus régulier pour la méthode multinomiale. De plus, l'erreur et la covariance diminuent pour les coordonnées x et y avec cette nouvelle méthode, bien que l'angle $\theta$ ne s'améliore pas de manière significative.

\begin{remark}
    D'autres méthoes peuvent être choisies en fonction de l'application. Dans le cas d'une trajectoire relativement stable, comme celle étudiée ici, les deux méthodes offrent des résultats satisfaisants.
\end{remark}
\end{document}
