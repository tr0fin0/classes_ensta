\documentclass[../CSC_5RO12_TA_TP3.tex]{subfiles}

\begin{document}
\section{Question 4}
% Faire varier le bruit de mesure du filtre (matrice REst), qu’observe-t-on ? Expliquer
\noindent Ci-dessous, quelques interactions ont été réalisées en faisant varier le bruit de mesure du filtre:
\begin{figure}[H]
    \centering
    \begin{subfigure}[b]{0.475\textwidth}
        \centering
        \includegraphics[width=\linewidth]{../../outputs/PF_1_1_005_300_low-variance_0-1_2_0-01_False.png}
        \caption{\texttt{R\_constant = 0.01}}
        \label{}
    \end{subfigure}\hfill
    \begin{subfigure}[b]{0.475\textwidth}
        \centering
        \includegraphics[width=\linewidth]{../../outputs/PF_1_1_005_300_low-variance_0-1_2_0-1_False.png}
        \caption{\texttt{R\_constant = 0.1}}
        \label{}
    \end{subfigure}
    \begin{subfigure}[b]{0.475\textwidth}
        \centering
        \includegraphics[width=\linewidth]{../../outputs/PF_1_1_005_300_low-variance_0-1_2_10_False.png}
        \caption{\texttt{R\_constant = 10}}
        \label{}
    \end{subfigure}\hfill
    \begin{subfigure}[b]{0.475\textwidth}
        \centering
        \includegraphics[width=\linewidth]{../../outputs/PF_1_1_005_300_low-variance_0-1_2_25_False.png}
        \caption{\texttt{R\_constant = 25}}
        \label{}
    \end{subfigure}
    \begin{subfigure}[b]{0.475\textwidth}
        \centering
        \includegraphics[width=\linewidth]{../../outputs/PF_1_1_005_300_low-variance_0-1_2_100_False.png}
        \caption{\texttt{R\_constant = 100}}
        \label{}
    \end{subfigure}\hfill
    \begin{subfigure}[b]{0.475\textwidth}
        \centering
        \includegraphics[width=\linewidth]{../../outputs/PF_1_1_005_300_low-variance_0-1_2_250_False.png}
        \caption{\texttt{R\_constant = 250}}
        \label{}
    \end{subfigure}
    \caption{Variance du bruit de mesure \texttt{R\_constant}}
    \label{}
\end{figure}
\noindent Il est à noter que lorsque la constante \texttt{R\_constant} est petite ($\approx \bf{R}_{k} < 1$) ou trop grande ($\approx \bf{R}_{k} > 5$), le résultat du filtre particulaire se dégradent. Dans le premier cas, il n'y a pas assez de variabilité entre les particules pour éxplorer différents configurations, tandis que dans le deuxième cas, une variabilité excessive conduit à une estimation instable, car le filtre prend plus de temps pour procéder au rééchantillonnage et, par conséquent, pour corriger ses estimations.\\

\begin{remark}
    Lorsque le bruit augmente, le filtre perd son efficacité, car il ne corrige pas son estimation et suit simplement l'odométrie.
\end{remark}
\end{document}
