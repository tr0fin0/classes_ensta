\documentclass{article}
\usepackage{tpack}


\title{MO201 - Linux}
\author{Guilherme Nunes Trofino}
\authorRA{217276}
\project{Résumé Théorique}


\begin{document}
\selectlanguage{french}
\maketitle

\newpage\tableofcontents

\section{Introduction}
le point "." est utilise pour considere que un fichier sera cache

cd
ls
mkdir
rmdir
pwd
chmod
find
cp [-r]
mv
rm [-rif]
cat
less
file
touch
cut [-d delim] [-f col]
paste [-d delim]
nl
head -n
tail -n
wc [-lmw]
echo

multiples commands
;
||
|


tr str1 str2
uniq
strings
sort [-n] [-d]
rev


tar -x désarchivage pour .tar
tar -c archivage
tar -cf donner un nom
tar -cz désarchivage pour .tar.gz
tar -cv verborase

compression
gzip
gzip -d


\section{Expressions Rationnelles}
RegEx
    BRE basic regular expressions
    EREextended regular expressions
    PCRE

filtrer
extraire
transformer

. n'importe quel caractère
[] un caractère parmi l'ensemble
[^] un caractère parmi de complémentaire de l'ensemble
anchors

() les groupements ou expressions marquées
| alternance
\1 groupe enregistré
^ début de chaîne
\$ fin de chaîne

* \[0, \[
+ \[1, \[
? \[0,1 \]
    
[:upper:] [A-Z]
[:alpha:] [[:upper:][:lower:]]
[:alnum:] [[:alpha:][:digit:]]

grep utilitaire permettant de faire de la recherche de chaines de caractères principales operations
-E pour utiliser les ERE
-color pour afficher en couleur la chaine trouvées
-o affiche seulement les chaines trouvées
-n affiche
-c
-v

grep [options] pattern [fichier ...]

sed stream editor
utilitaire permettant d'éditer un flux de caractères principales options de la commande
- n'affiche pas les lignes du fluxe
-e commande ajoute commande à la liste des traitements
-i
très utilise pour realizer la substitution de texte
s/motif/remplacement


shell script

change file names that have 2021 in their name with only command
#! /bin/bash

FICHIERS=`ls . | grep 2021`
N=0

for F in $FICHIERS$; do
    echo "Modification de $F ..."
    echo '*Fichier 2021*' | cat - $F > temp;
    mv temp $F;
    N=$((N + 1))
done

echo "$N fichier(s) modifié(s)"


execution  
bash script.sh
chmod u+x script.sh


variables
$# nombre d'arguments  sur la ligne de commande
$@ liste de tous les arguments de la ligne de commande
$0 nom du script
$1, $2, .... premier, second argument dela ligne de commande
\end{document}