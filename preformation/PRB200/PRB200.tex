\documentclass{article}
\usepackage{tpack}


\title{PRB200 - Probabilité}
\author{Guilherme Nunes Trofino}
\authorRA{217276}
\project{Résumé Théorique}


\begin{document}
\selectlanguage{french}
\maketitle

\newpage\tableofcontents

\section{Introduction}
une expérience aléatoire 
$ (\Omega, \mathbb{F}, \mathbb{P})$
Omega l'espace d'états pour être de différent natures
F l'ensemble des événements dans l'espace
P

F est un sous ensemble de Omega  
se Omega est fini, dénombrable, F = P(Omega) tous les subsections sont des événements
se Omega est fini, non dénombrable, F vérifie les propriétés d'une tribu F est une tribu ssi (iff) Omega in F and $A \in F$ implique $nA \in F$ and $(A_n)_{n\in\mathbb{N}} \in F$ implique $U_{n\in\mathbb{B}}A_n \in F$ exercice 7 par exemple

%% TODO review venn diagrams terms in french
%% TODO decrire chaque lettre

exercice 1:
on lance deux dés distinguables et on observe les points donnés par chacun d'eux:
$\Omega = \{ 1,2,3,4,5,6 \} \times \{ 1,2,3,4,5,6 \} = \{ (i,j); i \in \{1,2,3,4,5,6\}; j \in \{ 1,2,3,4,5,6 \} \}$
\\\\
exercice 2:
on lance deux dés identiques et on observe les points donnés par chacun d'eux
l'ensemble des paires
$\Omega = \{ \{i,j\}; i \in \{1,2,3,4,5,6\}; j \in \{ 1,2,3,4,5,6 \} \}$
\\\\
exercice 3:
grand A petit a
Aa est identique à aA
$\Omega = \{ \{ aa \}; \{ aA \}; \{ AA \} \}$
\\\\
exercice 4
on distribue n cartes différentes à r joueurs pas nécessairement le même nombre à chacun, 
n=4 r=3
carte 1 joueur 2
carte 2 joueur 3
carte 3 joueur 1
carte 4 joueur 2
(2,3,1,2) le joueur pour chaque carte donc on peut y écrire:
$\Omega = \{ (\omega_1, \omega_2, ..., \omega_n); \omega_i \in \{ 1, ..., n \}, 1 \leq i \leq n \}$
$\omega_i$ numéro du joueur a distributé la carte
\\\\

exercice 5
on demontre la quantité de 1 e 100 a l'aquele en i, en lui associe $n_i \in \mathbb{N}$ le numeuraux d'heures sans défaillance
$\Omega = \{ (n_1, n_2, ..., n_{100}) \in \mathbb{N} i\in\{1, ..., 100\}\} = \mathbb{N}^{100}$
\\\\

exercice 6
on observe les instants d'émission de particules par un corps radioactif
l'ensemble des suites croissantes parce qu'on n'a pas beaucoup d'informations donc on doit prendre un defintion plus generale
$\Omega = \{ (t_{n})_{n \in \mathbb{N}} ; t_{n} > 0 ; t_{n} \leq t_{n+1}\}$
c'est n'est pas la seul façon de le faire mais c'est la plus générale pour le faire. avec cette notation on peut écrire une suite croissante sans trop de problèmes
%% TODO review how french describe series (notation and operations)
\\\\

exercice 7
on observe la consommation instantanée d'électricité d'une ville au cours du temps
l'ensemble des functions defines pour la parte naturelle
$\Omega = \{ f : \mathbb{R}_{+} \to \mathbb{R}_{+} \}$ où pour $f \in \Omega, f(t) =$ consommation de la ville à l'instant t de temps $t\geq 0$

on peut voir des espaces boitant espace denombrable fine
espace de markov

l'evenement impossible avec le ensemble vide
levenement contre est discrite comment le compleméntaire de A avec la notation $A^{C}$
un événement implique le autre si quand A se produit B se produit et quand B se produit A se produit donc A est égale à B  
\end{document}