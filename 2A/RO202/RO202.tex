\documentclass{article}
\usepackage{C:/Users/guitr/Documents/git_repositories/tpack/tpack}
% \usepackage{C:/Users/Admin-PC/Documents/git_repository/tpack/tpack}
% \usepackage{/home/tr0fin0/Documents/git_repositories/tpack/tpack}
% \usepackage{/home/Documents/git_repositories/tpack/tpack}
\usetikzlibrary{decorations.pathreplacing,calligraphy}

\title{RO202 - Recherche Opérationnelle}
\project{Résumé Théorique}
\author{Guilherme Nunes Trofino}
\authorRA{2022-2024}


\makeatletter
\begin{document}\selectlanguage{french}
\maketitle
\setlength{\parindent}{0pt}

\newpage\tableofcontents

\section{Introduction}
\subfile{C:/Users/guitr/Documents/git_repositories/classes_ensta/intro.tex}
% \subfile{C:/Users/Admin-PC/Documents/git_repository/classes_ensta/intro.tex}
% \subfile{/home/tr0fin0/Documents/git_repositories/classes_ensta/intro.tex}
% \subfile{/home/Documents/git_repositories/classes_ensta/intro.tex}

\subsection{Information Matier}
\paragraph{Référence}Dans cette matière le but sera de comprendre comment une Système d'Exploitation marche.

ensemble des méthodes et techniques rationnelles orientées vers la recherche du meilleur choix

optimisation combinatoire: https://en.wikipedia.org/wiki/Combinatorial_optimization

objectif du étude
maximiser ou minimiser une fonction objectif tout en respectant un ensemble de contraintes

problème discrete et continuos

optimisation des graphes
    arbre couvrant
    chemin
    flot
programmation linéaire
programmation linéaire
    en nombres entiers


definition des graphes
    des points et des traots ou des flèches
    une relation binaire
    representation abstre des reseaux

    G=(V,A)
        ensemble sommets
        ensemble d'arcs

        sucesseur et predecesseurs
        gama(v)= {successeurs du somme}
        gama-1(v)= {predecesseurs du somme}

    on utikise des graphes simples, pas paralelel

    graphe value cosidere


circuit
chemin
racine existe un chemin de r  tut autre
degree

chaine sequence d'aretes telle qu'il existe un sequence de sommets telle que ei (ajouter la math)
exemple de la etoile n'est pas une chaine

voisinage les sommestx et y sont dits voisins si [xy] in E
N(x) voisins de x
d(x) = |N(x)|

cycle élémentaire chaine dont les deux extremites coincident

(faire le memse dessins et chapitres avec les donnes orientes et pas orientes avec chaqu'une avec uns section)

hypotheses pour la suite
les graphes sont simplesles cycles sont elementaires
les graphes 

connexite relation soit x et y deux sommets d'un graphe G=(V, A) xRy iff x y sont relies par une chaine

composante connexe r est une relation d'equivalence dont les classes d'équivalences sont appélles compossantes connexes

arbre
graphe ----- connexe et ---- sans cycles
theoreme avec des explications téoriques plus précises
forêt
graphe ----

arborescence
arbre possédant une racine r telle que r est reliée à tout v in V par un chemin unique
arborescence = "arbre enraciné"= "arbre" en informatique

arbre couvrant de poids minimal
    graphe non orienté valué
    objets à relier
    liens possibles
    longueur du lien
sélectionner des arêtes d'un graphe orienté valué G=(V, E, p) afin de former un arbre
solution optimale arbre couvrant de poids minimal
arbre praphe sans cycle et connexe
couvrant passant par tous les sommetsminimal avoir les poids les moins grand


algorithme de kruskal
données

résultat

complexe O(m log m)

p(v) leq p(u)
p(v) geq p(u)
docn p(v) eq p(u)

est un algorithme glouton
    a chaque étape faire le choix le plus intéressant à cet instant et ne plus le remettre en question
        facile 
        rapide
        rarement optimale
            l'arbre couvrant de poids minimal est une exception
            algorithme dit heuristique
        choix gouton choix localement optimal et pas globalment 

    detecter des cycles c'est une fonction important et essential pou rle fonctionnement des algoorithmes



complexité du algorithme
problème facile à résoudre sera résoudre avec une complexité polynomiale au maximale
on ne connaît aucun algorithme permettant de résoudre P en un nombre polynomial d'étapes


voyageur de commerce
comment passer un e fois par chaque ville tout en minimisant la longueur totale parcourue
on cherche un cycle hamiltonien de valeur minimale
passant par tous les sommets


cheminement
définir les problèmes différents possibles


circuit absorbant: somme des poids est négatif
théorème il existe un chemin de longueur minimale finie de r à tous les sommets du graphe
si et seulement si r est une racine du praphe et le praphe ne contient pas de circuit absorbant

algorithme de dijkstra
construire une arborescence H(V, A2) dont r est la racine et correspondant au plus court chemin entre r et les autres sommets
le plus court chemin entre r et son sommet le plus proche v est p(r,v)
même raisonnement pour le sommet le plus proche de r ou v
on répète cette idée jusqu'à ce que
    problème 1: le sommet cible soit atteint
    problème 2: tous les sommets soient atteints

soient les applications pred(x) et pi (x)
    où pred est le prédécesseur de x sur le meilleur chemin connu de r à x
    longueur du meilleur chemin connu entre r et x

    les nombres négatifs ne marchent pas pour cette algorithme
    minimisation avec les valeurs négatives est impossible



algorithme de Bellman
just pour des graphes sans circuits
tri topologique des sommets d'un grapheordre toatl sur V tel que i précède j pour tout ij in A
propriété on peut toujours trier topologiquement les osmmets d'un graphe sans circuit
algorithme
    le sommet de départ a pour valeur 0
    à chaque itération: on value un sommet


algorithme de roy-warshall-floyd
problèmes 1, 2 et 3

trouver le cheminement minimal entre toute paire de sommets pas de contraintes sur le graphe
chercher chaque algoritme pour bien le compreendre et faire une définition précis de comment il marche et comment il est implement sur python


faire le tableau de comparasion de chaque algorithme pour mieux visualiser comment chaque algorithme marche


définition de graphy.py


\end{document}