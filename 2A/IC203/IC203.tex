\documentclass{article}
\usepackage{C:/Users/guitr/Documents/git_repositories/tpack/tpack}
% \usepackage{C:/Users/Admin-PC/Documents/git_repository/tpack/tpack}
% \usepackage{/home/tr0fin0/git_repositories/tpack/tpack}


\title{IC203 - Reseaux d'Information}
\project{Résumé Théorique}
\author{Guilherme Nunes Trofino}
\authorRA{2022-2024}


\makeatletter
\begin{document}\selectlanguage{french}
\maketitle


\newpage\tableofcontents

\section{Introduction}
\subfile{C:/Users/guitr/Documents/git_repositories/classes_ensta/intro.tex}
% \subfile{C:/Users/Admin-PC/Documents/git_repository/classes_ensta/intro.tex}
% \subfile{/home/tr0fin0/git_repositories/classes_ensta/intro.tex}


\subsection{Information Matier}
\paragraph{Référence}



\section{ARP}
\begin{definition}
    Mapping of known IP Address into an unknown MAC Address in a table stored in which device.
\end{definition}


\subsection{1}

\subsection{2}

\section{Travail Dirigé}

\subsection{Séance 1}

\newpage\subsection{Séance 2}

\newpage\subsection{Séance 3}

\newpage\subsection{Séance 4}

\newpage\subsection{Séance 5}
\begin{exercise}
    On considère qu'un dispositif a la table de connections suivante:
    \begin{table}[H]
        \centering\begin{tabular}{lll | ll}
            Destination   & Mask Gateway    & Device        & First Address & Last Address\\
            \hline
            default       & 0.0.0.0         & 137.194.192.14 \\
            127.0.0.1     & 255.255.255.255 & 127.0.0.1      \\
            \hline
            137.194.160.0 & 255.255.254.0   & 137.194.160.3 & 137.194.160.1 & 137.194.161.255\\
            137.194.168.0 & 255.255.254.0   & 137.194.168.1 & 137.194.168.1 & 137.194.169.255\\
            137.194.192.0 & 255.255.254.0   & 137.194.192.1 & 137.194.192.1 & 137.194.193.255\\
            137.194.200.0 & 255.255.254.0   & 137.194.200.1 & 137.194.200.1 & 137.194.201.255\\
            137.194.204.0 & 255.255.254.0   & 137.194.204.1 & 137.194.204.1 & 137.194.205.255\\
            \hline
        \end{tabular}
    \end{table}

    \begin{remark}
        On considère l'adresse $137.194.192.22$ comme la \textbf{Notation Décimale} pointée.
    \end{remark}
    \begin{remark}
        On considère l'adresse $000.000.000.000$/x. x est la \textbf{Masque d'Adressage} e représente combien de bits de l'adresse sont immutables à compter du MSB en direction au LSB.
    \end{remark}
    \begin{remark}
        On considère que le premier adresse sera \textbf{l'Adresse du Sous-Réseau}.
    \end{remark}
    \begin{remark}
        On considère que le dernier adresse sera \textbf{l'Adresse de Diffusion}, Broadcast.
    \end{remark}

    Ça donne le diagramme suivant:
    \begin{figure}[H]
        \centering\begin{tikzpicture}[]
            % modules
            \node[module] (R0) {137.194.192.14};
            \node[module, right= of R0] (SR0) {137.194.192.0/23};
            
            \node[module, above right=5mm of SR0] (SR1) {137.194.200.0/23};
            \node[module, above=5mm of SR0] (SR2) {137.194.204.0/23};

            \node[module, below right=5mm of SR0] (SR3) {137.194.168.0/23};
            \node[module, below=5mm of SR0] (SR4) {137.194.160.0/23};
            
            % connections
            \foreach \i in {1,2,3,4}
                \draw[<->] (SR0)--(SR\i);
            \draw[<->] (R0)--(SR0);
        \end{tikzpicture}
    \end{figure}

    Cette machine cherche à joindre les équipements: 137.194.160.85, 137.194.161.122, 137.194.204.22 et 137.194.20.10.\\

    On note que les adresses 137.194.160.85, 137.194.161.122 et 137.194.204.22 sont dans l'intervale d'adresses valables pour le sous-réseau correspondant. Dans ce-cas là on a le premier cas de requête ARP.\\

    On note que l'adresses 137.194.20.10 n'est pas dans l'intervale d'adresses valables pour les sous-réseaux. Dans ce-cas là on a le deuxième cas de requête ARP.
\end{exercise}

\end{document}