\documentclass{article}
\usepackage{../../../../tpack/document/tpack}


\title{IC203 - Reseaux d'Information}
\project{Travail Dirige 4 - Résolution}
\author{Guilherme Nunes Trofino}
\authorRA{2022-2024}


\makeatletter
\begin{document}\selectlanguage{french}
\maketitle
\setlength{\parindent}{0pt}

\newcommand{\adressTable}[7]{
    \begin{table}[H]
        \centering\begin{tabular}{c|ll}
            \multirow{5}{*}{#1} & 193.215.124.#2/#7 & sous-réseau\\
            \cline{2-3}
            & 193.215.124.#3 & \multirow{2}{*}{adresses disponibles}\\
            & 193.215.124.#4 &\\
            \cline{2-3}
            & 193.215.124.#5 & broadcast\\
            \cline{2-3}
            & 255.255.255.#6 & netmask\\
        \end{tabular}
    \end{table}
}



\section*{Exercice 1}
\noindent Dans ce exercice on étudiera l'Adressage IP.\\

Les assurances Mondass s'installent dans leurs nouveaux locaux de Moulinsart, qui viennent d'être câblées. Il faut installer 150 machines réparties de cette manière:
\begin{enumerate}
    \item \textbf{Bâtiment A}:
    \begin{enumerate}[noitemsep]
        \item \texttt{1er Étage}: 30;
        \item \texttt{2ième Étage}: 25;
        \item \texttt{3ième Étage}: 50;
    \end{enumerate}
    
    \item \textbf{Bâtiment B}:
    \begin{enumerate}[noitemsep]
        \item \texttt{1er Étage}: 15;
        \item \texttt{2ième Étage}: 30;
    \end{enumerate}
\end{enumerate}
Le directeur du service informatique propose de connecter tous les équipements vers des commutateurs Ethernet au niveau d'un local technique pour chaque étage.\\

Il a obtenu la classe C 193.215.124.0 de son fournisseur de service. Il pense que le plus simple serait de segmenter le réseau en sous-réseaux correspondant chacun à un étage de chaque bâtiment.


\subsection*{Question 1}
\begin{exercise}
    Quel est l'avantage de cette division en sous-réseaux ?
\end{exercise}
\begin{resolution}
    C'est une division facile à comprendre car si on connaît le sous réseau on connaît par consequence sa localisation physique dans le terrain.
\end{resolution}

\subsection*{Question 2}
\begin{exercise}
    Que pensez vous de cette solution « Un étage = Un sous-réseau » ?
\end{exercise}
\begin{resolution}
    Il semble une bonne solution, il faut essayer un plan d'adressage pour valider si c'est viable et faisable avec le réseau classe C propose.
\end{resolution}

\newpage
\subsection*{Question 3}
\begin{exercise}
    Proposez un plan d'adressage pour les équipements.
\end{exercise}
\begin{resolution}
    On suppose que c'est mask variable. En suite il faut recalculer la quantité d'adresses nécessaires pour chaque sous-réseau:
    \begin{enumerate}
        \item \textbf{Bâtiment A}:
        \begin{enumerate}[noitemsep]
            \item \texttt{A1)}, 33 adresses nécessaires: sous-réseau de 64 adresses;
            \item \texttt{A2)}, 28 adresses nécessaires: sous-réseau de 32 adresses;
            \item \texttt{A3)}, 53 adresses nécessaires: sous-réseau de 64 adresses;
        \end{enumerate}
        
        \item \textbf{Bâtiment B}:
        \begin{enumerate}[noitemsep]
            \item \texttt{B1)}, 18 adresses nécessaires: sous-réseau de 32 adresses;
            \item \texttt{B2)}, 33 adresses nécessaires: sous-réseau de 64 adresses;
        \end{enumerate}
    \end{enumerate}
    Il faut considérer 3 adresses supplémentaires pour chaque sous-réseau:
    \begin{enumerate}[noitemsep]
        \item 1x \textbf{adresse routeur}: identification machine;
        \item 1x \textbf{adresse broadcast}: communication en diffusion;
        \item 1x \textbf{adresse sous-réseau}: identification réseau;
    \end{enumerate}
    \begin{remark}
        L'inclusion d'une adresse de sous-réseau n'est pas toujours obligatoire, c'est une recommendation.
    \end{remark}
    Après il faut choisir la taille de MASK, et par consequence, la quantité totale d'adresses disponibles pour comporte chaque sous-réseau:
    \begin{enumerate}
        \item \textbf{Bâtiment A}:
        \begin{enumerate}[noitemsep]
            \item \texttt{A1)}: sous-réseau de 64 adresses;
            \item \texttt{A2)}: sous-réseau de 32 adresses;
            \item \texttt{A3)}: sous-réseau de 64 adresses;
        \end{enumerate}
        
        \item \textbf{Bâtiment B}:
        \begin{enumerate}[noitemsep]
            \item \texttt{B1)}: sous-réseau de 32 adresses;
            \item \texttt{B2)}: sous-réseau de 64 adresses;
        \end{enumerate}
    \end{enumerate}
    \begin{remark}
        On choisit la puissance de 2 plus grand que la quantité nécessaire d'adresses la plus proche. On peut utiliser le tableau suivant:
        \begin{table}[H]
            \centering\begin{tabular}{rlll}
                sous-réseau & mask & quantité adresses & quantité sous-réseau\\
                \hline
                193.215.124.000/24 & 24 & 256 & 1\\
                193.215.124.128/25 & 25 & 128 & 2\\
                193.215.124.192/26 & 26 & 64  & 4\\
                193.215.124.224/27 & 27 & 32  & 8\\
                193.215.124.240/28 & 28 & 16  & 16\\
                193.215.124.248/29 & 29 & 8   & 32\\
                193.215.124.252/30 & 30 & 4   & 64\\
                193.215.124.254/31 & 31 & 2   & 128\\
                193.215.124.255/32 & 32 & 1   & 256\\
                \hline 
            \end{tabular}
            \caption{Relation de Sous-Réseaux}
        \end{table}
    \end{remark}
    Après il faut ordonner comment les sous-réseaux seront adresses. On commence avec les réseaux de plus grand taille à cause de la mask. Au fur et a mesure que des nouveaux adresses sont ajoutés on peut seulement augmenter le valeur de la masque. On considère l'ordre suivant: A1, A3, B2, A2, B1.
    \begin{remark}
        Quand il y a des sous-réseau avec la même taille l'ordre est arbitraire.
    \end{remark}  
    De cette façon on propose l'adressage suivante:
    \adressTable{A1}{000}{001}{062}{063}{192}{26}
    \begin{remark}
        Par convention on considère que l'adresse de sous-réseau sera le première et l'adresse de broadcast sera le dernière. 
    \end{remark}
    \adressTable{A3}{064}{065}{126}{127}{192}{26}
    \begin{remark}
        Quand un sous-réseau termine une autre commence, les adresses sont séquentielles. 
    \end{remark}
    \adressTable{B2}{128}{129}{190}{191}{192}{26}
    \begin{remark}
        L'adresse finale sera: adresse initiale plus taille du sous-réseau moins 1 car le premier adresse est inclue.
    \end{remark}
    \adressTable{A2}{192}{193}{222}{223}{224}{27}
    \adressTable{B1}{224}{225}{254}{255}{224}{27}

\end{resolution}

\section*{Exercice 2}
Le directeur des assurances Mondass, vous confie la mission de proposer un plan d'adressage pour la migration de son entreprise à l'IP. L'adresse 193.252.0.0/16 sera utilisée pour l'Intranet de l'entreprise:
\begin{itemize}
    \item Cette entreprise comprend 4 sites distribués sur toute la France, mais d'autres sites sont prévus en Europe par le biais d'échanges de capitaux ou d'acquisitions d'entreprises.  Un maximum de 4 sites est à prévoir dans les années à venir. Si la croissance s'accentue, une nouvelle plage d'adresse sera attribuée.
    \item Après enquête, il apparaît que les différents sites contiennent entre 300 et 2000 machines, toutes équipées de cartes Ethernet.
    \item Les constructeurs vous recommandent de ne pas placer plus de 100 équipements environ par sous-réseau IP.
    \item Finalement, il est nécessaire de donner aux différents sous-réseaux appartenant à un même site un préfixe d'adresse commun de manière à simplifier le routage dans le réseau fédérateur permettant d'interconnecter les différents sites.
\end{itemize}
\subsection*{Question 1}
\begin{exercise}
    Pour quelles raisons vous est-il recommandé de ne pas placer plus de 100 machines par sous-réseau IP ?
\end{exercise}
\begin{resolution}
    C'est recommandé de ne pas placer plus de 100 machines à cause des perturbations de réseau et la possibilité de interference quand plusieurs machines se communiquent.
\end{resolution}
    

\subsection*{Question 2}
\begin{exercise}
    Proposez un plan d'adressage pour l'entreprise (site, sous-réseau, machine). Utilisez la notation CIDR (id est 193.252.32.0/19, 193.252.33.0/24, etc.). Précisez en particulier l'adresse des machines A, B, C, D, E, F ci-dessous.
\begin{itemize}[noitemsep]
    \item A : Machine 25, sous-réseau 2, site 1;
    \item B : Machine 26, sous-réseau 2, site 1;
    \item C : Machine 52, sous-réseau 10, site 1;
    \item D : Machine 23, sous-réseau 10, site 2;
    \item E : Machine 4, sous-réseau 11, site 3;
    \item F : Machine 4, sous-réseau 11, site 4;
\end{itemize}
\end{exercise}
\begin{resolution}
    Tout d'abord il faut considère les contraintes du système:
    \begin{enumerate}[rightmargin=\leftmargin]
        \item \textbf{adresse entreprise}: on utilisera 193.252.0.0/16 comme adresse du système;
        \item \textbf{sites}:
        \begin{enumerate}[noitemsep, rightmargin=\leftmargin]
            \item 4x installes;
            \item 4x prévus;
        \end{enumerate}
        Dans ce cas on considère qu'il aura 8 sites au totale.
        \begin{remark}
            Il faut au moins 3 bits pour les sites du système, car $2^3 = 8$.
        \end{remark}

        \item \textbf{sous réseaux}:
        \begin{enumerate}[noitemsep, rightmargin=\leftmargin]
            \item chaque sous réseau doit avoir au plus 100 machines;
        \end{enumerate}

        \item \textbf{machines}:
        \begin{enumerate}[noitemsep, rightmargin=\leftmargin]
            \item \texttt{minimum}: 300 machines / site;
            \item \texttt{maximum}: 2000 machines / site;
        \end{enumerate}
        Dans ce cas on a deux options pour codifier les machines d'un sous réseau:
        \begin{itemize}[noitemsep, rightmargin = \leftmargin]
            \item utilise \texttt{6 bits} qui donne 64 adresses machines et 32 sous-réseaux / site, 5 bits;
            \item utilise \texttt{7 bits} qui donne 128 adresses machines et 16 sous-réseaux / site, 4 bits;
        \end{itemize}
        \begin{remark}
            Dans le deux cas la quantité totale d'adresses est 2048, en accord avec le numéro maximale d'un sous-réseau.
        \end{remark}
        Dans ce cas on considère qu'il aura 7 bits pour codifier les machines dans un sous-réseau et par consequence 4 bits pour codifier les sous-réseaux.
    \end{enumerate}
    De cette façon-ci on aura utilise $3+4+7=14$ bits pour faire du codage du système mais comme on doit utiliser 16 bits il manque 2 bits dans la proposition. On considère donc qu'il aura 4 bits pour codifier les sites et 5 bits pour codifier les sous-réseaux:
    \begin{definition}
        Proposition de codage pour ce réseau sera:
        \begin{equation}
            \boxed{
                193.252.\underbrace{0000}_{\text{site}}\;\underbrace{0000.0}_{\text{sous-réseau}}\;\underbrace{0000000}_{\text{machine}}
            }
        \end{equation}
        Où:
        \begin{center}
            \begin{minipage}[t]{0.3\textwidth}
                \begin{table}[H]
                    \centering\begin{tabular}{rl}
                        \multicolumn{2}{c}{\textbf{site}}\\
                        \hline
                        1 & 0000\\
                        2 & 0001\\
                        3 & 0010\\
                        4 & 0011\\
                        ... & ...\\
                        \hline
                    \end{tabular}
                \end{table}
            \end{minipage}
            \begin{minipage}[t]{0.3\textwidth}
                \begin{table}[H]
                    \centering\begin{tabular}{rl}
                        \multicolumn{2}{c}{\textbf{sous-réseau}}\\
                        \hline
                        1 & 00000\\
                        2 & 00001\\
                        3 & 00010\\
                        4 & 00011\\
                        ... & ...\\
                        \hline
                    \end{tabular}
                \end{table}
            \end{minipage}
            \begin{minipage}[t]{0.3\textwidth}
                \begin{table}[H]
                    \centering\begin{tabular}{rl}
                        \multicolumn{2}{c}{\textbf{machine}}\\
                        \hline
                        1 & 0000000\\
                        2 & 0000001\\
                        3 & 0000010\\
                        4 & 0000011\\
                        ... & ...\\
                        \hline
                    \end{tabular}
                \end{table}
            \end{minipage}
        \end{center}
    \end{definition}
    \begin{remark}
        On considère que le site, sous-réseau ou machine N désiré sera donne pour l'équivalent binaire de N-1.
    \end{remark}
    Avec cette définition on aura:
    \begin{itemize}[noitemsep]
        \item $\boxed{\text{A} : 193.252.0000\;0000.1\;0011001 = 193.252.000.153}$ pour machine 25, sous-réseau 2 et site 1;
        \item $\boxed{\text{B} : 193.252.0000\;0000.1\;0011010 = 193.252.000.154}$ pour machine 26, sous-réseau 2 et site 1;
        \item $\boxed{\text{C} : 193.252.0000\;0100.1\;0110100 = 193.252.004.180}$ pour machine 52, sous-réseau 10 et site 1;
        \item $\boxed{\text{D} : 193.252.0001\;0100.1\;0010111 = 193.252.020.151}$ pour machine 23, sous-réseau 10 et site 2;
        \item $\boxed{\text{E} : 193.252.0010\;0101.0\;0000100 = 193.252.037.004}$ pour machine 4, sous-réseau 11 et site 3;
        \item $\boxed{\text{F} : 193.252.0011\;0101.0\;0000100 = 193.252.053.004}$ pour machine 4, sous-réseau 11 et site 4;
    \end{itemize}
\end{resolution}

\subsection*{Question 3}
\begin{exercise}
    Les sites 2, 3 et 4 sont interconnectés par des liaisons spécialisées au site 1 (maison mère à Moulinsart). Ces liaisons se terminent au niveau d'un routeur, comme le montre la figure suivante. Représenter la table de routage de ce routeur avec le format suivant :
    \begin{table}[H]
        \centering\begin{tabular}{clll}
            (Site) & Préfixe d'adresse & Masque & Interface de sortie\\
            \hline
            1 & 193.252.... & 255.255.... & I0\\
            2 & 193.252.... & 255.255.... & I1\\
            3 & 193.252.... & 255.255.... & I2\\
            4 & 193.252.... & 255.255.... & I3\\
            ... & ... & ... & ...\\
            \hline
        \end{tabular}
    \end{table}
\end{exercise}
\begin{resolution}
    Avec la proposition de routage on aura la table suivante:
    \begin{table}[H]
        \centering\begin{tabular}{clll}
            (Site) & Préfixe d'adresse & Masque & Interface de sortie\\
            \hline
            1 & 193.252.00/20 & 255.255.240.0 & I0\\
            2 & 193.252.16/20 & 255.255.240.0 & I1\\
            3 & 193.252.32/20 & 255.255.240.0 & I2\\
            4 & 193.252.48/20 & 255.255.240.0 & I3\\
            ... & ... & ... & ...\\
            \hline
        \end{tabular}
    \end{table}
    \begin{remark}
        Chaque proposition aura un résultat différent et donc pour comparer il faut vérifie l'algorithme.
    \end{remark}
\end{resolution}
\end{document}