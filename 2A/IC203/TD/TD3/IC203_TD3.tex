\documentclass{article}
\usepackage{../../../../tpack/document/tpack}


\title{IC203 - Reseaux d'Information}
\project{Travail Dirige 3 - Résolution}
\author{Guilherme Nunes Trofino}
\authorRA{2022-2024}


\makeatletter
\begin{document}\selectlanguage{french}
\maketitle

\newcommand{\tablesCommutation}[3]{
    \begin{center}
        \begin{minipage}[t]{0.3\textwidth}
            \begin{table}[H]
                \centering\begin{tabular}{r|l}
                    \multicolumn{2}{l}{S1}\\
                    port & MAC\\
                    \hline
                        #1
                    \hline
                \end{tabular}
            \end{table}
        \end{minipage}
        \begin{minipage}[t]{0.3\textwidth}
            \begin{table}[H]
                \centering\begin{tabular}{r|l}
                    \multicolumn{2}{l}{S2}\\
                    port & MAC\\
                    \hline
                        #2
                    \hline
                \end{tabular}
            \end{table}
        \end{minipage}
        \begin{minipage}[t]{0.3\textwidth}
            \begin{table}[H]
                \centering\begin{tabular}{r|l}
                    \multicolumn{2}{l}{S3}\\
                    port & MAC\\
                    \hline
                        #3
                    \hline
                \end{tabular}
            \end{table}
        \end{minipage}
    \end{center}
}


\section*{Exercice 1}
\noindent Dans ce exercice on étudiera la Commutation Ethernet.

\subsection*{Question 1}
\begin{definition}
    On considère qu'un \textbf{hub} connecte des équipements et envoie toujours des signaux de diffusion d'information qui cause collisions de données.  
\end{definition}

\begin{definition}
    On considère qu'un \textbf{commutateur} implémente communication de couche 1 et 2 et permet de connecter des différents supports physiques en isolant les domaines de collision.

    \begin{remark}
        On considère qu'un commutateur à 2 ports est un \textbf{pont} qui permet de connecter les réseaux internet en fonction de filtrage de signaux.
    \end{remark}
    Nomme \textbf{switch} en anglais.
\end{definition}

\subsection*{Question 2}
\begin{definition}
    Une adresse MAC est également appelée adresse matérielle ou adresse Ethernet. C'est un identifiant unique et propre à la carte réseau de l'ordinateur. 
    \begin{remark}
        Généralement le format XX.XX.XX.XX.XX.XX, chaque X représentant un chiffre ou une lettre de A à F, numéros hexadécimales.
    \end{remark}
\end{definition}

\newpage
\subsection*{Question 3}
\noindent On considère la structure suivante pour les prochains questions:
\begin{exercise}
    On considère le réseau d'entreprise décrit dans la Figure suivant:
    \begin{figure}[H]
        \centering\begin{tikzpicture}[]
            \node[module, minimum width=15mm, minimum height=15mm, label={[right=8mm]{S1}}] (sw1) {};
            \node[module, minimum width=15mm, minimum height=15mm, label={[right=8mm]{S2}}, above=5mm of sw1] (sw2) {};
            \node[module, minimum width=15mm, minimum height=15mm, label={[right=8mm]{S3}}, above=5mm of sw2] (sw3) {};

            \node[module, left= of sw3] (a1) {aa:bb:cc:dd:ee:01 (A1)};
            \node[module, left= of sw2] (a2) {aa:bb:cc:dd:ee:02 (A2)};
            \node[module, left= of sw1] (a3) {aa:bb:cc:dd:ee:03 (A3)};
            \node[module, right= of sw1] (a4) {aa:bb:cc:dd:ee:04 (A4)};

            \draw[-] (sw1) node[above=-5mm of sw1]{U}  -- (sw2) node[below=-5mm of sw2] {U2};
            \draw[-] (sw2) node[above=-5mm of sw2]{U1} -- (sw3) node[below=-5mm of sw3] {U};
            \draw[-] (a1) -- (sw3) node[left=-6mm of sw3] {02};
            \draw[-] (a2) -- (sw2) node[left=-6mm of sw2] {03};
            \draw[-] (a3) -- (sw1) node[left=-6mm of sw1] {01};
            \draw[-] (a4) -- (sw1) node[right=-6mm of sw1] {19};
        \end{tikzpicture}
    \end{figure}
    Ce réseau comprend quatre machines et trois commutateurs Ethernet. Les adresses Ethernet de ces équipements sont donnes à l'image. À fin de simplification, on utilisera le nom de machine au lieu de son adresse MAC.\\

    Au début les Tables de Commutation des commutateurs sont vides comme représente:
    \tablesCommutation{&\\}{&\\}{&\\}
    \begin{definition}
        Une \textbf{Table de Commutations} associé les adresses MAC à des interfaces sur les ports du commutateur.
    \end{definition}
\end{exercise}

\newpage
\begin{resolution}
    On suppose que A3 envoi un trame à A4:\\

    A3 est connecte au switch S1 qui, au début, n'a pas l'adresse de A4 dans sa Table de Communication. Donc il faudrait faire un broadcast, un transmission de diffusion:
        \tablesCommutation{
            01 & A3\\
        }{
            & \\
        }{
            & \\
        }
    \begin{remark}
        Quand il y a un broadcast le switch fera la transmission de l'adresse de la machine qu'envoie des données pour que le réseau peut connaître cette nouvelle machine qui communique.
    \end{remark}
    Comme S1 et S2 sont connectes, la diffusion sera transmisse à S2 qui n'a pas non plus l'adresse de A4 dans sa Table de Communication. Donc il faudrait faire un broadcast:
        \tablesCommutation{
            01 & A3\\
        }{
            U2 & A3\\
        }{
            & \\
        }
    \begin{remark}
        L'adresse estoque sera l'adresse d'origine de la message.
    \end{remark}
    Le même se passe entre S2 et S3. Donc il faudrait un broadcast:
        \tablesCommutation{
            01 & A3\\
        }{
            U2 & A3\\
        }{
            U & A3\\
        }
    \begin{remark}
        Quand il aura un broadcast, transmission de diffusion, le signal sera retransmettre dans tous les ports sauf pour la port d'où le signal vient.
    \end{remark}
\end{resolution}

\newpage
\subsection*{Question 4}
\begin{resolution}
    On suppose que A4 envoi un trame à A1:\\

    A4 est connecte au switch S1 qui, au début, n'a pas l'adresse de A1 dans sa Table de Communication. Donc il faudrait faire un broadcast, un transmission de diffusion:
        \tablesCommutation{
            01 & A3\\
            19 & A4\\
        }{
            U2 & A3\\
        }{
            U & A3\\
        }
    Comme S1 et S2 sont connectes il aura faire un broadcast:
        \tablesCommutation{
            01 & A3\\
            19 & A4\\
        }{
            U2 & A3\\
            U2 & A4\\
        }{
            U & A3\\
        }
    \begin{remark}
        Quand il y a des machines qui sont connectes en dehors d'un commutateur c'est commun d'y avoir le même adresse.\\
        
        S'il y a un transmission l'autre commutateur s'occupe de retransmettre à la machine correcte. 
    \end{remark}
    Le même se passe entre S2 et S3. Donc il aura un broadcast:
        \tablesCommutation{
            01 & A3\\
            19 & A4\\
        }{
            U2 & A3\\
            U2 & A4\\
        }{
            U & A3\\
            U & A4\\
        }
\end{resolution}

\newpage
\subsection*{Question 5}
\begin{resolution}
    On suppose que A4 envoi un trame à A2:\\

    Comme les Tables de Communication n'ont pas l'adresse de A2 il y aura un broadcast sur les switches, mais cette fois-ci il n'y aura pas d'alteration sur les Tables car l'adresse de A4 est déjà-là:
        \tablesCommutation{
            01 & A3\\
            19 & A4\\
        }{
            U2 & A3\\
            U2 & A4\\
        }{
            U & A3\\
            U & A4\\
        }
\end{resolution}

\subsection*{Question 6}
\begin{resolution}
    On suppose que A2 envoi un trame à A4:\\

    A2 est connecte au switch 2 qui connaît déjà l'adresse de A4 et donc la trame sera directement envoyé à S1 sur le port U2:
        \tablesCommutation{
            01 & A3\\
            19 & A4\\
        }{
            U2 & A3\\
            U2 & A4\\
            03 & A2\\
        }{
            U & A3\\
            U & A4\\
        }
    \begin{remark}
        Comme S2 n'avait pas l'adresse de A2 sur sa Table de Commutation il l'ajoute.
    \end{remark}
    En suite S1 consulte sa Table de Commutation et relaye la trame vers le port 19 en ajoutant l'adresse de A2 à sa Table car il ne la connaît pas:
        \tablesCommutation{
            01 & A3\\
            19 & A4\\
            U  & A2\\
        }{
            U2 & A3\\
            U2 & A4\\
            03 & A2\\
        }{
            U & A3\\
            U & A4\\
        }
    \begin{remark}
        S2 ne relaye pas la trame vers le switch 3 car il connaissait déjà l'adresse d'A4 et donc S3 n'aura pas la possibilité de mémoriser l'adresse d'A2. 
    \end{remark}
\end{resolution}

\newpage
\subsection*{Question 7}
\begin{resolution}
    On suppose que A4 envoi un trame à A1:\\

    Comme les Tables de Communication n'ont pas l'adresse de A1 il y aura un broadcast sur les switches, mais cette fois-ci il n'y aura pas d'alteration sur les Tables car l'adresse de A4 est déjà-là:
        \tablesCommutation{
            01 & A3\\
            19 & A4\\
            U  & A2\\
        }{
            U2 & A3\\
            U2 & A4\\
            03 & A2\\
        }{
            U & A3\\
            U & A4\\
        }
\end{resolution}

\subsection*{Question 8}
\begin{resolution}
    On suppose que A1 envoi un trame à A4:\\

    A1 est connecte au switch 3 qui connaît déjà l'adresse de A4 et donc la trame sera directement envoyé à S2 sur le port U qu'enverra en suite à S1 sur le port U2:
        \tablesCommutation{
            01 & A3\\
            19 & A4\\
            U  & A2\\
            U  & A1\\
        }{
            U2 & A3\\
            U2 & A4\\
            03 & A2\\
            U1 & A1\\
        }{
            U & A3\\
            U & A4\\
            02 & A1\\
        }
    \begin{remark}
        Le même process décrit à la page precedent se répété ici. Pas besoin de préciser les détails encore un fois.
    \end{remark}
\end{resolution}
\end{document}