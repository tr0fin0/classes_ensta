\documentclass{article}
\usepackage{C:/Users/guitr/Documents/git_repositories/tpack/tpack}
% \usepackage{C:/Users/Admin-PC/Documents/git_repository/tpack/tpack}
% \usepackage{/home/tr0fin0/git_repositories/tpack/tpack}


\title{IC203 - Reseaux d'Information}
\project{Travail Dirige 5 - Résolution}
\author{Guilherme Nunes Trofino}
\authorRA{2022-2024}


\makeatletter
\begin{document}\selectlanguage{french}
\maketitle
\setlength{\parindent}{0pt}

\newcommand{\adressTable}[7]{
    \begin{table}[H]
        \centering\begin{tabular}{c|ll}
            \multirow{5}{*}{#1} & 193.215.124.#2/#7 & sous-réseau\\
            \cline{2-3}
            & 193.215.124.#3 & \multirow{2}{*}{adresses disponibles}\\
            & 193.215.124.#4 &\\
            \cline{2-3}
            & 193.215.124.#5 & broadcast\\
            \cline{2-3}
            & 255.255.255.#6 & netmask\\
        \end{tabular}
    \end{table}
}



\section*{Exercice 1}
\noindent Dans ce exercice on étudiera Routage et Protocole ARP.\\

On considère qu'un dispositif a la table de connections suivante:
\begin{table}[H]
    \centering\begin{tabular}{lll | ll}
        Destination   & Mask Gateway    & Device        & First Address & Last Address\\
        \hline
        default       & 0.0.0.0         & 137.194.192.14 \\
        127.0.0.1     & 255.255.255.255 & 127.0.0.1      \\
        \hline
        137.194.160.0 & 255.255.254.0   & 137.194.160.3 & 137.194.160.1 & 137.194.161.255\\
        137.194.168.0 & 255.255.254.0   & 137.194.168.1 & 137.194.168.1 & 137.194.169.255\\
        137.194.192.0 & 255.255.254.0   & 137.194.192.1 & 137.194.192.1 & 137.194.193.255\\
        137.194.200.0 & 255.255.254.0   & 137.194.200.1 & 137.194.200.1 & 137.194.201.255\\
        137.194.204.0 & 255.255.254.0   & 137.194.204.1 & 137.194.204.1 & 137.194.205.255\\
        \hline
    \end{tabular}
\end{table}

\begin{remark}
    On considère l'adresse $137.194.192.22$ comme la \textbf{Notation Décimale} pointée.
\end{remark}
\begin{remark}
    On considère l'adresse $000.000.000.000$/x. x est la \textbf{Masque d'Adressage} e représente combien de bits de l'adresse sont immutables à compter du MSB en direction au LSB.
\end{remark}
\begin{remark}
    On considère que le premier adresse sera \textbf{l'Adresse du Sous-Réseau}.
\end{remark}
\begin{remark}
    On considère que le dernier adresse sera \textbf{l'Adresse de Diffusion}, Broadcast.
\end{remark}

Ça donne le diagramme suivant:
\begin{figure}[H]
    \centering\begin{tikzpicture}[]
        % modules
        \node[module] (R0) {137.194.192.14};
        \node[module, right= of R0] (SR0) {137.194.192.0/23};
        
        \node[module, above right=5mm of SR0] (SR1) {137.194.200.0/23};
        \node[module, above=5mm of SR0] (SR2) {137.194.204.0/23};

        \node[module, below right=5mm of SR0] (SR3) {137.194.168.0/23};
        \node[module, below=5mm of SR0] (SR4) {137.194.160.0/23};
        
        % connections
        \foreach \i in {1,2,3,4}
            \draw[<->] (SR0)--(SR\i);
        \draw[<->] (R0)--(SR0);
    \end{tikzpicture}
\end{figure}

Cette machine cherche à joindre les équipements: 137.194.160.85, 137.194.161.122, 137.194.204.22 et 137.194.20.10.\\

On note que les adresses 137.194.160.85, 137.194.161.122 et 137.194.204.22 sont dans l'intervale d'adresses valables pour le sous-réseau correspondant. Dans ce-cas là on a le premier cas de requête ARP.\\

On note que l'adresses 137.194.20.10 n'est pas dans l'intervale d'adresses valables pour les sous-réseaux. Dans ce-cas là on a le deuxième cas de requête ARP.

\subsection*{Question 1}
\begin{exercise}
    Pour chaque sous réseau raccordé à la station, identifier le netmask, l'adresse du sous réseau et l'adresse de diffusion du sous réseau (en notation décimale pointée).
\end{exercise}
\begin{resolution}

\end{resolution}

\subsection*{Question 2}
La présence de plusieurs interfaces sur cette machine rend nécessaire l’activation du routage.
La commande netstat –rn retourne le contenu de la table de routage de la machine courante.
Les informations suivantes sont alors retournées :
Destination Mask Gateway Device Mxfrg Rtt Ref Flg Out In/Fwd
-------------------- --------------- -------------------- ------ ----- ----- --- --- ----- ------
137.194.168.0 255.255.254.0 137.194.168.1 qfe2 1500* 0 2 U 41867 0
137.194.200.0 255.255.254.0 137.194.200.1 qfe3 1500* 0 3 U 46857 0
137.194.204.0 255.255.254.0 137.194.204.1 qfe1 1500* 0 2 U 34388 0
137.194.160.0 255.255.254.0 137.194.160.3 hme0 1500* 0 2 U 293056 0
137.194.192.0 255.255.254.0 137.194.192.1 qfe0 1500* 0 2 U 618612 0
default 0.0.0.0 137.194.192.14 1500* 0 0 UG 8513070 0
127.0.0.1 255.255.255.255 127.0.0.1 lo0 8232* 0 0 UH 1229083 0
\begin{exercise}
    Cette machine cherche à joindre l'équipement 137.194.160.85.
    • Expliquer brièvement ce qui se passe.
    • Même question pour les destinations 137.194.161.122, 137.194.204.22 et
    137.194.20.10
\end{exercise}


\section*{Exercise 2}
Un utilisateur de la machine esmeralda.enst.fr cherche des documents sur l’IP. Il se connecte
au serveur FTP de l’ENST (ftp.enst.fr) pour rapatrier des RFCs (Request for Comments).


Le même utilisateur désire à présent consulter des documents administratifs sur la page web
de l’ENST (www.enst.fr). (2,5 points)

Données :
esmeralda.enst.fr Adr.IP : 137.194.160.71 Masque : 255.255.254.0

Adr.MAC: 8 :0 :20:AC:3F:38
Routeur par default: 137.194.160.121

ftp.enst.fr Adr.IP : 137.194.160.3 Masque : 255.255.254.0

Adr.MAC: 8 :0 :20:A2:8E:AC
Routeur par default: 137.194.160.121
www.enst.fr Adr.IP : 137.194.2.45Masque : 255.255.254.0

Adr.MAC: 08 :00 :20:a6:8a:5c
Routeur par default: 137.194.2.96

benelos.enst.fr Adr.IP : 137.194.160.121 Masque : 255.255.254.0

Adr.MAC: 0 :80:2D:6F:EC:81
Adr.IP : 137.194. 2.96 Masque : 255.255.254.0
Adr.MAC : 0 :80:2D:6F:EC:2B
\subsection*{Question 1}
\begin{exercise}
    1. Décrire le déroulement des actions au niveau des couches 2 et 3 nécessaires à ces
transferts de fichier. Donner les en-têtes IP et Ethernet des paquets envoyés par le
serveur ftp vers le client.
\end{exercise}
\begin{resolution}
    a
\end{resolution}

\subsection*{Question 2}
\begin{exercise}
    2. Même question : Décrire la « vie d’un paquet IP » de esmeralda jusqu’au serveur
Web et donner les en-têtes IP et Ethernet des paquets reçus par esmeralda. (2,5
points)
\end{exercise}
\begin{resolution}

\end{resolution}


\end{document}