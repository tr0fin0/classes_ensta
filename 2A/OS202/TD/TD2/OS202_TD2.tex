\documentclass{article}
\usepackage{../../../../tpack/document/tpack}
\usetikzlibrary{decorations.pathreplacing,calligraphy}

\title{OS202 - Programming Parallel Computers}
\project{Travail Dirigée}
\author{Guilherme Nunes Trofino}
\authorRA{2022-2024}


\makeatletter
\begin{document}\selectlanguage{french}
\maketitle
\setlength{\parindent}{0pt}




\newpage\tableofcontents

\section{Introduction}
\subfile{../../../../intro.tex}


\subsection{Information Matier}
\paragraph{Référence}Dans cette matière le but sera de comprendre \title{}. Ce travail est sur \href{https://github.com/JuvignyEnsta/Course2023/blob/main/TravauxDirig%C3%A9s/TD_numero_1/Sujet.pdf}{https://github.com/} avec l'objectif d'étudier et démontrer l'augmentation de performance quand on utilise la programmation parallèle.

\subsection{Caracteristiques Ordinateur}
\paragraph{CPU}On utilisé le commande \texttt{lscpu} pour avoir des informations sur le processeur de mon ordinateur en retournant le suivant:
\begin{scriptsize}
    % \mycode\lstinputlisting[language=bash]{example/main.cpp}
    \mycode
    \begin{lstlisting}[language=bash]
    Architecture:           x86_64
        CPU op-mode(s):         32-bit, 64-bit
        Address sizes:          39 bits physical, 48 bits virtual
        Byte Order:             Little Endian
        CPU(s):                 20
        On-line CPU(s) list:    0-19
        Vendor ID:              GenuineIntel
    Model name:            12th Gen Intel(R) Core(TM) i7-12700H
        CPU family:          6
        Model:               154
        Thread(s) per core:  2
        Core(s) per socket:  14
        Socket(s):           1
        Stepping:            3
        CPU max MHz:         4700.0000
        CPU min MHz:         400.0000
    \end{lstlisting}
\end{scriptsize}
On peut voir qui mon ordinateur a, théoriquement, 20 CPU's disponibles avec les mémoires suivants:
\begin{scriptsize}
    \mycode
    \begin{lstlisting}[language=bash]
    Caches (sum of all):     
        L1d:    544  KiB    (14 instances)
        L1i:    704  KiB    (14 instances)
        L2:     11.5 MiB    ( 8 instances)
        L3:     24   MiB    ( 1 instance)
    \end{lstlisting}
\end{scriptsize}
Ces données seront utilisés pour l'analyse des performances.


\section{Cours Amphi}
\begin{resolution}

\end{resolution}

\subsection{Interblocage}
\begin{resolution}
    On considère le l'exercice sur l'interblocage donné dans le cours et décrivez deux scénarios...
    \begin{enumerate}
        \item \texttt{Blocking}: 
    \begin{scriptsize}
        \mycode
        \begin{lstlisting}[language=C++]
    if (rank==0)
    {
        MPI_Recv(rcvbuf, count, MPI_DOUBLE, 1, 101, commGlob, &status);
        MPI_Send(sndbuf, count, MPI_DOUBLE, 1, 102, commGlob);
    }
    else if (rank==1)
    {
        MPI_Recv(rcvbuf, count, MPI_DOUBLE, 0, 102, commGlob, &status);
        MPI_Send(sndbuf, count, MPI_DOUBLE, 0, 101, commGlob);
    }
        \end{lstlisting}
    \end{scriptsize}
        \item \texttt{Non-Blocking}: 
    \begin{scriptsize}
        \mycode
        \begin{lstlisting}[language=C++]
    MPI_Request req;
    if (rank == 0)
    {
        double vecteur[5] = { 1., 3., 5., 7., 22. };
        MPI_Isend(vecteurs, 5, MPI_DOUBLE, 1, 101, commGlob, &req);
        // Some compute with other data can be executed here!
        MPI_Wait(req, MPI_STATUS_IGNORE);
    }
    else if (rank==1)
    {
        MPI_Status status; double vecteurs[5];
        MPI_Irecv(vecteurs, 5, MPI_DOUBLE, 0, 101, commGlob, &req);
        int flag = 0;
        
        do {
            // Do computation while message is not received on another data
            MPI_Test(&req, &flag, &status);
        } while(flag);
    }
        \end{lstlisting}
    \end{scriptsize}
    \end{enumerate}
    La probabilité d'une interblocage est grand car une erreur dans le rank d'une variable pour causer une loop.
\end{resolution}

\subsection{Alice Problem's}
\begin{resolution}
    On considère les equations suivants:
    \begin{equation}
        \boxed{
            S(n) = \lim_{n\to\infty}\frac{n}{1 + (n-1)f} \to \frac{1}{f}
        }
    \end{equation}
    Où:
    \begin{enumerate}[noitemsep]
        \item $n$: number of computing units;
        \item $f$: fraction of $t_s$ which can't be parallelized;
        \item $t_s$: time to run code sequential;
    \end{enumerate}
    Comme le code 90\% du code peut-être parallèle $f = 1 - 0.9 = 10\%$ et comme $n \gg 1$ on a que $S(n)$ sera au maximum $S(n) = 10$.
\end{resolution}

\section{Mandelbrot}
\begin{resolution}
    On considère que Mandelbrot est un ensemble fractal:
    \begin{equation}
        z_{n+1} = 
        \begin{cases}
            z_{0}   &= 0\\
            z_{n+1} &= z_{n}^{2} + c\\
        \end{cases}
        \qquad\text{où $c$: valeurs complexe donnée}
    \end{equation}
    On peut montrer que si il existe $N$ tel que $|z_N|>2$, alors la suite diverge
    % TODO chercher la démonstration

    \begin{equation}
        c = (x_{\min} + p_{i}\frac{x_{\max} - x_{\min}}{W}) + i(y_{\min} + p_{j}\frac{y_{\max} - y_{\min}}{H})
    \end{equation}
    Image de $W$ par $H$ pixels telle qu'à chaque pixel $(p_{i}, p_{j})$
    % TODO comprendre equation

    Il semble qu'on étude ses fractals car un français les a découvert.
    \begin{scriptsize}
        \mycode
        \begin{lstlisting}[language=C++]
    MPI_Request req;
    if (rank == 0)
    {
        double vecteur[5] = { 1., 3., 5., 7., 22. };
        MPI_Isend(vecteurs, 5, MPI_DOUBLE, 1, 101, commGlob, &req);
        // Some compute with other data can be executed here!
        MPI_Wait(req, MPI_STATUS_IGNORE);
    }
    else if (rank==1)
    {
        MPI_Status status; double vecteurs[5];
        MPI_Irecv(vecteurs, 5, MPI_DOUBLE, 0, 101, commGlob, &req);
        int flag = 0;
        
        do {
            // Do computation while message is not received on another data
            MPI_Test(&req, &flag, &status);
        } while(flag);
    }
        \end{lstlisting}
    \end{scriptsize}
    mpirun -np 4 ./helloWorldMPI.exe
\end{resolution}

\section{Produit Matrice-Vector}
\subsection{Vecteur Colonne}
\begin{resolution}
    
\end{resolution}

\subsection{Vecteur Ligne}
\begin{resolution}
    
\end{resolution}



\end{document}