\documentclass{article}
% \usepackage{C:/Users/guitr/Documents/git_repositories/tpack/tpack}
% \usepackage{C:/Users/Admin-PC/Documents/git_repository/tpack/tpack}
\usepackage{/home/tr0fin0/Documents/git_repositories/tpack/tpack}
% \usepackage{/home/Documents/git_repositories/tpack/tpack}
\usetikzlibrary{decorations.pathreplacing,calligraphy}

\title{OS202 - Programming Parallel Computers}
\project{Travail Dirigée}
\author{Guilherme Nunes Trofino}
\authorRA{2022-2024}


\makeatletter
\begin{document}\selectlanguage{french}
\setlength{\parindent}{0pt}

\maketitle
\newpage\tableofcontents

\section{Introduction}
% \subfile{C:/Users/guitr/Documents/git_repositories/classes_ensta/intro.tex}
% \subfile{C:/Users/Admin-PC/Documents/git_repository/classes_ensta/intro.tex}
\subfile{/home/tr0fin0/Documents/git_repositories/classes_ensta/intro.tex}
% \subfile{/home/Documents/git_repositories/classes_ensta/intro.tex}


\subsection{Information Matier}
\paragraph{Référence}Dans cette matière le but sera de comprendre \title{}. Ce travail est sur \href{https://github.com/JuvignyEnsta/Course2023/blob/main/TravauxDirig%C3%A9s/TD_numero_1/Sujet.pdf}{https://github.com/} avec l'objectif d'étudier et démontrer l'augmentation de performance quand on utilise la programmation parallèle.

\subsection{Caracteristiques Ordinateur}
\paragraph{CPU}On utilisé le commande \texttt{lscpu} pour avoir des informations sur le processeur de mon ordinateur en retournant le suivant:
\begin{scriptsize}
    % \mycode\lstinputlisting[language=bash]{example/main.cpp}
    \mycode
    \begin{lstlisting}[language=bash]
    Architecture:           x86_64
        CPU op-mode(s):         32-bit, 64-bit
        Address sizes:          39 bits physical, 48 bits virtual
        Byte Order:             Little Endian
        CPU(s):                 20
        On-line CPU(s) list:    0-19
        Vendor ID:              GenuineIntel
    Model name:            12th Gen Intel(R) Core(TM) i7-12700H
        CPU family:          6
        Model:               154
        Thread(s) per core:  2
        Core(s) per socket:  14
        Socket(s):           1
        Stepping:            3
        CPU max MHz:         4700.0000
        CPU min MHz:         400.0000
    \end{lstlisting}
\end{scriptsize}
On peut voir qui mon ordinateur a, théoriquement, 20 CPU's disponibles avec les mémoires suivants:
\begin{scriptsize}
    \mycode
    \begin{lstlisting}[language=bash]
    Caches (sum of all):     
        L1d:    544  KiB    (14 instances)
        L1i:    704  KiB    (14 instances)
        L2:     11.5 MiB    ( 8 instances)
        L3:     24   MiB    ( 1 instance)
    \end{lstlisting}
\end{scriptsize}
Ces données seront utilisés pour l'analyse des performances.


% Ce TD peut être réalisé au choix, en C++ ou en Python

% Implémenter l'algorithme "bucket sort" tel que décrit sur les deux dernières planches du cours n°3 :

% le process 0 génère un tableau de nombres arbitraires,
% il les dispatch aux autres process,
% tous les process participent au tri en parallèle,
% le tableau trié est rassemblé sur le process 0.
\end{document}
