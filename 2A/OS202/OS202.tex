\documentclass{article}
% \usepackage{C:/Users/guitr/Documents/git_repositories/tpack/tpack}
% \usepackage{C:/Users/Admin-PC/Documents/git_repository/tpack/tpack}
\usepackage{/home/tr0fin0/Documents/git_repositories/tpack/tpack}
% \usepackage{/home/Documents/git_repositories/tpack/tpack}
\usetikzlibrary{decorations.pathreplacing,calligraphy}

\title{OS202 - Programming Parallel Computers}
\project{Résumé Théorique}
\author{Guilherme Nunes Trofino}
\authorRA{2022-2024}


\makeatletter
\begin{document}\selectlanguage{french}
\maketitle
\setlength{\parindent}{0pt}

\newpage\tableofcontents

\section{Introduction}
% \subfile{C:/Users/guitr/Documents/git_repositories/classes_ensta/intro.tex}
% \subfile{C:/Users/Admin-PC/Documents/git_repository/classes_ensta/intro.tex}
\subfile{/home/tr0fin0/Documents/git_repositories/classes_ensta/intro.tex}
% \subfile{/home/Documents/git_repositories/classes_ensta/intro.tex}

% https://github.com/JuvignyEnsta/IN203_SUPPORT_COURS
% https://github.com/JuvignyEnsta/Promotion_2022


% 07/03 examen écrit
% 21/03 examen algorithme
% projet
% jean-didier.garaud@onera.fr



% différent configurations pour la constrution d'une système parallele
%     bus partager
%         plus facile
%         moins efficace
%         conflit d'information entre les différents processeurs

% latency memory example haswell architecture
% l1 cache
% l2 cache
% l3 cache
% ram
% swap

% la vélocité est inverse à la taille
% plus grand, plus lente
% plus petit, plus rapide

% solutions pour résoudre le problème, temps d'accès, de mémoire
%     interleaved RAMS
%         many physical memory units interleaved by the memory bus
%         number of physical memory units
%         number of contiguous bytes in a unique physique memory

%         utilise la mémoire pendant qu'on attend la réception d'une information d'une requeté anterieur

%     cache memory fast small memory unit where one stores temporary data
%         when multiple access to a same variable in a short time, speedup the read or write access
%         cache memory managed by the CPU

%         to optimize his program the programmer must know the strategies used by the CPU

%         associative memory cache each cache memory address mapped on fixed RAM address with a modulo

% UC calculus unit

% problème de coerance de cache
% \href{https://en.wikipedia.org/wiki/Cache_coherence}{cache coherence}

% % #include <mpi.h> // assembly of library used for parallel programming

% it must precise what type of data is used because different machines can have different endiness and memory configuration, parallel computing was conceived to work in heterogenous machines



% interlocking, interblocage
% situation where many processes are waiting each other for an infinite time to complete their messages
% I uppercase in a command is a non blocking function


% Distributed parallel rules
%     ethernet data exchange is very slow compared to memory access limit as possible the data exchanges
%     to hide data exchange cost it's better to compute some values during the exchange of data : prefer to use non blocking message exchange
%     each message has an initial cost: prefer to regroup data in a temporary buffer if needed
%     all processes quit the program at the same time: try to balance the computing load among computing nodes


% TD 1
% https://github.com/JuvignyEnsta/Course2023/blob/main/TravauxDirig%C3%A9s/TD_numero_1/Sujet.pdf

% amphi2
% speed up
% % mesure pour évaleur l'efficace d'une code parrallel par rapport à sa version sérielle
% % S(n) = \frac{ts}{tp(n)}

% amdahl's law
% give a limit for the speed up 
% let ts, temps sequentielle, be the time necessary to run the code in sequential
% let f be the fraction of ts to the part of the code which can't be parallelise
% so the best expected speed up is

% s(n) =  frac{ts}{f ts + (1-f)ts/n} = \frac{n}{1 + (n+1)f}
% this law is useful to find a reasonable number of computing cores to use for an application 
% limitation may change with volue of input data 


% gustafson's law
% speedup behavior with constant volume input data per processes


% granularity
% ratio between ocmputing intensity and quantity of data exchanged between processes

% load balancing
% all processes execute a comptation section of the code with asme duration
% speedup is badly impacted if some parts of the code are far away from load balancing


% embarrassingly parallel algorithms
% each data used and computed are independent
% no data race in multithread context
% no communication between processes in distributed environment

% property
% in distributed parallel context no data must be exchanged between processes to compute the results


% syracuse series
% definition
% uo chosen
% un+1 un/2 if un is even ou 3un+1 si un is odd

% one cycle exists a conjucture the series reaches the cycle below in a finite number of iterations
% 1 4 2 1 ...

% some definiions
% lenght of flight number of iterations
% height of the flight maximal value reached by a strategies

% goal of the program compute de taille




\subsection{Information Matier}
\paragraph{Référence}Dans cette matière le but sera de comprendre comment une Système d'Exploitation marche.

\subsection{Utils}
% Tools for shared memory computation
% Many tools can be used to use many "threads" and the synchronization in memory. The most used
% are :
%     OpenMP : Compilation directives and small C API (#pragma) ;
%     Standard library in C++ (threads, asynchronous functions) ;
%     TBB (Threading Building Block library, Intel) : Open source library from Intel
% But the programmer must take care of memory conflict access :
%     When a thread writes a datum and some other threads read simultaneously that same datum ;
%     When some thread writes at the same datum ;
%     Doesn't rely on the instruction order in the program ! (out-of-order optimization by compiler and
% processor).

% All libraries managing the distributed parallelism computation provide similar functionalities.
% Running a distributed parallel application
% • An application is provided to the user to run his application(s) on a wanted number nbp of
% computing nodes (given when running the application) ;
% • The computing nodes where the application(s) is launched is defined by default or in a file
% provided by the user ;
% • The default output for all processes are the terminal output from which was launched the
% application ;
% • A communicator (defining a set of processes) is defined by default including all launched
% processes (MPI_COMM_WORLD) ;
% • The application gives a unique number for each process in a communicator (numbering from
% zero to nbp-1) ;
% • All processes terminate the program at the same time ;

% Constitution of a data message to send
% • The communicator used to send the data ;
% • The memory address of the contiguous data to send ;
% • The number of data to send ;
% • The type of the data (integer, real, user def type, and so.) ;
% • The rank of destination process ;
% • A tag number to identify the message
% Constitution of a data message to receive
% • The communicator used to receive the data ;
% • A memory address of a buffer where store the received data ;
% • The number of data to receive ;
% • The type of the data (integer, real, user def type, and so.)
% • The rank of the sender process (can be any process) ;
% • A tag number to identify the message (can be any tag if needed)
% • Status of the message (receive status, error, sender, tag) ;
% if (rank == 0) {
% double vecteur[5] = { 1., 3., 5., 7., 22. };
% MPI_Send(vecteurs, 5, MPI_DOUBLE, 1, 101, commGlob); }
% else if (rank==1) {
% MPI_Status status; double vecteurs[5];
% MPI_Recv(vecteurs, 5, MPI_DOUBLE, 0, 101, commGlob, &status); }

% Interlocking
% Definition
% • Interlocking is a situation where many processes are waiting each other for an infinite time to complete their messages ;
% • For example, process 1 waits to receive a message from 0 and 0 waits to receive a message from 1 ;
% • Or process 0 sends a message to 1 and process 1 waits a message from 0 but with wrong tag !


% Blocking and non blocking message
% Definition
% • Blocking message : Wait the complete reception of the message before returning from the function ;
% • Non blocking message : Post the message to send or receive and return from the function immediatly !
% A non blocking send is copied in a buffer before to be sent ;

% A scheme to avoid interlocking situations
% The scheme for all processes
% • First do receptions in non blocking mode ;
% • Second, do send in blocking mode (or non blocking mode if you want to overlay message cost with computing)
% • Third, synchronize yours receptions (waiting for completion or testing to overlay message cost with computing)

% MPI_Request req; MPI_Status status;
% if (rank==0)
% { MPI_Irecv(rcvbuf, count, MPI_DOUBLE, 1, 101, commGlob, &req);
% MPI_Send(sndbuf, count, MPI_DOUBLE, 1, 102, commGlob);
% MPI_Wait(&req, &status);
% }
% else if (rank==1)
% { MPI_Irecv(rcvbuf, count, MPI_DOUBLE, 0, 102, commGlob, &req);
% MPI_Send(sndbuf, count, MPI_DOUBLE, 0, 101, commGlob);
% MPI_Wait(&req, &status);
% }

% Distributed parallel rules
% • Ethernet data exchange is very slow compared to memory access : limit as possible
% the data exchanges ;
% • To hide data exchange cost, it’s better to compute some values during the exchange
% of data : prefer to use non blocking message exchange
% • Each message has an initial cost : prefer to regroup data in a temporary buffer if
% needed ;
% • All processes quit the program at the same time : Try to balance the computing load
% among computing nodes ;

% Scalability
% Definition
% For a parallel program, scalability is the behaviour of the speed-up when we raise up the
% number of processes or/and the amount of input data.

% How to evaluate the scalability ?
% • Evaluate the worst speed-up : For a global fixed amount of data, draw the speed-up
% curve in function of the number of processes ;
% • Evaluate the best speed-up : For a fixed amount of data per process, draw the
% speed-up curve in function of the number of processes ;
% • In concrete use of the program, the speed-up may be between the worst and best
% scenario.

% Granularity
% Ratio between computing intensity and quantity of data exchanged between processes
% • Sending and receiving data is prohibitive :
% • Initial cost of a message : each message has an initial cost : set the connection, get the same protocol, etc.This cost
% is constant.
% • Cost of the data transfer : at last, the cost of the data flow is linear with the number of data to exchange
% • These costs are greater than the cost of memory operations in RAM
% • Better to copy some sparse data in a buffer and send the buffer, rather than send scattered data with multiple send
% and receives
% • Try to minimize the number of data exchange between processes
% • The greater the ratio between number of computation instructions and messages to exchange, the better will be your
% speed-up !
% • Low speedup can be improved with non blocking data exchanges.

% Definition
% Embarrassingly parallel algorithm
% • Each data used and computed are independent ;
% • No data race in multithread context ;
% • No communication between processes in distributed environment

% Memory access and computing operation have the same complexity : On shared memory, memory bound limits the
% performance
% • On distributed memory, each process uses his own physical memory and no data must be exchanged : Speed-up may be
% % linear relative to the number of processes (if data intensity is enough)

% Data exchanges between processes is very expensive, so find some algorithms which minimize data exchanges ;
% • In general, the receive operation doesn’t know the number of values to receive ⇒ one must probe the received message
% to get the number of data to receive, allocate the relative buffer and receive the data !

\end{document}