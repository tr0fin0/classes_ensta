\documentclass{article}
% \usepackage{C:/Users/guitr/Documents/git_repositories/tpack/tpack}
% \usepackage{C:/Users/Admin-PC/Documents/git_repository/tpack/tpack}
\usepackage{/home/tr0fin0/Documents/git_repositories/tpack/tpack}
% \usepackage{/home/Documents/git_repositories/tpack/tpack}
\usetikzlibrary{decorations.pathreplacing,calligraphy}

\title{OS202 - Programming Parallel Computers}
\project{Résumé Théorique}
\author{Guilherme Nunes Trofino}
\authorRA{2022-2024}


\makeatletter
\begin{document}\selectlanguage{french}
\maketitle
\setlength{\parindent}{0pt}

\newpage\tableofcontents

\section{Introduction}
% \subfile{C:/Users/guitr/Documents/git_repositories/classes_ensta/intro.tex}
% \subfile{C:/Users/Admin-PC/Documents/git_repository/classes_ensta/intro.tex}
\subfile{/home/tr0fin0/Documents/git_repositories/classes_ensta/intro.tex}
% \subfile{/home/Documents/git_repositories/classes_ensta/intro.tex}

https://github.com/JuvignyEnsta/IN203_SUPPORT_COURS
https://github.com/JuvignyEnsta/Promotion_2022


07/03 examen écrit
21/03 examen algorithme
projet
jean-didier.garaud@onera.fr



différent configurations pour la constrution d'une système parallele
    bus partager
        plus facile
        moins efficace
        conflit d'information entre les différents processeurs

latency memory example haswell architecture
l1 cache
l2 cache
l3 cache
ram
swap

la vélocité est inverse à la taille
plus grand, plus lente
plus petit, plus rapide

solutions pour résoudre le problème, temps d'accès, de mémoire
    interleaved RAMS
        many physical memory units interleaved by the memory bus
        number of physical memory units
        number of contiguous bytes in a unique physique memory

        utilise la mémoire pendant qu'on attend la réception d'une information d'une requeté anterieur

    cache memory fast small memory unit where one stores temporary data
        when multiple access to a same variable in a short time, speedup the read or write access
        cache memory managed by the CPU

        to optimize his program the programmer must know the strategies used by the CPU

        associative memory cache each cache memory address mapped on fixed RAM address with a modulo

UC calculus unit

problème de coerance de cache
\href{https://en.wikipedia.org/wiki/Cache_coherence}{cache coherence}

#include <mpi.h> // assembly of library used for parallel programming

it must precise what type of data is used because different machines can have different endiness and memory configuration, parallel computing was conceived to work in heterogenous machines



interlocking, interblocage
situation where many processes are waiting each other for an infinite time to complete their messages
I uppercase in a command is a non blocking function


Distributed parallel rules
    ethernet data exchange is very slow compared to memory access limit as possible the data exchanges
    to hide data exchange cost it's better to compute some values during the exchange of data : prefer to use non blocking message exchange
    each message has an initial cost: prefer to regroup data in a temporary buffer if needed
    all processes quit the program at the same time: try to balance the computing load among computing nodes


TD 1
https://github.com/JuvignyEnsta/Course2023/blob/main/TravauxDirig%C3%A9s/TD_numero_1/Sujet.pdf



\subsection{Information Matier}
\paragraph{Référence}Dans cette matière le but sera de comprendre comment une Système d'Exploitation marche.

\end{document}