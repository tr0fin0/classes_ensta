\documentclass{article}
% \usepackage{C:/Users/guitr/Documents/git_repositories/tpack/tpack}
% \usepackage{C:/Users/Admin-PC/Documents/git_repository/tpack/tpack}
\usepackage{/home/tr0fin0/Documents/git_repositories/tpack/tpack}
% \usepackage{/home/Documents/git_repositories/tpack/tpack}
\usetikzlibrary{decorations.pathreplacing,calligraphy}

\title{ES201 - Architecture de Microprocesseurs}
\project{Travail Dirigée}
\author{Guilherme Nunes Trofino}
\authorRA{2022-2024}


\makeatletter
\begin{document}\selectlanguage{french}
\maketitle
\setlength{\parindent}{0pt}

\newpage\tableofcontents

\section{Introduction}
% \subfile{C:/Users/guitr/Documents/git_repositories/classes_ensta/intro.tex}
% \subfile{C:/Users/Admin-PC/Documents/git_repository/classes_ensta/intro.tex}
\subfile{/home/tr0fin0/Documents/git_repositories/classes_ensta/intro.tex}
% \subfile{/home/Documents/git_repositories/classes_ensta/intro.tex}

\subsection{Information Matier}
\paragraph{Référence}


microprocesseur faire une bouque infinite
modele de machine von neummann
mode d'exécution générale
si il n'y a pas des taches à faire qu'est-ce que le processeur faire pendant ce temps: If the hardware doesn't make allowance for this, then the CPU will have to run useless instructions until it is needed for real work

me mets en reve et me reveie 
faire surplace, bloque sur le meme adresse memoire, rester en bloque sur lui meme

adresse PC est program counter necessaire pour savoir ce que le programme doit executer

instruction memory
operation instruction decode par l'unité de contrôle control units


methodologie de conception
unité de calcule, machine de miller machine de ... états finits
multiplexeur
registres
unites fonctionnelles
mémoire

ALU
R instructions

000 and
001 or
010 add
110 sub
111 slt


temps d'accès et temps de propagation
plus grand mémoire plus grand le temps d'accès et propagation c'est une contrainte, plus de temps moins grand CPU


dans une processeur monocycle
cache données et cache instruction
le chemin le plus long est le chemin critique et denote le temps de cycle

inconvenienst du monocycle
toutes les instructions
worst case execution time, si minimizer le systeme sera plus efficace

avantage
connait le temps pour executer les instructions


modele de séquencement


le decoupage ne peut pa passer le temps de memoirasion
aléas de structures structural hazards
aléas de contrôle control hazards
aléas de données data hazards


\newpage\subsection*{Question 1}
\begin{resolution}
    a
    \begin{scriptsize}
        \myRISCV
        \begin{lstlisting}
    switch:
    andi  s6, s6, 0          // counter = 0
    
    case0:
    bne  s6, s5, case1      // k = 0
    add  s0, s3, s4         // f = i + j

    case1:
    addi  s6, s6, 1
    bne   s6, s5, case2      // k = 1
    add   s0, s1, s2         // f = g + h

    case2:
    addi  s6, s6, 1
    bne   s6, s5, case3      // k = 2
    sub   s0, s1, s2         // f = g - h

    case3:
    addi  s6, s6, 1
    bne   s6, s5, end        // k = 3
    sub   s0, s3, s4         // f = i - j

    end:
    j ret
        \end{lstlisting}
    \end{scriptsize}
    Le code précédent sera scalable, donc il peut être facilement changer pour ajouter des nouvelles conditions dans le case.\\

    Ici il y a le problème de temps d'exécution pas constant, ça veut dire qu'il y aura des temps d'exécution différent par chaque cas car il y aura besoin de plusieurs essayes avant d'arriver à le bon résultat.
\end{resolution}


\newpage\subsection*{Question 2}
\begin{resolution}
    a
    \begin{scriptsize}
        \myRISCV
        \begin{lstlisting}
    andi s1, s1, 0          // sum = 0
    andi s2, s2, 0          // i = 0
    addi s3, zero, 500      // j = 500
    andi s4, s4,0           // k = 0
    addi t2, zero, 1000     // 1st loop upper bound = 1000
    andi t3, s6, 0          // 2nd loop lower bound = 0
    addi t4, zero, 300      // 3rd loop upper bound = 300

    loop1:
    loop2:
    loop3:
    add  s1, s1, s2         // sum = sum + (i);
    add  s1, s1, s3         // sum = sum + (i + j);
    add  s1, s1, s4         // sum = sum + (i + j + k);
    addi s4, s4, 10         // k = k + 10

    bne  s4, t4, loop3      // if (k < 300) goto loop3, else execute 2nd loop
    andi s4, s4,0           // reinitialize k = 0
    subi s3, s3, 1          // j--

    bne  s3, t3, loop2      // if (j > 0) goto loop2, else execute 1st loop
    addi s3, zero, 500      // reinitialize j = 500
    addi s2, s2, 1          // i++

    bne  s2, t2, loop1      // if (i < 1000) goto loop1, else end program
    exit:
        \end{lstlisting}
    \end{scriptsize}
    Chaque register devrait avoir un \$ dans la convection de cette matière mais ici ce caractère a été élevé car le style du code ne marchait pas.\\
    
    Ici, en fait, au lieu d'avoir 3 couples c'est plus facile pour comprendre qu'il y a un grand couple avec 3 conditions necessaires à attendre pour que le code s'arrêt. 
\end{resolution}



\newpage\subsection*{Question 3}

\end{document}