\documentclass{article}
% \usepackage{C:/Users/Admin-PC/Documents/git_repository/tpack/tpack}
\usepackage{/home/tr0fin0/git_repositories/tpack/tpack}


\title{IC204 - Programation par Objets}
\project{Résumé Théorique}
\author{Guilherme Nunes Trofino}
\authorRA{2022-2024}


\makeatletter
\begin{document}\selectlanguage{french}
\maketitle

\newpage\tableofcontents

\section{Introduction}
\subfile{/home/tr0fin0/git_repositories/classes_ensta/intro.tex}
% \subfile{C:/Users/Admin-PC/Documents/git_repository/classes_ensta/intro.tex}\paragraph{Présentation}


\subsection{Informations Matière}
\paragraph{Présentation}Ce cours sera présenter par M. ... qui a pour but d'étudier la Programmation Orienté à Objets.

\subsection{run code}
\paragraph{Définition}C++, et C aussi, est unn langage compilé, c'est-à-dire que l'analyse syntaxique et sémantique du programme est réalisée intégralement avant l'éxecution et produit un code dit éxecutable, code écrit dans un langage de très bas-niveau permetant que l'ordinateur peut directement et efficacement exécuter.\\

\subsubsection{compilator}
\paragraph{Définition}Converte l'archive source dans une archive de byte code, parfois appeler code objet. Pour compiler un code il faut exécuter le commande suivante:
\begin{scriptsize}
    \myStyleCPP
    \begin{lstlisting}
    g++ file.cpp -o executable.exe && ./executable.exe
    \end{lstlisting}
\end{scriptsize}
Avec ces commandes le code sera d'abbord compilé et après executé.

\subsubsection{linker}
\paragraph{Définition}Après la création d'une archive de byte code le linker va faire reference à d'autres archives necessaires pour le code principale.

\subsection{\texttt{main}}
\paragraph{Définition}Inicialization du code se fera toujours avec la function \texttt{main} comme démontre:
\begin{scriptsize}
    \myStyleCPP
    \lstinputlisting[]{class_0/main.cpp}
\end{scriptsize}
La function retournerai 0 si le programme est éxecute sans aucune problème.

\subsubsection{\texttt{include}}
\paragraph{Définition}Quand une programme utilise des functions ou classes externes à le code il faut les incluire dans le code principale comme démontré:
\begin{scriptsize}
    \myStyleCPP
    \lstinputlisting[]{class_0/include.cpp}
\end{scriptsize}
C'est important de diviser et séparer les archives du code pour qu'il soit plus facile à comprendre et à l'éditer pendant le dévélopment.

\subsection{Mémoire}
\paragraph{Définition}Langages compilés sont, d'habitude, plus efficaces avec l'usage de mémoire car, la plus part du temps, l'utilisateur doit préciser la quantité de mémoire necessaire.
\subsubsection{\texttt{pointers}}
\paragraph{Définition}Variable particulière servant à stocker l'adresse en mémoire centrale d'une variable comme démontré:
\begin{scriptsize}
    \myStyleCPP
    \lstinputlisting[]{class_0/pointer.cpp}
\end{scriptsize}

\subsubsection{Allocation Statique}
\paragraph{Définition}

\subsubsection{Allocation Dynamique}
\paragraph{Définition}


\newpage\section{\texttt{class}}
\paragraph{Définition}

\subsection{\texttt{objet}}
\paragraph{Définition}

\subsection{\texttt{methode}}
\paragraph{Définition}

\subsubsection{\texttt{visibilité}}
\paragraph{Définition}
\end{document}