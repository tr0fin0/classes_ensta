\documentclass{article}
\usepackage{C:/Users/Admin-PC/Documents/git_repository/tpack/tpack}
% \usepackage{/home/tr0fin0/git_repositories/tpack/tpack}


\title{IC204 - Programation par Objets}
\project{Résumé Théorique}
\author{Guilherme Nunes Trofino}
\authorRA{2022-2024}


\makeatletter
\begin{document}\selectlanguage{french}
\maketitle

\newpage\tableofcontents

\section{Introduction}
\subfile{C:/Users/Admin-PC/Documents/git_repository/classes_ensta/intro.tex}
% \subfile{/home/tr0fin0/git_repositories/classes_ensta/intro.tex}


\subsection{Informations Matière}
\paragraph{Présentation}Ce cours sera présenter par M. Jean-Baptiste Laurent qui a pour but d'étudier la Programmation Orienté à Objets:
\begin{enumerate}[noitemsep]
    \item \url{https://perso.ensta-paris.fr/~bmonsuez/Cours};
    \item \url{https://isocpp.org/wiki/faq/templates};
    \item \url{https://accu.org/journals/overload/9/43/frogley_442/};
\end{enumerate}
Reference externe: \url{https://www.youtube.com/watch?v=vLnPwxZdW4Y};

\subsection{run code}
\paragraph{Définition}C++, et C aussi, est un langage compilé, c'est-à-dire que l'analyse syntaxique et sémantique du programme est réalisée intégralement avant l'éxecution et produition d'un code éxecutable, écrit dans un langage de bas-niveau permetant que l'ordinateur peut directement et efficacement exécuter.\\

\subsubsection{compilator}
\paragraph{Définition}Converte l'archive source dans une archive de byte code, parfois appeler code objet. Pour compiler un code il faut exécuter le commande suivante:
\begin{scriptsize}
    \mycode
    \begin{lstlisting}
    g++ file.cpp -o executable.exe && ./executable.exe
    \end{lstlisting}
\end{scriptsize}
Avec ces commandes le code sera d'abbord compilé et après executé.

\subsubsection{linker}
\paragraph{Définition}Après la création d'une archive de byte code le linker va faire reference à d'autres archives necessaires pour le code principale.

\subsection{\texttt{main}}
\paragraph{Définition}Inicialization du code se fera toujours avec la function \texttt{main} comme démontre:
\begin{scriptsize}
    \mycode\lstinputlisting[language=C++]{example/main.cpp}
\end{scriptsize}
La function retournerai 0 si le programme est éxecute sans aucune problème.

\subsubsection{\texttt{include}}
\paragraph{Définition}Quand une programme utilise des functions ou classes externes à le code il faut les incluire dans le code principale comme démontré:
\begin{scriptsize}
    \mycode\lstinputlisting[language=C++]{example/include.cpp}
\end{scriptsize}
C'est important de diviser et séparer les archives du code pour qu'il soit plus facile à comprendre et à l'éditer pendant le dévélopment.\\

\noindent Dans C++ il faut avoir l'extention \texttt{.cpp} ou \texttt{.hpp} pour convetion.

\subsubsection{\texttt{functions}}
\paragraph{Définition}Quand un programme a besoin d'estoquer seulement des functions necessaires pour la \texttt{main.cpp} c'est recomende de créer un archive \texttt{functions.hpp} et estoquer les informations là dedans pour mieux diviser les archives.
\subsection{\texttt{types}}
\subsubsection{\texttt{enumerate}}
\paragraph{Défition}Une enumerate est un nouveau type listant des éléments ne correspondant pas un type particulier. La déclaration la plus simple est:
\begin{scriptsize}
    \mycode\lstinputlisting[language=C++]{example/enum.cpp}
\end{scriptsize}

\subsubsection{\texttt{struct}}
\paragraph{Définition}Une struct permet de définir un nouveau type de variables regroupant plusieurs variables de types différents. La déclaration la plus simple est:
\begin{scriptsize}
    \mycode\lstinputlisting[language=C++]{example/struct.cpp}
\end{scriptsize}
Struct c'est un type de \texttt{class} où tous les méthodes sont publiques par défaut.

\subsubsection{\texttt{typedef}}
\paragraph{Définition}Attribuer un alias à un type. La déclaration la plus simple est:
\begin{scriptsize}
    \mycode\lstinputlisting[language=C++]{example/typedef.cpp}
\end{scriptsize}

\subsubsection{\texttt{template}}
\paragraph{Définition}Création de functions avec des différents types mais avec la même implementation. La déclaration la plus simple est:
\begin{scriptsize}
    \mycode\lstinputlisting[language=C++]{example/template.cpp}
\end{scriptsize}
Après avoir déclarer la function generic il faut déclarer les functions avec les différents types pour que le compilateur peut savoir comment le faire quand necessaire.

\subsection{\texttt{class}}
\paragraph{Définition}

\subsubsection{\texttt{objet}}
\paragraph{Définition}

\subsection{\texttt{methode}}
\paragraph{Définition}

\subsubsection{\texttt{public}}
\paragraph{Définition}

\subsubsection{\texttt{protected}}
\paragraph{Définition}

\subsubsection{\texttt{private}}
\paragraph{Définition}

\subsection{Héritage}
\paragraph{Défintion}


\section{Travail Dirigé}
\subsection{07/09/2022}
\paragraph{Présentation}Dans ce \href{https://perso.ensta-paris.fr/~bmonsuez/Cours/doku.php?id=in204:seances:seance1}{TD} on avait besoin d'implementer une \texttt{struct}, qui ressemble plutôt à une \texttt{class}, pour voir l'influence du marquer \texttt{const} dans le code.
\begin{scriptsize}
    \mycode
    \lstinputlisting[language=C++, linerange={1-5}]{TD/07_09_2022/counter.hpp}
\end{scriptsize}
\begin{scriptsize}
    \mycode
    \lstinputlisting[language=C++, linerange={1-5}]{TD/07_09_2022/main.cpp}
\end{scriptsize}

\newpage\subsection{14/09/2022}
\paragraph{Présentation}Dans ce \href{https://perso.ensta-paris.fr/~bmonsuez/Cours/doku.php?id=in204:seances:seance2}{TD} on avait besoin d'implementer deux \texttt{class} pour étudier comme les classes sont implementés et comme marché l'heritage entre deux classes.
\begin{scriptsize}
    \mycode
    \lstinputlisting[language=C++, linerange={1-5}]{TD/14_09_2022/Counter.hpp}
\end{scriptsize}
\begin{scriptsize}
    \mycode
    \lstinputlisting[language=C++, linerange={1-5}]{TD/14_09_2022/Double.hpp}
\end{scriptsize}
\begin{scriptsize}
    \mycode
    \lstinputlisting[language=C++, linerange={1-5}]{TD/14_09_2022/main.cpp}
\end{scriptsize}

\newpage\subsection{21/09/2022}
\paragraph{Présentation}Dans ce \href{https://perso.ensta-paris.fr/~bmonsuez/Cours/doku.php?id=in204:seances:seance3}{TD} on avait besoin d'implementer des \texttt{templates} pour étudier comme réduire et améliore le code avec la même implementation mais des types différents.
\begin{scriptsize}
    \mycode
    \lstinputlisting[language=C++, linerange={1-5}]{TD/21_09_2022/join.hpp}
\end{scriptsize}
\begin{scriptsize}
    \mycode
    \lstinputlisting[language=C++, linerange={1-5}]{TD/21_09_2022/main.cpp}
\end{scriptsize}

\end{document}