\documentclass{article}
% \usepackage{C:/Users/Admin-PC/Documents/git_repository/tpack/tpack}
\usepackage{C:/Users/guitr/Documents/git_repositories/tpack/tpack}
% \usepackage{/home/tr0fin0/git_repositories/tpack/tpack}


\title{IC204 - Programation par Objets}
\project{Résumé Théorique}
\author{Guilherme Nunes Trofino}
\authorRA{2022-2024}


\makeatletter
\begin{document}\selectlanguage{french}
\maketitle

\newpage\tableofcontents

\section{Introduction}
\subfile{C:/Users/guitr/Documents/git_repositories/classes_ensta/intro.tex}
% \subfile{C:/Users/Admin-PC/Documents/git_repository/classes_ensta/intro.tex}
% \subfile{/home/tr0fin0/git_repositories/classes_ensta/intro.tex}


\subsection{Informations Matière}
\paragraph{Présentation}Ce cours sera présenter par M. Jean-Baptiste Laurent qui a pour but d'étudier la Programmation Orienté à Objets: \url{https://perso.ensta-paris.fr/~bmonsuez/Cours} et reference externe: \url{https://www.youtube.com/watch?v=vLnPwxZdW4Y}.

\subsection{run code}
\paragraph{Définition}C++, et C aussi, est un langage compilé, c'est-à-dire que l'analyse syntaxique et sémantique du programme est réalisée intégralement avant l'éxecution et produition d'un code éxecutable, écrit dans un langage de bas-niveau permetant que l'ordinateur peut directement et efficacement exécuter.\\

\subsubsection{compilator}
\paragraph{Définition}Converte l'archive source dans une archive de byte code, parfois appeler code objet. Pour compiler un code il faut exécuter le commande suivante:
\begin{scriptsize}
    \mycode
    \begin{lstlisting}
    g++ file.cpp -o executable.exe && ./executable.exe
        g++ file.cpp -o executable.exe  | ./executable.exe
    \end{lstlisting}
\end{scriptsize}
Avec ces commandes le code sera d'abbord compilé et après executé.

\subsubsection{linker}
\paragraph{Définition}Après la création d'une archive de byte code le linker va faire reference à d'autres archives necessaires pour le code principale.

\subsection{\texttt{main}}
\paragraph{Définition}Inicialization du code se fera toujours avec la function \texttt{main} comme démontre:
\begin{scriptsize}
    \mycode\lstinputlisting[language=C++]{example/main.cpp}
\end{scriptsize}
La function retournerai 0 si le programme est éxecute sans aucune problème.

\subsubsection{\texttt{include}}
\paragraph{Définition}Quand une programme utilise des functions ou classes externes à le code il faut les incluire dans le code principale comme démontré:
\begin{scriptsize}
    \mycode\lstinputlisting[language=C++]{example/include.cpp}
\end{scriptsize}
C'est important de diviser et séparer les archives du code pour qu'il soit plus facile à comprendre et à l'éditer pendant le dévélopment.\\

\noindent Dans C++ il faut avoir l'extention \texttt{.cpp} ou \texttt{.hpp} pour convetion.

\subsubsection{\texttt{functions}}
\paragraph{Définition}Quand un programme a besoin d'estoquer seulement des functions necessaires pour la \texttt{main.cpp} c'est recomende de créer un archive \texttt{functions.hpp} et estoquer les informations là dedans pour mieux diviser les archives.
\subsection{\texttt{types}}
\subsubsection{\texttt{enumerate}}
\paragraph{Défition}Une enumerate est un nouveau type listant des éléments ne correspondant pas un type particulier. La déclaration la plus simple est:
\begin{scriptsize}
    \mycode\lstinputlisting[language=C++]{example/enum.cpp}
\end{scriptsize}

\subsubsection{\texttt{struct}}
\paragraph{Définition}Une struct permet de définir un nouveau type de variables regroupant plusieurs variables de types différents. La déclaration la plus simple est:
\begin{scriptsize}
    \mycode\lstinputlisting[language=C++]{example/struct.cpp}
\end{scriptsize}
Struct c'est un type de \texttt{class} où tous les méthodes sont publiques par défaut.

\subsubsection{\texttt{typedef}}
\paragraph{Définition}Attribuer un alias à un type. La déclaration la plus simple est:
\begin{scriptsize}
    \mycode\lstinputlisting[language=C++]{example/typedef.cpp}
\end{scriptsize}

\subsubsection{\texttt{template}}
\paragraph{Définition}Création de functions avec des différents types mais avec la même implementation. La déclaration la plus simple est:
\begin{scriptsize}
    \mycode\lstinputlisting[language=C++]{example/template.cpp}
\end{scriptsize}
Après avoir déclarer la function generic il faut déclarer les functions avec les différents types pour que le compilateur peut savoir comment le faire quand necessaire.

\subsection{Mémoire}
\paragraph{Définition}Langages compilés sont, d'habitude, plus efficaces avec l'usage de mémoire car, la plus part du temps, l'utilisateur doit préciser la quantité de mémoire necessaire.
\subsubsection{\texttt{pointers}}
\paragraph{Définition}Variable particulière servant à stocker l'adresse en mémoire centrale d'une variable comme démontré:
\begin{scriptsize}
    \mycode\lstinputlisting[language=C++]{example/pointer.cpp}
\end{scriptsize}

\subsubsection{Allocation Statique}
\paragraph{Définition}

\subsubsection{Allocation Dynamique}
\paragraph{Définition}

\subsection{C++ 20}
\paragraph{Définition}C'est necessaire de préciser la bonne version du compilateur parmi le commande suivante:
\begin{scriptsize}
    \mycode
    \begin{lstlisting}
    g++ -std=c++20 file.cpp -o executable.exe && ./executable.exe
    g++ -std=c++2a file.cpp -o executable.exe && ./executable.exe
    \end{lstlisting}
\end{scriptsize}
Il y a deux options et il faut essayer lequel marché sur sa machine.

% \subsubsection{\texttt{constraint}}
% \paragraph{Définition}
% \begin{scriptsize}
%     \mycode\lstinputlisting[language=C++]{example/constraint.cpp}
% \end{scriptsize}

\subsubsection{\texttt{concept}}
\paragraph{Définition}Il y a plusieurs manières d'utiliser le concept parmi lequel les deux les plus commans sont:
\begin{scriptsize}
    \mycode\lstinputlisting[language=C++]{example/concept.cpp}
\end{scriptsize}


\section{Programation Orientaté à Objets}
\paragraph{Définition}

\subsection{\texttt{class}}
\paragraph{Définition}On considere une implementation generale de \texttt{class} comme présenté:
\begin{scriptsize}
    \mycode\lstinputlisting[language=C++]{example/class.cpp}
\end{scriptsize}
Par défaut tout que n'a pas de modificateur de visibilité, \texttt{acess modifier} sera considere comme prive.


\subsubsection{\texttt{objet}}
\paragraph{Définition}

\subsubsection{\texttt{constructor}}
\paragraph{Définition}Function qui inicialize l'objet et doit suivre trois lois:
\begin{enumerate}[noitemsep]
    \item It is a method with no return type;
    \item It has the same name of it's class;
    \item It must be, in most cases, public;
\end{enumerate}
Le compilateur ira, par défaut, généré une constructeur vide pour qu'un \texttt{class} peut être inicialisé mais après inicialisé une constructeur non vide le constructeur vide n'existira plus

\subsubsection{\texttt{destructor}}
\paragraph{Définition}

\subsection{\texttt{encapsulation}}
\paragraph{Définition}

\subsubsection{\texttt{public}}
\paragraph{Définition}

\subsubsection{\texttt{protected}}
\paragraph{Définition}

\subsubsection{\texttt{private}}
\paragraph{Définition}

\subsection{\texttt{abstraction}}
\paragraph{Définition}

\subsection{\texttt{inheritance}}
\paragraph{Définition}

\subsection{\texttt{polymorphism}}
\paragraph{Définition}


\section{Good Practices}
\subsection{\texttt{namespace}}
\paragraph{Définition}\url{https://www.youtube.com/watch?v=etQX4Mme2f4}

\section{Travail Dirigé}
\subsection{Séance 07/09/2022}
\paragraph{Présentation}Dans ce \href{https://perso.ensta-paris.fr/~bmonsuez/Cours/doku.php?id=in204:seances:seance1}{TD} on avait besoin d'implementer une \texttt{struct}, qui ressemble plutôt à une \texttt{class}, pour voir l'influence du marquer \texttt{const} dans le code.
\begin{scriptsize}\mycode
    \lstinputlisting[language=C++, linerange={1-5}]{TD/2022_09_07/counter.hpp}
\end{scriptsize}
\begin{scriptsize}\mycode
    \lstinputlisting[language=C++, linerange={1-5}]{TD/2022_09_07/main.cpp}
\end{scriptsize}

\newpage\subsection{Séance 14/09/2022}
\paragraph{Présentation}Dans ce \href{https://perso.ensta-paris.fr/~bmonsuez/Cours/doku.php?id=in204:seances:seance2}{TD} on avait besoin d'implementer deux \texttt{class} pour étudier comme les classes sont implementés et comme marché l'heritage entre deux classes.
\begin{scriptsize}\mycode
    \lstinputlisting[language=C++, linerange={1-5}]{TD/2022_09_14/Counter.hpp}
\end{scriptsize}
\begin{scriptsize}\mycode
    \lstinputlisting[language=C++, linerange={1-5}]{TD/2022_09_14/Double.hpp}
\end{scriptsize}
\begin{scriptsize}\mycode
    \lstinputlisting[language=C++, linerange={1-5}]{TD/2022_09_14/main.cpp}
\end{scriptsize}

\newpage\subsection{Séance 21/09/2022}
\paragraph{Présentation}Dans ce \href{https://perso.ensta-paris.fr/~bmonsuez/Cours/doku.php?id=in204:seances:seance3}{TD} on avait besoin d'implementer des \texttt{templates} pour étudier comme réduire et améliore le code avec la même implementation mais des types différents.
\begin{scriptsize}\mycode
    \lstinputlisting[language=C++, linerange={1-5}]{TD/2022_09_21/join.hpp}
\end{scriptsize}
\begin{scriptsize}\mycode
    \lstinputlisting[language=C++, linerange={1-5}]{TD/2022_09_21/main.cpp}
\end{scriptsize}

\newpage\subsection{Séance 28/09/2022}
\paragraph{Présentation}Dans ce \href{https://perso.ensta-paris.fr/~bmonsuez/Cours/doku.php?id=in204:seances:seance4}{TD} on avait besoin d'implementer des \texttt{class} avec \texttt{template} pour étudier comment les classes peut-être dépendre d'une \texttt{type} externe.
\begin{scriptsize}\mycode
    \lstinputlisting[language=C++, linerange={1-5}]{TD/2022_09_28/Number.hpp}
\end{scriptsize}
\begin{scriptsize}\mycode
    \lstinputlisting[language=C++, linerange={1-5}]{TD/2022_09_28/Array.hpp}
\end{scriptsize}
\begin{scriptsize}\mycode
    \lstinputlisting[language=C++, linerange={1-5}]{TD/2022_09_28/Vector.hpp}
\end{scriptsize}

\begin{scriptsize}\mycode
    \lstinputlisting[language=C++, linerange={1-5}]{TD/2022_09_28/main.cpp}
\end{scriptsize}
difference entre array and Vector

\newpage\subsection{Séance 05/10/2022}
\paragraph{Présentation}Dans ce \href{https://perso.ensta-paris.fr/~bmonsuez/Cours/doku.php?id=in204:seances:seance5}{TD} on avait besoin d'implementer nouvelles functions pour les \texttt{operator}'s the une classe.
\begin{scriptsize}\mycode
    \lstinputlisting[language=C++, linerange={1-5}]{TD/2022_10_05/Number.hpp}
\end{scriptsize}
\begin{scriptsize}\mycode
    \lstinputlisting[language=C++, linerange={1-5}]{TD/2022_10_05/main.cpp}
\end{scriptsize}

\newpage\subsection{Séance 12/10/2022}
\paragraph{Présentation}Dans ce \href{https://perso.ensta-paris.fr/~bmonsuez/Cours/doku.php?id=in204:seances:seance6}{TD} on avait besoin d'étudier \href{https://isocpp.org/blog/2021/11/cpp-20-concepts}{C++20} et c'est concepts sur \texttt{constraint} et \texttt{concept}.
\begin{scriptsize}\mycode
    \lstinputlisting[language=C++, linerange={1-5}]{TD/2022_10_12/main.cpp}
\end{scriptsize}

\newpage\subsection{Séance 19/10/2022}
\paragraph{Présentation}Dans ce \href{https://perso.ensta-paris.fr/~bmonsuez/Cours/doku.php?id=in204:seances:seance7}{TD} on avait besoin d'étudier \href{https://www.youtube.com/watch?v=kjEhqgmEiWY}{exceptions} et c'est concepts sur \texttt{try} et \texttt{catch}.
\begin{scriptsize}\mycode
    \lstinputlisting[language=C++, linerange={1-5}]{TD/2022_10_19/main.cpp}
    % TODO faire les codes et étudier le sujet
\end{scriptsize}

\newpage\subsection{Séance 26/10/2022}
\paragraph{Présentation}Dans ce \href{https://perso.ensta-paris.fr/~bmonsuez/Cours/doku.php?id=in204:seances:seance8}{TD} on avait besoin d'étudier \href{https://stackoverflow.com/questions/2659116/how-does-virtual-inheritance-solve-the-diamond-multiple-inheritance-ambiguit}{diamond inheritance} et c'est concepts sur \texttt{inheritance} et \texttt{polymorphisme}.
\begin{scriptsize}\mycode
    \lstinputlisting[language=C++, linerange={1-5}]{TD/2022_10_26/main.cpp}
\end{scriptsize}
\paragraph{Présentation}Utilisation de programmation parallele. \href{https://github.com/JBL-Repo/IN204/blob/main/cours_recap.pdf}{matériaux}.
\begin{scriptsize}\mycode
    \lstinputlisting[language=C++, linerange={1-5}]{TD/2022_10_26/reference_TD_1.cpp}
\end{scriptsize}

\newpage\subsection{Séance 15/11/2022}
\paragraph{Présentation}Dans ce TD on avait besoin d'essayer des fonctions avec des \texttt{threads}. Pour ça on inicialise une número arbitrer de threads qui vont afficher des mensages avec des variables locaux et variables partages/
\begin{scriptsize}\mycode
    \lstinputlisting[language=C++, linerange={1-5}]{TD/2022_11_15/main.cpp}
\end{scriptsize}
Pour ça c'est necessaire d'ajouter une flag à la compilation:
\begin{scriptsize}
    \mycode
    \begin{lstlisting}
    g++ -std=c++11 -pthread main.cpp -o main.exe | ./main.exe
    \end{lstlisting}
\end{scriptsize}
https://stackoverflow.com/questions/15632198/c11-include-thread-gives-compile-error
https://medium.com/swlh/c-thread-synchronization-at-the-restaurant-ab0d125a0b7b
\end{document}