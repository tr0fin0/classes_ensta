\documentclass{article}
% \usepackage{C:/Users/Admin-PC/Documents/git_repository/tpack/tpack}
\usepackage{/home/tr0fin0/git_repositories/tpack/tpack}


\title{IC204 - Programation par Objets}
\project{Résumé Théorique}
\author{Guilherme Nunes Trofino}
\authorRA{2022-2024}


\makeatletter
\begin{document}\selectlanguage{french}
\maketitle

\newpage\tableofcontents

\section{Introduction}
\subfile{/home/tr0fin0/git_repositories/classes_ensta/intro.tex}
% \subfile{C:/Users/Admin-PC/Documents/git_repository/classes_ensta/intro.tex}\paragraph{Présentation}


\subsection{Informations Matière}
\paragraph{Présentation}Ce cours sera présenter par M. ... qui a pour but d'étudier la Programmation Orienté à Objets.

\subsection{run code}
\paragraph{Définition}C++, et C aussi, est unn langage compilé, c'est-à-dire que l'analyse syntaxique et sémantique du programme est réalisée intégralement avant l'éxecution et produit un code dit éxecutable, code écrit dans un langage de très bas-niveau permetant que l'ordinateur peut directement et efficacement exécuter.\\

\begin{scriptsize}
    \myStyleCPP
    \begin{lstlisting}
    g++ <code name>.cpp -o <executable name>.exe && ./ <executable name>.exe
    \end{lstlisting}
\end{scriptsize}
% TODO chercer définition de compilateur

\subsection{Code}
\subsubsection{\texttt{main}}
\paragraph{Définition}Inicialization du code se fera avec...
\begin{scriptsize}
    \myStyleCPP
    \lstinputlisting[]{main.cpp}
\end{scriptsize}

\subsubsection{\texttt{include}}
\paragraph{Définition}


\newpage\section{\texttt{class}}
\paragraph{Définition}

\subsection{\texttt{objet}}
\paragraph{Définition}

\subsection{\texttt{methode}}
\paragraph{Définition}

\subsubsection{\texttt{visibilité}}
\paragraph{Définition}
\end{document}