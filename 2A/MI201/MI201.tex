\documentclass{article}
\usepackage{C:/Users/guitr/Documents/git_repositories/tpack/tpack}
% \usepackage{C:/Users/Admin-PC/Documents/git_repository/tpack/tpack}
% \usepackage{/home/tr0fin0/git_repositories/tpack/tpack}


\title{MI201 - Apprentissage Automatique}
\project{Résumé Théorique}
\author{Guilherme Nunes Trofino}
\authorRA{2022-2024}


\makeatletter
\begin{document}\selectlanguage{french}
\maketitle


\newpage\tableofcontents

\section{Introduction}
\subfile{C:/Users/guitr/Documents/git_repositories/classes_ensta/intro.tex}
% \subfile{C:/Users/Admin-PC/Documents/git_repository/classes_ensta/intro.tex}
% \subfile{/home/tr0fin0/git_repositories/classes_ensta/intro.tex}


\subsection{Information Matier}
\paragraph{Référence}
\url{https://cs231n.github.io/python-numpy-tutorial/}

\subsection{Modélisations Basiques}
\subsubsection{Régression Linéaire}
\paragraph{Définition}

\subsubsection{Bayésienne}
\paragraph{Définition}

\subsubsection{Bayésienne Naïve}
\paragraph{Définition}


\section{Apprendisage Automatique}
\paragraph{Définition}L'apprentissage automatique est une démarche de conception d'un prédicteur et par une modélisation ou programmation non explicité à partir d'exemples.

\subsection{Apprentissage}
\paragraph{Définition}
\subsubsection{Inicialization}
\subsubsection{Visualization}
\subsubsection{Trainning}


\subsection{Prédiction}
\paragraph{Définition}
\subsubsection{Analysis}

\subsection{Surapprentissage}
\paragraph{Définition}En anglais, overfitting, veut dire que le modèle propose est très specifique et portant est mauvais pour la generalisation des données de validation.
\subsubsection{Limiter Surapprentissage}
\paragraph{Définition}Surapprentissage est une problème que doit être eviter et pour ça les méthodes suivants peuvent être implementer:
\begin{enumerate}[noitemsep]
    \item ajouter des données;
    \item réduire le nombre de caractéristiques;
\end{enumerate}



\section{Supervised Learning}

\subsubsection{Prédiction}


\subsection{La Généralisation}
\paragraph{Définition}


\section{Travail Dirigé}
\subsection{Séance 10/11/2022}
\paragraph{Définition}

\end{document}