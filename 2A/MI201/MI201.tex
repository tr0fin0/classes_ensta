\documentclass{article}
\usepackage{C:/Users/guitr/Documents/git_repositories/tpack/tpack}
% \usepackage{C:/Users/Admin-PC/Documents/git_repository/tpack/tpack}
% \usepackage{/home/tr0fin0/git_repositories/tpack/tpack}


\title{MI201 - Apprentissage Automatique}
\project{Résumé Théorique}
\author{Guilherme Nunes Trofino}
\authorRA{2022-2024}


\makeatletter
\begin{document}\selectlanguage{french}
\maketitle


\newpage\tableofcontents

\section{Introduction}
\subfile{C:/Users/guitr/Documents/git_repositories/classes_ensta/intro.tex}
% \subfile{C:/Users/Admin-PC/Documents/git_repository/classes_ensta/intro.tex}
% \subfile{/home/tr0fin0/git_repositories/classes_ensta/intro.tex}


\subsection{Information Matier}
\paragraph{Référence}
\url{https://cs231n.github.io/python-numpy-tutorial/}


\subsection{Procédure}
\paragraph{Définition}L'apprentissage automatique est une démarche de conception d'un prédicteur et par une modélisation ou programmation non explicité à partir d'exemples.

\subsubsection{Apprentissage}
\paragraph{}
$\arg\text{max}$
$\arg_{\text{min}}$

\subsubsection{Prédiction}


\subsection{La Généralisation}
\paragraph{Définition}


\section{Travail Dirigé}
\subsection{Séance 10/11/2022}
\paragraph{Définition}

\end{document}