\documentclass{article}
\usepackage{C:/Users/guitr/Documents/git_repositories/tpack/tpack}
% \usepackage{C:/Users/Admin-PC/Documents/git_repository/tpack/tpack}
% \usepackage{/home/tr0fin0/git_repositories/tpack/tpack}


\title{MI201 - Apprentissage Automatique}
\project{Résumé Théorique - SVM}
\author{Guilherme Nunes Trofino}
\authorRA{2022-2024}


\makeatletter
\begin{document}\selectlanguage{french}
\maketitle
\setlength{\parindent}{0pt}

\newpage\tableofcontents

\section{Introduction}
\subfile{C:/Users/guitr/Documents/git_repositories/classes_ensta/intro.tex}
% \subfile{C:/Users/Admin-PC/Documents/git_repository/classes_ensta/intro.tex}
% \subfile{/home/tr0fin0/git_repositories/classes_ensta/intro.tex}

\section{Apprentissage}
\subsection{Algorithme}
On considère que l'algorithme peut être définie par la définition suivante:
% \subsection{SVM, Support vector machine}
% \paragraph{Définition}The optimization problem SVM training solves has two terms:
% \begin{enumerate}[noitemsep]
%     \item A regularization term that benefits "simpler" weights;
%     \item A loss term that makes sure that that the weights classify the training data points correctly;
% \end{enumerate}
% C is just the balance between the importance of these to terms. If C is high you're giving a lot of weight to (2), if C is low you're giving a lot of weight to (1).

% \paragraph{Surapprentissage}En générale augmenter le paramètre C ira ammeliorer la précision pour les données de trainning mais pour les données de testing la précision sera faiable donc le surapprentissage augmente.
\begin{definition}
    .
\end{definition}

Généralement on aura le comportement suivant:
\begin{table}[H]
    \centering\begin{tabular}{lll}
        k $\uparrow  $ & biais $\uparrow  $ & variance $\downarrow$\\
        k $\downarrow$ & biais $\downarrow$ & variance $\uparrow$\\
    \end{tabular}
    \caption{Comportement SVM}
\end{table}

\subsubsection{Avantages}
Dans ce cas, cet algorithme 

\subsubsection{Incovenients}
Dans ce cas, cet algorithme 

\subsubsection{Applications}
Cet algorithme est souvent utilisé 


% \subsection{Initialization}
% \subsection{Visualization}
% \subsection{Training}



% \section{Prédiction}
% \subsection{Analyses}
\end{document}