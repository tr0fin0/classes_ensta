\documentclass{article}
\usepackage{C:/Users/guitr/Documents/git_repositories/tpack/tpack}
% \usepackage{C:/Users/Admin-PC/Documents/git_repository/tpack/tpack}
% \usepackage{/home/tr0fin0/git_repositories/tpack/tpack}


\title{MI201 - Apprentissage Automatique}
\project{Résumé Théorique - Neural Network}
\author{Guilherme Nunes Trofino}
\authorRA{2022-2024}


\makeatletter
\begin{document}\selectlanguage{french}
\maketitle
\setlength{\parindent}{0pt}

\newpage\tableofcontents

\section{Introduction}
\subfile{C:/Users/guitr/Documents/git_repositories/classes_ensta/intro.tex}
% \subfile{C:/Users/Admin-PC/Documents/git_repository/classes_ensta/intro.tex}
% \subfile{/home/tr0fin0/git_repositories/classes_ensta/intro.tex}

\section{Apprentissage}
\subsection{Algorithme}
On considère que l'algorithme peut être définie par la définition suivante:
\begin{definition}
    .
\end{definition}

La choix de n déterminera la qualité de la prédiction. Si on réduire n la qualité de la prediction améliore et le sur-apprentissage diminue. Généralement on aura le comportement suivant:
\begin{table}[H]
    \centering\begin{tabular}{lll}
        n $\uparrow$   & biais $\downarrow$ & variance $\uparrow$\\
        n $\downarrow$ & biais $\uparrow  $ & variance $\downarrow$\\
    \end{tabular}
    \caption{Comportement Neural Network}
\end{table}

Comme fonction d'activation on considère souvent la fonction \textbf{relu} pour sa simplicité et capacité de reproduire d'autres fonctions:
\begin{definition}
    On considère que la \textbf{relu} sera donne pour:
    \begin{equation}
        \boxed{
            relu(x) = \max(x, 0)
        }
    \end{equation}
\end{definition}
C'est possible d'avoir plusieurs fonctions à partir de la \textbf{relu} comme 
\begin{equation}
    |x| = relu(x) + relu(-x)
\end{equation}
\begin{equation}
    \| \mathbf{x}  \|_{1} = |x_{1}| + |x_{2}| = 
    relu((1,0)\cdot\mathbf{x}) + relu((-1,0)\cdot\mathbf{x}) + 
    relu((0,1)\cdot\mathbf{x}) + relu((0,-1)\cdot\mathbf{x})
\end{equation}
\begin{equation}
    \max(x, y) = relu(x-y)\frac{x}{x-y} + relu(y-x)\frac{y}{y-x}
\end{equation}
\begin{equation}
    \min(x, y) = relu(x-y)\frac{y}{x-y} + relu(y-x)\frac{x}{y-x}
\end{equation}

Comme les réseaux de neurones sont beaucoup utilises on peut faire une modification de structure pour implémenter le \textbf{Deep Learning}:
\begin{definition}
    On utilise des \textbf{Neurones Convolutif}...

    \begin{remark}
        \textbf{Neurones Convolutifs} sont une variation de construction de réseaux de neurones où...
    \end{remark}
\end{definition}
\subsubsection{Avantages}
Dans ce cas, cet algorithme 

\subsubsection{Incovenients}
On peut citer:
\begin{enumerate}[noitemsep, rightmargin=\leftmargin]
    \item pour apprendre par coeur il suffit d'avoir $2nd + n + 1$ neurones;
    \begin{enumerate}[noitemsep]
        \item $2nd$: numéro d'entrées;
        \item $n$: numéro de données;
        \item $1$: sorti;
    \end{enumerate} 
\end{enumerate}

\subsubsection{Applications}
Cet algorithme est souvent utilisé 


% \subsection{Initialization}
% \subsection{Visualization}
% \subsection{Training}



% \section{Prédiction}
% \subsection{Analyses}
\end{document}