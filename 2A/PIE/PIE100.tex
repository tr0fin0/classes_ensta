\documentclass{article}
% \usepackage{C:/Users/Admin-PC/Documents/git_repository/tpack/tpack}
\usepackage{/home/tr0fin0/git_repositories/tpack/tpack}


\title{PIE101 - Projet d'Ingenieur en Equipe}
\project{Résumé Théorique}
\author{Guilherme Nunes Trofino}
\authorRA{217276}


\begin{document}\selectlanguage{french}
\maketitle

\newpage\tableofcontents

\section{Introduction}
\section{Le Bilan Émissions de GES}
\subsection{Facteur d'Emission}
mesurée par l'expérience
chercher le site https://www.bilians-ges.ademe.fr


\section{}
\subsection{Principes et Démarche}
\paragraph{Définition}
3 scopes
1: direct
2: indirect
3: indirect

activités en amont
emission possédes

\subsection{Energie Primaire}
\paragraph{Énergie Primaire}Désigne les différentes sources d'énergie disponibles dans la nature avant transformation. Elle englobe notamment l'énergie du vent, du soleil, de la chaleur terrestre, de l'eau stockée dans un barrage, des combustibles renouvelables ou fossiles.

\paragraph{Énergie Finale }Désigne l'énergie livrée au consommateur final pour satisfaire ses besoins après transformations par 'homme.\\

\noindent Entre l'énergie primaire et l'énergie primaire et l'énergie finale fournie aux consommateurs il s'opère des pertes lors d'opérations de transformation et de transport.\\

\noindent Quoi à faire
\begin{enumerate}
    \item \textbf{Réduire} le besoin;
    \item \textbf{Améliorer} l'efficacité énergétique;
    \begin{enumerate}[noitemsep]
        \item Gestion Énergétique;
        \item Froid Process;

        \item CVF;
        \item Air Comprimé;

        \item Eclairage;

        \item Chaleur Process;
    \end{enumerate}
    \item \textbf{Substituer} par un vecteur énergétique noins carboné;
    \item \textbf{Recourir} à des énergies bas carbone et renouvelables;
\end{enumerate}
En ce moment les auternatives les plus viables sont l'hydrogène, la géothermie, le gaz naturelle et d'autres. Il y a beaucoup a faire mais c'est possible d'y arriver aux objects du climat.\\

\noindent C'est important de comprendre la diffèrence entre moyanne dans la journée et dans l'annèe car il y a des diffèrences importantes de besoin de consumation et de production d'énergie, principalemente pour les sources renouvelables.



\end{document}