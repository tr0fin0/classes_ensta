\documentclass{article}
% \usepackage{C:/Users/Admin-PC/Documents/git_repository/tpack/tpack}
\usepackage{/home/tr0fin0/git_repositories/tpack/tpack}


\title{IC206 - Bases de Données}
\project{Résumé Théorique}
\author{Guilherme Nunes Trofino}
\authorRA{2022-2024}


\makeatletter
\begin{document}\selectlanguage{french}
\maketitle

\newpage\tableofcontents

\section{Introduction}
\subfile{/home/tr0fin0/git_repositories/classes_ensta/intro.tex}
% \subfile{C:/Users/Admin-PC/Documents/git_repository/classes_ensta/intro.tex}\paragraph{Présentation}


\subsection{Informations Matière}
\paragraph{Présentation}Ce cours sera présenter par M. Nicolas ANCIAUX qui a pour but d'étudier des Bases de Données avec le site \url{http://www-smis.inria.fr/~anciaux/ENSTA/IN206/}.

\section{Modélisation du Réel}
\paragraph{Définition}
\begin{enumerate}[noitemsep]
    \item Modèle Conceptuelle;
    \item Modèle Logique;
    \item Modèle Physique;
\end{enumerate}
Erreus:
\begin{enumerate}[noitemsep]
    \item Redondances;
    \item Valeurs Nulles;
    \item Difficulté de Gestion;
    \item Impossibilité de Répondre à Certaines Questions;
\end{enumerate}
Approche:
\begin{enumerate}
    \item \textbf{Définir l'Application}:
    \item \textbf{Définir les Données}:
    \item \textbf{Définir les Questions Nécessaires}:
    \item \textbf{Validation}:
    \item \textbf{Passer du MCD au MLD}:
    \item \textbf{Définir les Requêtes}:
    \item \textbf{Passer du MLD au MPD}:
\end{enumerate}

\end{document}