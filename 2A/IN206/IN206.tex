\documentclass{article}
\usepackage{../../tpack/document/tpack}


\title{IN206 - Bases de Données}
\project{Résumé Théorique}
\author{Guilherme Nunes Trofino}
\authorRA{2022-2024}


\makeatletter
\begin{document}\selectlanguage{french}
\maketitle

\newpage\tableofcontents

\section{Introduction}
\subfile{../../intro.tex}


\subsection{Informations Matière}
\paragraph{Présentation}Ce cours sera présenter par M. Nicolas ANCIAUX qui a pour but d'étudier des Bases de Données avec le site \url{http://www-smis.inria.fr/~anciaux/ENSTA/IN206/}.

\section{Modélisation du Réel}
\paragraph{Définition}
\begin{enumerate}[noitemsep]
    \item Modèle Conceptuelle;
    \item Modèle Logique;
    \item Modèle Physique;
\end{enumerate}
Erreus:
\begin{enumerate}[noitemsep]
    \item Redondances;
    \item Valeurs Nulles;
    \item Difficulté de Gestion;
    \item Impossibilité de Répondre à Certaines Questions;
\end{enumerate}
Approche:
\begin{enumerate}
    \item \textbf{Définir l'Application}:
    \item \textbf{Définir les Données}:
    \item \textbf{Définir les Questions Nécessaires}:
    \item \textbf{Validation}:
    \item \textbf{Passer du MCD au MLD}:
    \item \textbf{Définir les Requêtes}:
    \item \textbf{Passer du MLD au MPD}:
\end{enumerate}

\subsection{Caracteristiques}
\subsubsection{Entité}
\paragraph{Définition}Représentation d'un objet du monde réel par rapport au problème à modéliser peut donc être:
\begin{enumerate}
    \item \textbf{Abstraite}: une commande, une action;
    \item \textbf{Concrète}: une objet pyshique;
\end{enumerate}
\paragraph{Type d'Entité}Représentation d'un ensemble... . Normalment represente avec une figure carre.

\subsubsection{Association}
\paragraph{Définition}Représentation d'un lien non orienté entre plusieurs entités, normalmente relacionne avec les verbes. Normalement represente avec une figure rounde.

\paragraph{Type d'Association}Représentation d'un ensemble d'associations ayant la même sémantique et les mêmes caractéristiques.

\subsubsection{Propríetes}
\paragraph{Définition}Donnée élémentaire permettant décrire une entité ou une association, n'a pas besoin de le couper après.

\paragraph{Identifiant d'Entité}Une entité est identifiée de manière unique par au moins une propriété, généralement une.

\paragraph{Identifiant d'Association}On peut identifier une association par l'ensemble des identifiants des entités associées.

\subsubsection{Cardinalités}
\paragraph{Définition}Exprime le nombre minimum et maximum d'association(s) par entité. Il est indiqué sur chaque arc, entre le type d'entité et le type d'association concernées.

\subsection{Régles}
\paragraph{Définition}

\subsubsection{Respect des Règles de Gestion}
\subsubsection{Propriétés Élémentaires}
\subsubsection{Propriétés Répétitives}
\subsubsection{Propriétés sans Signification}
\subsubsection{Identifiants d'Entités}
\subsubsection{Dépendance Pleine de l'Identifiant}
\subsubsection{Entités et Associations}
\subsubsection{Pas de Dépendance Transitive}

\section{Travail Dirigé}
\subsection{Séance 15/09/2022}
\paragraph{Définition}

\subsection{Séance 29/09/2022}
\paragraph{Définition}Dans ce \href{http://www-smis.inria.fr/~anciaux/ENSTA/IN206/6-VM/}{TD} on avait besoin de télécharger et essayer d'utiliser le serveur SQL dans une VM et pour se connecter il fallait utiliser:
\begin{enumerate}[noitemsep]
    \item login: TP;
    \item mdp: oracle;
\end{enumerate}
C'est possible qu'il y avait une problème avec l'enconding de l'ordinateur. Pour être sûr change les configurations de clavier et choisi le clavier Américain International.\\

\noindent Après c'est necessaire d'ouvrir une terminale et tapper les commandes suivantes:
\begin{scriptsize}\mycode
    \begin{lstlisting}[language=Bash]
    sqlplus sys/oracle as sysdba;
    ALTER SESSION SET CONTAINER = XEPDB1;
    CREATE USER TP IDENTIFIED BY oracle;
    GRANT ALL PRIVILEGES TO TP;
    exit;
    \end{lstlisting}
\end{scriptsize}
\noindent Après il fallait tapper:
\begin{scriptsize}\mycode
    \begin{lstlisting}[language=Bash]
    sqldeveloper
    \end{lstlisting}
\end{scriptsize}
\subsubsection{SQL Developer}
\paragraph{Définition}Tout d'abbord il faut suivre les commandes suivantes:
\begin{enumerate}
    \item créer une nouvelle connexion NON: TP;
    \begin{enumerate}[noitemsep]
        \item login: TP;
        \item mdp: oracle;
    \end{enumerate}
    \item NON de service: XEPDB1;
\end{enumerate}
pas oblige de nomene la constraint mais c'est plus facile de le trouveer car le sql va donner une valeur alphanumerique pas facile de travailer avec
% https://www.w3schools.com/sql/sql_constraints.asp

\end{document}