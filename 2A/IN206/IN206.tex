\documentclass{article}
% \usepackage{C:/Users/Admin-PC/Documents/git_repository/tpack/tpack}
\usepackage{/home/tr0fin0/git_repositories/tpack/tpack}


\title{IC206 - Bases de Données}
\project{Résumé Théorique}
\author{Guilherme Nunes Trofino}
\authorRA{2022-2024}


\makeatletter
\begin{document}\selectlanguage{french}
\maketitle

\newpage\tableofcontents

\section{Introduction}
\subfile{/home/tr0fin0/git_repositories/classes_ensta/intro.tex}
% \subfile{C:/Users/Admin-PC/Documents/git_repository/classes_ensta/intro.tex}\paragraph{Présentation}


\subsection{Informations Matière}
\paragraph{Présentation}Ce cours sera présenter par M. Nicolas ANCIAUX qui a pour but d'étudier des Bases de Données avec le site \url{http://www-smis.inria.fr/~anciaux/ENSTA/IN206/}.

\section{Modélisation du Réel}
\paragraph{Définition}
\begin{enumerate}[noitemsep]
    \item Modèle Conceptuelle;
    \item Modèle Logique;
    \item Modèle Physique;
\end{enumerate}
Erreus:
\begin{enumerate}[noitemsep]
    \item Redondances;
    \item Valeurs Nulles;
    \item Difficulté de Gestion;
    \item Impossibilité de Répondre à Certaines Questions;
\end{enumerate}
Approche:
\begin{enumerate}
    \item \textbf{Définir l'Application}:
    \item \textbf{Définir les Données}:
    \item \textbf{Définir les Questions Nécessaires}:
    \item \textbf{Validation}:
    \item \textbf{Passer du MCD au MLD}:
    \item \textbf{Définir les Requêtes}:
    \item \textbf{Passer du MLD au MPD}:
\end{enumerate}

\subsection{Caracteristiques}
\subsubsection{Entité}
\paragraph{Définition}Représentation d'un objet du monde réel par rapport au problème à modéliser peut donc être:
\begin{enumerate}
    \item \textbf{Abstraite}: une commande, une action;
    \item \textbf{Concrète}: une objet pyshique;
\end{enumerate}
\paragraph{Type d'Entité}Représentation d'un ensemble... . Normalment represente avec une figure carre.

\subsubsection{Association}
\paragraph{Définition}Représentation d'un lien non orienté entre plusieurs entités, normalmente relacionne avec les verbes. Normalement represente avec une figure rounde.

\paragraph{Type d'Association}Représentation d'un ensemble d'associations ayant la même sémantique et les mêmes caractéristiques.

\subsubsection{Propríetes}
\paragraph{Définition}Donnée élémentaire permettant décrire une entité ou une association, n'a pas besoin de le couper après.

\paragraph{Identifiant d'Entité}Une entité est identifiée de manière unique par au moins une propriété, généralement une.

\paragraph{Identifiant d'Association}On peut identifier une association par l'ensemble des identifiants des entités associées.

\subsubsection{Cardinalités}
\paragraph{Définition}Exprime le nombre minimum et maximum d'association(s) par entité. Il est indiqué sur chaque arc, entre le type d'entité et le type d'association concernées.

\subsection{Régles}
\paragraph{Définition}

\subsubsection{Respect des Règles de Gestion}
\subsubsection{Propriétés Élémentaires}
\subsubsection{Propriétés Répétitives}
\subsubsection{Propriétés sans Signification}
\subsubsection{Identifiants d'Entités}
\subsubsection{Dépendance Pleine de l'Identifiant}
\subsubsection{Entités et Associations}
\subsubsection{Pas de Dépendance Transitive}

\section{Travail Dirigé}
\subsection{15/09/2022}
\paragraph{Définition}
\end{document}