\documentclass{article}
\usepackage{C:/Users/Admin-PC/Documents/git_repository/tpack/tpack}
% \usepackage{/home/tr0fin0/git_repositories/tpack/tpack}


\title{EA206 - Eeconomie Industrielle}
\author{Guilherme Nunes Trofino}
\authorRA{217276}
\project{Résumé Théorique}


\begin{document}\selectlanguage{french}
\maketitle

\newpage\tableofcontents
\section{Introduction}
\paragraph{Présentation}Ce cours sera présenter par M. Richard LE GOFF, Professeur de Sciences Économiques, qui a pour but d'étudier l'ensemble de utiles a disposition des entreprises pour améliorer sa performace peu importe ce secteur.
% bien arriver dans le mercado les utiles pour etre utilise dans quelque entriprise meme de de differente secteurs de l'économie


\paragraph{Produit}Chaque entreprise aura comme finalité produire un produit soit un produire finale ou un service, pas nécessairement pour le vendre. L'ONU a un produit qui ne sera pas vendu dans le marché.
% but la maniere de produire des produits et des services
% etude de la production quelque soit le produit finale, pas necessairement avec la necessite de la vendre


\paragraph{}pas seulement production marchande, industrie n'est pas necessairement de la part de production de biens de consume, pas seulement du secteur secondaire. le secteur de services est très dominannt dans quelques pays


\paragraph{}primeiere secondaire terciere, optimitaion de point de vue de la gestion, soit elle publique ou empreseriere.


\paragraph{}en france la presence de l'état est très presente dans la economie avec regulations dans beaucoup de secteurs dans la economie comme le transporte. regulation de comme il marcher


\paragraph{}innovation est important pour quelque emtreprise soit elle prive ou publique. considere l'ensemble des interdependences dans un espace dans le monde, national ou internationaux 


\paragraph{}transition ecologique affacte tout de le monde dans la planete des differents manieres dans le monde dans des differents niveaux

\subsection{Informations Iniciales}
\paragraph{Descriptif}
\paragraph{Objectif Pédagogique}
\paragraph{Programme Détaillé}paragdme considere dans ce cours sera SCP, % TODO chercher le significa
\paragraph{Méthode Pédagogique}Le cours sera divisé dans ces parts:
\begin{enumerate}
    \item Cours Magistraux;
    \item Etude de Cas et Conférences: éclairages sectoriels ou thématiques sur l'innovation;
\end{enumerate}

\subsection{Innovation}
\paragraph{Études de Cas}Dans différents entreprises il y a des différents défis à faire pour acomplir la régulametation environnmentale comme:
\begin{enumerate}
    \item BMW: Roulerons nous à l'hydrogène demain? Vers de nouvelles solutions pour une mobilité plus verte
    \item SAFRAN: Quel environnement de marché pour les innovations et produits destinés à l'aéronautique militaire, civile?
    \item SNCF: Quelle régulation pour quelles innovations et quelle concurrence pour le transport de voyageurs?
\end{enumerate}

la legislation si imposise dans tous les secteurs differenment decide dans chaque pays
les consomataires sont plus sensibles a la consumations dans le sens de la transition écologique

\subsection{L'Organization Industrial}
\paragraph{Définition}Les utiles de l'économie globale est important pour comprendre la situation de chaque évenement externe dans le monde comme:
\begin{enumerate}
    \item Changement Climatique: Le changement radical du Market Environnement s'impose à tous les acteurs industriles
    \item Guérre Russie - Ukraine:
\end{enumerate}

les soubalites des clients et la capacite du client de payer pour le produit
Aujourd'hui il y a des entreprises qui basent la qualité des produits pour attendre à les metes necessaires


\newpage\section{Outils et Méthodes}
\paragraph{Définition}C'est necessaire de comprendre l'historie de l'économie pour comprendre comment la publique marche aujourd'hui. Le pouvoir change des mains dans les années pour les personnes, entreprises et governements.
le marche est un place dans une espace ou il y a le echange des produits

\subsection{Définitions Théoriques}
\paragraph{Organisation Industrielle}Propose par MASON et CLARK dans 1941 AEA IO reconnue comme une discipline posser comme ça: l'étude des structures de marché représente le principal objet de l'organisation industrielle, principe de leur fonctionnnement

il y a différence entre théporie normative et théorique positive.

la science est hypotetique ou indutive, en faisant des experiences ont essaye quelque chose et apres on voir les resultas pour proporser comment chaque changement pour influencer le résultat finale. dans l'économie est très diffice de metrises tous les variables pour controler presicement le resultat finale car le functionement dans le marche est tres interconnecté et pas concise en ce functionement

aujourd'hui il y a de la recherce pour comprendre comme chaque individuel pour prendre ces decitions avec les experiments controles dans le cadre de une laboratoire emais cest seulement utile pour chaque personnes mais pas dans le marche comment dans tout une marche ou un systeme complexe

étude de cas est très utilise mauvaisement dans le marche aujourd'hui mais très effective pour comprendre une situation et comprendre les situations avec un exemple pratique et concrete. c'est très utilise pour comprendre comme le business marcher

teorie normative propose une norme pour les dections dans le sens plus logique que les decitions

actioneres et les managers


le marche du travail est différent du marche des produits plus connait pour les personnes mais aussi important pour comprendre l'économie.
le marche du travail est différent du qu'est courrant dans le quotidien, la force de travail et la demande pour le travail. la vision marxiste du travail est impose dans l'étude de léconomie


SCP  
structures de la concurrence
dans le temps et dans l'espace pour compreendre le problème dont ont va travail. par exemple le banque dans la france est différent dans le banque en allemand
les comportements Conducts ou strategies des firmes et 
les strategies, comme faire quelque chose en étudiant la concurrence pour prendre les mieux décitions
les performances des firmes et des industries. 
la technogie  sera limitant dans les industries pour determiner comme le marche va marcher dans le temps pour ce quon a besoin pour que on faire marcher dans le temps et dans le

pas toujours SCP et après il y a l'aternative SPC car il y a des différent formes des posser les questions dans le temps pour faire marcher 

la performance est importante de annalyser dans les entreprises dans un espace et un temps mas c'est important de definir quels sont les aspects necessaires et importants pour evaleur da performance et evaleur ce qu'est bonne et ce que n'est pas. il peut varier dans les decitions et la perspective d'une entreprise.

cela consitue en soit une méthode d'étude de tout type d'industrie de la banque à l'automobile en passant par le tourisme ou l'agroalimentaire


dans la macro et micro economie 
les conditions de base se imposent à tous les agents dans une marche. il faut considerer, basiquemente les facteurs suivants:
\begin{enumerate}
    \item travail: 
        habitudes et aptitudes de travail d'organisation
        de management
        niveau educatif
        etat des relations sociales
        negociation
    \item Capital Financier
        actionnariat
        disponibilite financieres
    \item Bâtiments
        technologies
        outils de productions
    \item Ressources
        etat des stocks
        possiblites de substitution
\end{enumerate}

dans une vision plus contemporaine, l'information fiabilité transparence du systéme fluidité infrastructures informationnes

les structures du point de vue de la théorie standar sont constituées situations de concurrence:
\begin{enumerate}
    \item Modeles Économiques:
    \begin{enumerate}
        \item monopole
        \item monopsone
        \item oligopole
        \item intégration verticale
        \item quasiintégration
        \item structures conglomérales
    \end{enumerate}
    C'est important de comprendre comme le marche est constituées par apport de force, comment est "l'atomatization" du mercard, comment chaque agent contribuer pour le functionement du marche
    \item concurrence plus ou moins atomissée à la fois du côté de l'offre et de la demande
    \item imperfection de la concurrence liée au différenciations des produtis à la wprésence de barriès à entrée
\end{enumerate}
aussi dans une apprche plus contemporaine les structures incluent aussi les échanges interindustriels perçus notamment grâce aux données mésoéconomiques

% TODO considerer chaque seance comme en section? je ne sais pas

Les comportements des firmes ou stratégies ou conducts recouvrent l'ensemble des stratégies de l'entreprise en termes de prix, de production de recherche et développement, de différenciation....

section 1 quelques définitions théoriques de l'organisation industrielle. organisation industrielle, industrial organization BAIN (1951 et 1954) pose la séquence SCP et une méthode
\begin{enumerate}
    \item Structure naturelle d'un marché
    \item Stratégies industrielles pour accroîte la concentration industrielle en érigeant des barrières à l'entréecritère de performance des marchés: profitabilité
    RNC/CAHT: résultat net comptable au chiffre d'affaires hores taxes
\end{enumerate}
reconnaisance sociale de l'IO aux USA. Il n'y a rien de naturelle dans l'economie mais c'est utilise dans l'economie juste pour preciser qu'il des limites dans un espace, dans un temps et dans un contexte.\\

mais les limites de l'IO, alors confondue avec le paradigme SCP, apparaissent:
\begin{enumerate}
    \item les CdB, Market Environment, déterminants des strucctures, sont exogènes;
    \item SC, structures - Conducts ou CS, conducts-structures;
    \item La pluralité de critères de performances implique la multicité des structures de la concurrence à considérer, plusieurs échelles;
    \item Absence de relations inter-industrielles dans le paradigme SCP;
\end{enumerate}

en réaction à partir des années 70 (SHEPHERD et GALE, 1972 - Boston COnsulting GRoup - PHILLIPS 1978), mise en évidence du rôle crucial de l'apprentissage des firmes dans la performance atteinte en termes de profitabilité du secteur

c'est important que les régles soit construir par tous les parts du marche, les especialistes, les avocats et les travail. les conditions de base sont importes pour comprendre comme le marche ira marcher

en d'autres termes si l'apprentissage est crucial alors les stratégies dominent les structures puisqu'elles dérterminent las profitabilité de toute une industriele paradigme

le paradigme SCP est contesté au profit du paradigme CSP puis débassé par certains

Learding by doing learning by using apprentissage et routines paradigmes et trajectoires technologiques
l'innovation est un processus combinant techno-push et demand pull

la réalité des marchés est très complexe et nécessite un enrichissement théorique pour établir un pont entre la réalité économique et la théorie économique

la réglementation s'avère quasiment impossible avec les outils de l'UI à cause de ses limites la problèmatique de la délimitation de l'industrie ou du marché par 'offre ou pas la demande s'impose alors comme le coeur de l'economie industrielle relevant Market

la definition du marche pertinent est importante pour comprendre si uhn fusion sera bien ou mauvais pour la concorrunce dans le marche. s'u marche est tres grand pour que un nouvel entreprise arrive le gouvernement peut  dire que les entreprises doivent partager les recurses commment des operateures de telecom

marchés contestables 

la déréglementation s'imposera dans les années 80, 90
la réglementation ne fera son retour que dans les années 2000


entre 1970 et actuellement les interactions entre l'ei l'IO l'Economie Internationale et la Microéconomie sont nombreuses et très prolifiques
L'IO est un objet d'é'tude economies d'envergure
etude de la relation comportement strucure concurrence imparfaite dynamique ou statique avec les travaux de CLARK et DAVIES jeux dynamiques à information incomplète

classes de facteurs élargissant le champ stratégique l'irréversiblité l'indivisibilité les rigidités, l'incertitude et l'asymétrie groupe stratégiquess et stratégies de préemption segmentation différentiation

partagement des caracteristiques entre differents produits qui permet de developer des differente produits avec les mêmes recurses


dans la theorie neoclassique standart il n'y a pas de place pour l'estrategie car l'entreprise est considere comment un boite noir et on déconnece ce qu'il y a dedans

l'inversibilite des quelques process sont importants pour comprendre que quelques projets ne pouvont pas marcher

l'indivisiblite n'est pas possible de investir à la motie

classes de facteurs élargissant le champ stratégique l'irréversiblité, l'indivisibilité, les rigidités, l'incertitude et l'asymétrie groupes stratégiques et stratégies de préemption, segmentation, différentiation

risque est probabilisade et l'incertitude ne peut pas se metrisser 

asymétrie de pouvoir ses est different dans l'echale du produit. dentre les especialistes et de lieu ou chaque agente a des differents connaisances dans ce sujet

quelques définitions théoriques de l'organisation industrielle



approche théorique des systèmes industriels plus fonctionnaliste et théorique courrant de l'évolutionnisme positif
analogie avec la nature et l'evolution des idees


etude de la croissance des firmes par les coûts de transaction les côut de l'organisation des groupes stratégiques des chaînes de valeur


l'industrial economics se réfère à la théorie de la concurrence, à l'optimum à l'équilibre et donc à un paradigme bien affirmé
l'industrial organization se réfère quant à elle à une théorie de la concurrence mal affirmée ou plutôt à un paradigme mal affirmé appuyé sur le système et la dynamique

instrumelazer les conceptes pour etre capable de utiliser ces parametres et analyser les decistions prendre par quelqu'un d'autres


section 2
de l'organissation industrielle à la logique d'organisation economique
derrière l'organisation économique on peut penser à :
l'organisation industrielle en tant que disciplines
l'organisation de l éntreprise à l'organisation par opposition au marché
fordisme toyotisme firmes réseauxorganisation du travail 
organisation de la prodution 
organization des échanges inter-entreprises

méthodes agiles t organisation agiles: a cote de un organisation hierarchique

proposition de recours à la logique d'organisation economique comme objet d'étude pour dépasser la rigidité paradigmatique de l'IO et la fragilité conceptuelle de l'OE et finalement comprendre
    la combinaison productive
    l'organisation du travail de la production des échanges SCP CSP
en mobilisant les apports de BAIN à PORTER en passant par chandler, baumol, panzar, willig, coase, williamson

proposition de définition de l'objet d'étude
la logique d'organisation economique: caractérise une économie ou une activité en termes micro et marcoéconomiques en décrivant les conditions de base, les structures, les comportements, les performances et leurs logiques d'influences dynamiques intra et intertemporelles
interdependences economiques poissent des forces et faiblaisses pour  les entreprices pour les contrats et les decistions precises par le pays

pour ne se tromper pas on peut connecter tout mais ce sera plus dificile à comprendre et à epxlique et à previsioner

les conditions de base
les comportements des firmes ou stratégies ou conducts recouvrent l'ensemble des stratégies de l'entreprises en termes de prix, de production de recherche et développement, de différenciation, d'innovation de marketing, de publicité, d'attitude juridique

elles relèvent notamment des cinq familles stratégiques suivantes:
    domination par les coûtsdifférenciation
    concentration
    repli sur le coeur de compétences
    copération

diagram de porter
https://tex.stackexchange.com/questions/10060/how-to-draw-kiviat-diagrams



\newpage\section{Conférence}
\subsection{Innovation de la Défense}
\paragraph{Présentation} Cette conférence au lieu 2022/09/02.

\paragraph{Définition}Explication comment l'agence ira marcher pendant les années à l'avenir pour mantenir la superitaire militaire de la France dans l'Europe et dans le Monde.

\paragraph{Opnition}La présentation avait beaucoup de chiffres et des vídeos qui donnent plus de dynamisme à comprendre les points principales de chaque part

\subsubsection{Missons et Moyens}
\paragraph{Définitions}En ce moment il y a des missions suivants:
\begin{enumerate}[noitemsep]
    \item Orienter et piloter l'innovation;
    \item Détecter, expérimenter et adaptar l'innovation;
    \item Accélérer et déployer au profit;
\end{enumerate}

\subsubsection{Accélérer les Projets}
\paragraph{Définition}Le mynistére considere la division suivant:
\begin{enumerate}[noitemsep]
    \item \textbf{Technologies de Défense};
    \item \textbf{Accélération};
    \item \textbf{Innovation Partipative};
    \item \textbf{Recherche};
\end{enumerate}
Dans ce cas tous les agents dans le marche sont envolvés dans l'investiment pour l'état: des de les petites et moyennes entreprises, PME, jusqu'à les ...

\subsubsection{Orientation de l'Innovation}
\paragraph{Défition}Pendant le 2020 il avait la publication de l'instruction ministérielle de l'innovation de défense avec les acteurs, gouvernance et processus d'orientation.

\paragraph{Projets}DROID vient de la reference à Star Wars, car les responsables aiment la science fiction, qui veut dire: Document de Référence de l'orientation de l'innovation de défense avec ses caracteristiques suivants:
\begin{enumerate}[noitemsep]
    \item 80 Millard d'euros sur les nouvelles formes d'innovation;
    \item Donne les évolutions validées par la MINARM pour l'A2PM;
\end{enumerate}

\subsubsection{Innovation Défense}
\paragraph{Dérfinition}Le laboratoire responsable pour faire de la recherche et la validation des conceptes en estrassant le projet à des situations le plus réales possibles et garantir le projet marchera dans son application avec les parts suivants:
\begin{enumerate}[noitemsep]
    \item \textbf{Tester avec les Utilisateurs Finaux};
    \item \textbf{Accompagner les Projets};
    \item \textbf{Faire Rayonner};
\end{enumerate}
Pour la robotique, par exemple, il y a des projets proporser pour différents entreprises comme \href{https://www.shark-robotics.com/}{Shark Robotics}.


\newpage\subsection{Innovation à l'Afrique}
\subsubsection{Environnement Économique}
\paragraph{Définition}Dans l'Afrique centrale il y a plus de development des entreprises car il y a une mieux structure pour la création des entreprises. La diversité dans le continant explique les différences de development de chaque pays. Présenter par Marfini Dossoz.\\

\noindent Dans l'Agriculture empregue 60\% de la main d'óeuvre dans les 25\% des terres fertiles qui contribue pour 33\% du PIB du continent.

\subsubsection{Startups Digitales}
\paragraph{Défintion}Comme dans d'autres pays il y a des différents secteurs où chaque start-up develope d'entre autres:
\begin{enumerate}[noitemsep]
    \item agro-tech: important pour le futur;
    \item clean-tech;
    \item e-tech;
    \item fin-tech: plus grand en ce moment, croissance exponentiel;
\end{enumerate}
\noindent On considere aussi le Savanna Valley comme un endroit avec plusieurs start-ups. En ce moment les donnes des ces entreprises ne sont pas très clairs pour les investisseurs à cause des différences de chaque pays.\\

\noindent La répartition du financement fait écho à la croissance des start-ups dans les "quatre grands", le Nigeria, l'Afrique du Sud, l'Égypte et....\\

\noindent Avec le development des start-ups le marche associe à grossie avec les  accélérateurs, les incubateurs, les centres d'innovation, les créateurs de capital-risque, les réseaux et d'autres.\\

\subsubsection{Projets d'Innovation}
\paragraph{Définition}Il y a beaucoup des problèmes de base dans le pays qui sont adresse pour des start-ups. Le development des technologies open-sources sont très importants pour le continant.

\subsubsection{Sources et Barrières des Start-Ups}
\paragraph{Définition}Chaque entreprise a des différents problèmes pour son development pour cette raison le Ci20 à fait un enquête pour découvrir des informations:
\begin{enumerate}[noitemsep]
    \item Renseignent;
    \item Profil d'Innovation;
    \item Impacts sur la Productivité;
\end{enumerate}
\noindent Entreprises:
\begin{enumerate}[noitemsep]
    \item Anelys;
    \item colib;
    \item cinetpay;
    \item digitech group;
    \item etudesk;
    \item Grain Theque;
    \item innoving;
    \item jool;
    \item mon artisan;
    \item pass+mousso;
    \item skaned;
\end{enumerate}
\noindent Les innovations de chaque entreprise doit considere le marche. Il y a des innovations pour l'entreprise, pour le marche locaux, pour le marche African et pour le marche mondiale. C'est essential de considérer le contexte pour comprendre l'impacte de chaque produit.\\
\end{document}