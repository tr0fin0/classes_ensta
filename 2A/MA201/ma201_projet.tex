\documentclass{article}
\usepackage{C:/Users/Admin-PC/Documents/git_repository/tpack/tpack}
% \usepackage{tpack}
\title{MA201 - Statistique}
\project{Projet Filtre Kalman Etendu}
\author{Guilherme Nunes Trofino et Vanessa López Noreña}
\authorRA{2022-2024}


\makeatletter
\begin{document}\selectlanguage{french}
\maketitle

\newpage\tableofcontents

\section*{Introduction}
\subsection*{Problème}
\paragraph{Présentation}Un dispositif de poursuite, dont la position est fixe et connue, $(x_{0}, \; y_{0})$ l'origine du système, mesure à des instants discrets $t = kT$ où $T$ est le période de mesure connue également.\\

\noindent On considère qu'on aura les mesures suivants:
\begin{enumerate}
    \item $D_{k}$, \textbf{Distance} jusqu'à la cible;
    \item $\alpha_{k}$, \textbf{Angle} entre l'horizontale et la cible;
\end{enumerate}
On considère que ces mesures sont entachées d'erreurs additives modélisées par des bruits blancs centrés suivants:
\begin{enumerate}
    \item $n_{D}(k)$, Bruit de la Distance:
    \begin{enumerate}[noitemsep]
        \item \texttt{Espérance}: $\mathbf{E}[n_{D}(k)] = 0$;
        \item \texttt{Variance}: $\text{Var}(n_{D}(k)) = \sigma_{D}^{2}(k)$;
    \end{enumerate}
    \item $n_{\alpha}(k)$, Bruit de l'Angle:
        \begin{enumerate}[noitemsep]
        \item \texttt{Espérance}: $\mathbf{E}[n_{\alpha}(k)] = 0$;
        \item \texttt{Variance}: $\text{Var}(n_{\alpha}(k)) = \sigma_{\alpha}^{2}(k)$;
    \end{enumerate}
\end{enumerate}
\paragraph{Supposition}Bruits sont indépendants.

\subsection*{Espace Physique}
\paragraph{Définition}On considère que à chaque instant la position de la cible sera donne pour $(x(t);y(t))$.

\begin{figure}[H]
    \centering
    \begin{tikzpicture}
      \begin{axis}[
    		xtick distance=1, ytick distance=1,
    		xmin=-8, xmax=8,
    		ymin=-8, ymax=8,
    		axis lines = center,
    		ticks=none]

        \draw [red!60, -{Latex[round]}] (1.5,0) arc [
            start angle = 0,
            end angle   = 45,
            radius      = 1.5
        ] 
        % node [near start, bottom] {$\alpha_{k}^{\circ}$}
        node [near start] {$\alpha_{k}^{\circ}$}
        ;
        \draw (1.25, 2.75) node []  {$D_{k}$};

        \draw[-] (axis cs:{0,0}) -- (axis cs:{6,6}) node [above] {$(x(t);y(t))$};

        \draw[gray, dashed] (axis cs:{6,0}) -- (axis cs:{6,6});
        \draw[gray, dashed] (axis cs:{0,6}) -- (axis cs:{6,6});
      \end{axis}
    \end{tikzpicture}
    \caption{Représentation Système}
\end{figure}
\noindent On considère que le radar c'est positionne à l'origine du système noté comme: $(x_{0}, y_{0}) = (0, 0)$.



\section{Exercice 1}
\paragraph{Définition}On suppose que que la fréquence d'échantillonnage du radar est suffisamment faible pour qu'entre deux instants de mesures, $kT$ et $(k+1)T$, l'accélération de la cible soit constante et donner pour le suivant: $a(k) = (a_{x}(k), a_{y}(k))$. Pour l'accélération $a_{y}(k)$ il faut considère la gravité, égal à $g = 9.81$ orientée suivant $y$ négativement.
\subsection{Part 1}
\paragraph{Résolution}On considère de l'énonce:
\begin{equation}
    r_{x}(l) = \mathbf{E}[a_{x}(k) \cdot a_{x}(k+l)] = \sigma_{a}^{2} \cdot \exp{(-\mu l)} \implies
    r_{x}(0) = \boxed{\mathbf{E}[a_{x}(k)^{2}] = \sigma_{a}^{2}}
\end{equation}
\noindent De cette façon-là on aura l'équation suivante:
\begin{align*}
    (a_{x}(k+1))^2 &= (\beta \cdot a_{x}(k) + w_{x}(k))^2\\
    \mathbf{E}[(a_{x}(k+1))^2] &= \mathbf{E}[(\beta \cdot a_{x}(k) + w_{x}(k))^2]\\\\
    \sigma_{a}^{2} &= \mathbf{E}[\beta^2 a_{x}(k)^{2} + 2 \beta a_{x}(k) \cdot w_{x}(k) + w_{x}(k)^2]\\
    &= \beta^2 \mathbf{E}[a_{x}(k)^{2}] + 2 \beta \mathbf{E}[a_{x}(k) \cdot w_{x}(k)] + \mathbf{E}[w_{x}(k)^2]\\
    &= \beta^2 \mathbf{E}[a_{x}(k)^{2}] + 2 \beta \mathbf{E}[a_{x}(k)] \cdot \mathbf{E}[w_{x}(k)] + \mathbf{E}[w_{x}(k)^2]
\end{align*}
\noindent On considère que: $\mathbf{E}[a_{x}(k) \cdot w_{x}(k)] = \mathbf{E}[a_{x}(k)] \cdot \mathbf{E}[w_{x}(k)]$ parce que $a_{x}(k)$ est indépendant de $w_{x}(k)$
\begin{align*}
    \sigma_{a}^{2} &= \beta^2 \mathbf{E}[a_{x}(k)^{2}] + 2 \beta \mathbf{E}[a_{x}(k)] \cdot \cancelto{0}{\mathbf{E}[w_{x}(k)]} + \mathbf{E}[w_{x}(k)^2]\\
     &= \beta^2 \mathbf{E}[a_{x}(k)^{2}] + \mathbf{E}[w_{x}(k)^2]
\end{align*}
\noindent On note que $w_{x}(k)$ est considère comme un bruit centrée sa Espérance sera nulle et donc: $\mathbf{E}[w_{x}(k)] = 0$. On sait aussi que:
\begin{align*}
     Var(w_{x}(k)) &= \mathbf{E}[w_{x}(k)^2] - (\mathbf{E}[w_{x}(k)])^2\\
     &= \mathbf{E}[w_{x}(k)^2] - \cancelto{0}{(\mathbf{E}[w_{x}(k)])^2}\\
     \Aboxed{\sigma_{w}(k)^2 &= \mathbf{E}[w_{x}(k)^2]}
\end{align*}
\noindent Ainsi, on a:
\begin{equation*}
    \boxed{\sigma_{w}(k)^2 = \sigma_{a}(k)^2 \cdot ( 1 - \beta^2 )}
\end{equation*}
Après:
\begin{align*}
    r_{x}(1) &= \mathbf{E}[a_{x}(k) \cdot a_{x}(k+1)]\\
    &= \mathbf{E}[a_{x}(k) \cdot (\beta \cdot a_{x}(k) + w_{x}(k))]\\
    &= \mathbf{E}[\beta a_{x}(k)^{2} + a_{x}(k) \cdot w_{x}(k)]\\
    \sigma_{a}^{2} \cdot \exp{(-\mu)} &= \mathbf{E}[\beta a_{x}(k)^{2}] + \cancelto{0}{\mathbf{E}[a_{x}(k) \cdot w_{x}(k)]}\\
    &= \beta\mathbf{E}[a_{x}(k)^{2}]\\
    &= \beta\sigma_{a}(k)^2\\
    \Aboxed{\exp{(-\mu)} &= \beta}
\end{align*}
Finalement on a le même résolution pour $x$ et pour $y$ donc ça démontrer que:
\begin{align}
    \Aboxed{\beta &= \exp{(-\mu)}}\\
    \Aboxed{\sigma_{w}(k)^2 &= \sigma_{a}(k)^2 \cdot ( 1 - \beta^2 )}
\end{align}

\subsection{Part 2}
La dynamique de l'état est obtenue avec les équations du mouvement:

\begin{align*}
    &x(k+1) = x(k) + \dot{x}(k) + \frac{1}{2} \ddot{x}(k)\cdot T^2 \\
    &y(k+1) = y(k) + \dot{y}(k) + \frac{1}{2}  \ddot{y}(k)\cdot T^2\\
    \\
    &\dot{x}(k+1) = \dot{x}(k) + \ddot{x}\cdot T\\
    &\dot{y}(k+1) = \dot{y}(k) + \ddot{y}\cdot T\\
    \\
    % &L'exercice \; précédant \; propose:
    \\
    &\ddot{x}(k+1) = \beta \cdot \ddot{x}(k) + W_x(k)\\
    &\ddot{y}(k+1) = \beta \cdot \ddot{y}(k) + W_y(k)\\
\end{align*}

Avec ces équations, nous pouvons expliciter les matrices:


\subsection{Part 3}
\subsection{Part 4}


\section{Exercice 2}
\subsection{Part 1}
\subsection{Part 2}
\subsection{Part 3}


\section{Exercice 3}
\subsection{Part 1}


\section{Exercice 4}
\subsection{Part 1}


\section{Exercice 5}
\subsection{Part 1}


\section{Exercice 6}
\subsection{Part 1}


\section{Exercice 7}



\section{Algorithmes}
\paragraph{Présentation}

\begin{scriptsize}\mycode
    \begin{lstlisting}[language=Bash]
    cd
    \end{lstlisting}
\end{scriptsize}
\end{document}