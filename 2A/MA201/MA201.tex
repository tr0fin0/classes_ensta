\documentclass{article}
\usepackage{C:/Users/Admin-PC/Documents/git_repository/tpack/tpack}
% \usepackage{/home/tr0fin0/git_repositories/tpack/tpack}


\title{MA201 - Estimation Identification Statistique}
\project{Résumé Théorique}
\author{Guilherme Nunes Trofino}
\authorRA{217276}


\makeatletter
\begin{document}\selectlanguage{french}
\maketitle

\newpage\tableofcontents

\section{Introduction}
\subfile{C:/Users/Admin-PC/Documents/git_repository/classes_ensta/intro.tex}
% \subfile{/home/tr0fin0/git_repositories/classes_ensta/intro.tex}


\subsection{Informations Matière}
\paragraph{Présentation}Ce cours sera présenter par M. Luc Meyer qui a pour but d'étudier Statistique .


\section{Statistique}
\subsection{Variables Quantitatives}
\paragraph{Définition}

\subsubsection{Moyenne}
\paragraph{Définition}
\begin{equation}
    \boxed{
        \bar{X} = \frac{1}{n} \sum^{n}_{i=1} X_{i}
    }
\end{equation}

\subsubsection{Variance}
\paragraph{Définition}
\begin{equation}
    \boxed{
        V = \frac{1}{n} \sum^{n}_{i=1} (X_{i} - \bar{X})^{2}
    }
\end{equation}

\subsubsection{Covariance}
\paragraph{Définition}
\begin{equation}
    \boxed{
        void
    }
\end{equation}

\subsubsection{Corrélation}
\paragraph{Définition}
\begin{equation}
    \boxed{
        void
    }
\end{equation}


\subsection{Modèle Statistique}
\paragraph{Définition} \href{https://en.wikipedia.org/wiki/Statistical_model}{Statistical Model}
\begin{equation}
    \boxed{
        \mathcal{M} = (\mathcal{X}^{n}, \mathcal{A}^{n}, \mathcal{P}^{n}_{\theta}, \theta\in\Theta)
    }
\end{equation}

\subsubsection{Éspace}
\paragraph{Définition}
\begin{equation}
    \boxed{
        void
    }
\end{equation}

\subsubsection{Tribu}
\paragraph{Définition} \href{https://fr.wikipedia.org/wiki/Tribu_(math%C3%A9matiques)}{Tribu}
\begin{equation}
    \boxed{
        void
    }
\end{equation}

\subsubsection{Loi de Probabilité}
\paragraph{Définition} \href{https://fr.wikipedia.org/wiki/Loi_de_probabilit%C3%A9}{Loi de Probabilité}
\begin{equation}
    \boxed{
        void
    }
\end{equation}

\subsubsection{n-Échantillon i.i.d}
\paragraph{Définition}
\begin{equation}
    \boxed{
        void
    }
\end{equation}

\subsection{Modélisation Analyse}
\subsubsection{Biais}
\paragraph{Définition}
\begin{equation}
    \boxed{
        B_{\theta}(\bar{X}_{n}, X) = \mathbb{E}[\bar{X}_{n}] - X
    }
\end{equation}
Quand $B_{\theta}(\bar{X}_{n}, X) = 0$ on considere que $\bar{X}_{n}$ est non biaisé.

\subsubsection{Variance}
\paragraph{Définition}
\begin{equation}
    \boxed{
        var(\bar{X}_{n}) = \mathbb{E}_{\theta}[(\bar{X}_{n} - \mathbb{E}_{\theta}[\bar{X}_{n}])^{2}]
    }
\end{equation}

\subsubsection{Risque Quadratique}
\paragraph{Définition}

\subsection{Estimateur}
\paragraph{Définition}
\begin{equation}
    \boxed{
        void
    }
\end{equation}

\subsubsection{Convergence}
\paragraph{Définition}

\subsubsection{Moyenne Empirique}
\paragraph{Définition}
\begin{equation}
    \boxed{
        \bar{X}_{n} = \frac{1}{n} \sum^{n}_{i=1} X_{i}
    }
\end{equation}

\subsubsection{Variance Empirique}
\paragraph{Définition}
\begin{equation}
    \boxed{
        \bar{X}_{n} = \frac{1}{n} \sum^{n}_{i=1} (X_{i} - \bar{X})^{2}
    }
\end{equation}

\subsubsection{Méthode des Moments}
\paragraph{Définition}
\begin{equation}
    \boxed{
        void
    }
\end{equation}

\subsubsection{Loi Uniforme}
\paragraph{Définition} \href{https://fr.wikipedia.org/wiki/Loi_uniforme_continue}{Loi Uniforme Continue}
\begin{equation}
    \boxed{
        f(x) = 
        \begin{cases}
            \frac{1}{b-a},  & \text{pour } a \leq x \leq b\\
            0               & \text{sinon}    
        \end{cases}
    }
\end{equation}


\section{Estimateur Bayésienne}
\paragraph{Définition}



% \subsubsection{}
% \paragraph{Définition}
% \begin{equation}
%     \boxed{

%     }
% \end{equation}



\newpage
population
individuel
échantillon
n-échantillon
observation

défition d'un estimateur
estimation par intervel de confiance pas dans ce cours

consistance d'un estimateur
bias et variance
estimateurs sans biais à minimum de variance
vraisemblance

cramer-rao
aider à construir des estimateur s non biaises
covariance
risque quadratique moyennes
estimateur à RQM minimal
estimateurs empiriques
variance empirique
estimateur de moment
moment théorique
moment empirique
moment centre
extention
Etheta juste pour dénoter que l'esperance depende de la variable theta
Estimateur du maximum de vraisemblance
faire la déduction des exemples
support d'un function
explore les avantages et les limites de ce méthode

% https://fr.m.wikipedia.org/wiki/Covariance

\section{Travail Dirigé}
\subsection{05/09/2022}
\subsection{19/09/2022}
\end{document}