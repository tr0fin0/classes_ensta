\documentclass{article}
\usepackage{C:/Users/Admin-PC/Documents/git_repository/tpack/tpack}


\title{MA201 - Estimation Identification Statistique}
\author{Guilherme Nunes Trofino}
\authorRA{217276}
\project{Résumé Théorique}


\begin{document}\selectlanguage{french}
\maketitle

\newpage\tableofcontents

\section{Introduction}
\subsection{}
population
individuel
échantillon
n-échantillon
observation

défition d'un estimateur
estimation par intervel de confiance pas dans ce cours

consistance d'un estimateur
bias et variance
estimateurs sans biais à minimum de variance
vraisemblance

cramer-rao
aider à construir des estimateur s non biaises
covariance
risque quadratique moyennes
estimateur à RQM minimal
estimateurs empiriques
variance empirique
estimateur de moment
moment théorique
moment empirique
moment centre
extention
E_theta juste pour dénoter que l'esperance depende de la variable theta
Estimateur du maximum de vraisemblance
faire la déduction des exemples
support d'un function
explore les avantages et les limites de ce méthode

% https://fr.m.wikipedia.org/wiki/Covariance
\end{document}