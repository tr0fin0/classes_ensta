\documentclass{article}
\usepackage{C:/Users/Admin-PC/Documents/git_repository/tpack/tpack}
% \usepackage{tpack}
\title{MA201 - Statistique}
\project{Projet Filtre Kalman Etendu}
\author{Guilherme Nunes Trofino et Vanessa López Noreña}
\authorRA{2022-2024}


\makeatletter
\begin{document}\selectlanguage{french}
\maketitle

\newpage\tableofcontents

\section*{Introduction}
\subsection*{Problème}
\paragraph{Présentation}Un dispositif de poursuite, dont la position est fixe et connue, $(x_{0}, \; y_{0})$ l'origine du système, mesure à des instants discrets $t = kT$ où $T$ est le période de mesure connue également.\\

\noindent On considère qu'on aura les mesures suivants:
\begin{enumerate}[noitemsep]
    \item $D_{k}$, \textbf{Distance} jusqu'à la cible;
    \item $\alpha_{k}$, \textbf{Angle} entre l'horizontale et la cible;
\end{enumerate}
On considère que ces mesures sont entachées d'erreurs additives modélisées par des bruits blancs centrés \cite{bruitBlanc} suivants:
\begin{enumerate}
    \item $n_{D}(k)$, Bruit de la Distance:
    \begin{enumerate}[noitemsep]
        \item \texttt{Espérance}: $\mathbf{E}[n_{D}(k)] = 0$;
        \item \texttt{Variance}: $\text{Var}(n_{D}(k)) = \sigma_{D}^{2}(k)$;
    \end{enumerate}
    \item $n_{\alpha}(k)$, Bruit de l'Angle:
        \begin{enumerate}[noitemsep]
        \item \texttt{Espérance}: $\mathbf{E}[n_{\alpha}(k)] = 0$;
        \item \texttt{Variance}: $\text{Var}(n_{\alpha}(k)) = \sigma_{\alpha}^{2}(k)$;
    \end{enumerate}
\end{enumerate}
On note que les bruits sont considères centrés et donc l'Esperance d'eux sera nulle.
\paragraph{Supposition}Bruits sont indépendants, ça-t-a dit qu'il n'y a pas d'interférence entre les bruits.

\subsection*{Espace Physique}
\paragraph{Définition}On considère que à chaque instant la position de la cible sera donne pour $(x(t), \; y(t))$, represente en bleu.

\begin{figure}[H]
    \centering
    \begin{tikzpicture}
      \begin{axis}[
    		xtick distance=1, ytick distance=1,
    		xmin=-8, xmax=8,
    		ymin=-8, ymax=8,
    		axis lines = center,
    		ticks=none]

        \draw [red!60, -{Latex[round]}] (1.5,0) arc [
            start angle = 0,
            end angle   = 45,
            radius      = 1.5
        ] 
        % node [near start, bottom] {$\alpha_{k}^{\circ}$}
        node [near start] {$\alpha_{k}^{\circ}$};
        
        \draw[-] (axis cs:{0,0}) -- (axis cs:{6,6}) node [blue, above] {$(x(t), \; y(t))$};
        
        \draw[gray, dashed] (axis cs:{6,6}) -- (axis cs:{0,6}) node [left]  {$D_{k_y}$};
        \draw[gray, dashed] (axis cs:{6,6}) -- (axis cs:{6,0}) node [below] {$D_{k_x}$};

        \draw (1.25, 2.75) node []  {$D_{k}$};
        \draw[violet] (-1.75, -1) node []  {$(x_{0}, \; y_{0})$};

        \draw[violet, thin] (0, 0) circle (3pt);
        \draw[blue, thin] (6, 6) circle (3pt);
      \end{axis}
    \end{tikzpicture}
    \caption{Représentation Système}\label{fig:systeme}
\end{figure}
\noindent On considère que le radar c'est positionne à l'origine du système noté comme: $(x_{0}, y_{0}) = (0, 0)$, represente en violet.



\section{Exercice 1}
\paragraph{Définition}On suppose que que la fréquence d'échantillonnage du radar est suffisamment faible pour qu'entre deux instants de mesures, $kT$ et $(k+1)T$, l'accélération de la cible soit constante et donner pour le suivant: $a(k) = (a_{x}(k), a_{y}(k))$. Pour l'accélération $a_{y}(k)$ il faut considère la gravité, égal à $g = 9.81$ orientée suivant $y$ négativement.
\subsection{Part 1}
\paragraph{Proposition}Pour la modélisation on considère qu'une cible qui accélère à l'instant $kT$ ira, probablement, accélère à l'instant $(k+1)T$. Ainsi on introduit la matrice de corrélation \cite{jacobianCovariance} suivante entre les composantes succesives des accélérations:
\begin{equation}
    r_{x} (l) = \mathbf{E}[a_{x}(k) \cdot a_{x}(k+l)] = \sigma_{a}^{2} \exp{(-\mu l)}
    \qquad
    r_{y} (l) = \mathbf{E}[a_{y}(k) \cdot a_{y}(k+l)] = \sigma_{a}^{2} \exp{(-\mu l)}
\end{equation}
\noindent Où:
\begin{enumerate}[noitemsep]
    \item $\sigma_{a}^{2}$, \textbf{Intensité de l'Accélération};
    \item $\mu$, \textbf{Rapidité de la Manoeuvre};
\end{enumerate}
\paragraph{Résolution}On considère de l'énonce:
\begin{equation}
    r_{x}(0) \implies \boxed{\mathbf{E}[a_{x}(k)^{2}] = \sigma_{a}^{2}}
\end{equation}
\noindent De cette façon-là, en considerant l'équation propose par l'exercice $a_{x}(k+1) = \beta \cdot a_{x}(k) + w_{x}(k)$, on aura l'équation suivante:
\begin{align*}
    (a_{x}(k+1))^2 &= (\beta \cdot a_{x}(k) + w_{x}(k))^2\\
    \mathbf{E}[(a_{x}(k+1))^2] &= \mathbf{E}[(\beta \cdot a_{x}(k) + w_{x}(k))^2]\\\\
    \sigma_{a}^{2} &= \mathbf{E}[\beta^2 a_{x}(k)^{2} + 2 \beta a_{x}(k) \cdot w_{x}(k) + w_{x}(k)^2]\\
    &= \beta^2 \mathbf{E}[a_{x}(k)^{2}] + 2 \beta \mathbf{E}[a_{x}(k) \cdot w_{x}(k)] + \mathbf{E}[w_{x}(k)^2]\\
    &= \beta^2 \mathbf{E}[a_{x}(k)^{2}] + 2 \beta \mathbf{E}[a_{x}(k)] \cdot \mathbf{E}[w_{x}(k)] + \mathbf{E}[w_{x}(k)^2]
\end{align*}
\noindent On considère que: $\mathbf{E}[a_{x}(k) \cdot w_{x}(k)] = \mathbf{E}[a_{x}(k)] \cdot \mathbf{E}[w_{x}(k)]$ parce que $a_{x}(k)$ est indépendant de $w_{x}(k)$, consideration du exercice. Après on aura:
\begin{align*}
    \sigma_{a}^{2} &= \beta^2 \mathbf{E}[a_{x}(k)^{2}] + 2 \beta \mathbf{E}[a_{x}(k)] \cdot \cancelto{0}{\mathbf{E}[w_{x}(k)]} + \mathbf{E}[w_{x}(k)^2]\\
     &= \beta^2 \mathbf{E}[a_{x}(k)^{2}] + \mathbf{E}[w_{x}(k)^2]
\end{align*}
\noindent On note que $w_{x}(k)$ est considère comme un bruit stationnaire d'ordre 2 centrée donc sa Espérance sera nulle: $\mathbf{E}[w_{x}(k)] = 0$. On sait aussi que:
\begin{align*}
     Var(w_{x}(k)) &= \mathbf{E}[w_{x}(k)^2] - (\mathbf{E}[w_{x}(k)])^2 = \sigma_{w}^2(k)\\
     &= \mathbf{E}[w_{x}(k)^2] - \cancelto{0}{(\mathbf{E}[w_{x}(k)])^2}\\
     \Aboxed{\sigma_{w}^2(k) &= \mathbf{E}[w_{x}(k)^2]}
\end{align*}
\noindent Ainsi, on a:
\begin{equation*}
    \boxed{\sigma_{w}(k)^2 = \sigma_{a}(k)^2 \cdot ( 1 - \beta^2 )}
\end{equation*}
Après:
\begin{align*}
    r_{x}(1) &= \mathbf{E}[a_{x}(k) \cdot a_{x}(k+1)]\\
    &= \mathbf{E}[a_{x}(k) \cdot (\beta \cdot a_{x}(k) + w_{x}(k))]\\
    &= \mathbf{E}[\beta \cdot a_{x}(k)^{2} + a_{x}(k) \cdot w_{x}(k)]\\
    \sigma_{a}^{2}(k) \cdot \exp{(-\mu)} &= \mathbf{E}[\beta a_{x}(k)^{2}] + \cancelto{0}{\mathbf{E}[a_{x}(k) \cdot w_{x}(k)]}\\
    &= \beta \cdot \mathbf{E}[a_{x}(k)^{2}]\\
    &= \beta \cdot \sigma_{a}(k)^2\\
    \Aboxed{\exp{(-\mu)} &= \beta}
\end{align*}
Comment les équations sont les mêmes pour $x$ et pour $y$ sont les mêmes on peut appliquer les résultats suivantes aux deux coordonnées:
\begin{align}
    \Aboxed{\beta &= \exp{(-\mu)}}\\
    \Aboxed{\sigma_{w}(k)^2 &= \sigma_{a}(k)^2 \cdot ( 1 - \beta^2 )}
\end{align}
De cette façon là on peut representer les accélérations peuvent être donne par les équations suivantes:
\begin{equation}
    \boxed{a_{x}(k+1) = \beta \cdot a_{x}(k) + w_{x}(k)}
    \qquad
    \boxed{a_{y}(k+1) = \beta \cdot a_{y}(k) + w_{y}(k)}
\end{equation}

\subsection{Part 2}
\paragraph{Proposition}Écrire les lois d'évolution de $x(t)$, $y(t)$, et $\dot{x}(t)$, $\dot{y}(t)$ sous forme matricielle en explicitant les matrices de l'équation de la dynamique de l'état sous la forme:
\begin{equation}
    \mathbf{X}(k+1) = \mathbf{F}_{k} \times \mathbf{X}(k) + \mathbf{G}_{k} \times \mathbf{u}(k)
    \quad
    \text{où}
    \quad
    \mathbf{X}(k) = [x(k)\;y(k)\;\dot{x}(k)\;\dot{y}(k)\;\ddot{x}(k)\;\ddot{y}(k)]^{\text{T}}
\end{equation}
D'abbord on considere les équations de mouvement pour un corps dans le temps sont:
\begin{equation*}
    x(t) = x(0) + \dot{x}(t) \cdot t + \ddot{x}(t) \cdot \frac{1}{2}t^2
    \qquad
    \dot{x}(t) = \dot{x}(0) + \ddot{x}(t) \cdot t
\end{equation*}
\paragraph{Résolution}Avant c'est considère le cas continue. Pour le cas discrete la dynamique de l'état est obtenue avec les équations suivant:
\begin{center}
    \begin{minipage}{0.45\textwidth}
        \begin{align*}
            x(k+1)        &= x(k) + \dot{x}(k) \cdot T + \frac{1}{2}T^2 \cdot \ddot{x}(k)\\
            \dot{x}(k+1)  &= \dot{x}(k) + \ddot{x}(k) \cdot T\\
            \ddot{x}(k+1) &= \beta \cdot \ddot{x}(k) + w_{x}(k)
        \end{align*}
    \end{minipage}
    \begin{minipage}{0.45\textwidth}
        \begin{align*}
            y(k+1)        &= y(k) + \dot{y}(k) \cdot T + \frac{1}{2}T^2 \cdot \ddot{y}(k)\\
            \dot{y}(k+1)  &= \dot{y}(k) + \ddot{y}(k) \cdot T\\
            \ddot{y}(k+1) &= \beta \cdot \ddot{y}(k) + w_{y}(k)
        \end{align*}
    \end{minipage}
\end{center}
Avec ces équations, on peut expliciter les matrices:
\begin{equation}
    \underbrace{\begin{bmatrix}
        x(k+1)\\
        y(k+1)\\
        \dot{x}(k+1)\\
        \dot{y}(k+1)\\
        \ddot{x}(k+1)\\
        \ddot{y}(k+1)\\
    \end{bmatrix}}_{\mathbf{X}(k+1)}
    = 
    \underbrace{\begin{bmatrix}
        1 & 0 & T & 0 & \frac{1}{2}T^2 & 0\\
        0 & 1 & 0 & T & 0              & \frac{1}{2}T^2\\
        0 & 0 & 1 & 0 & T              & 0 \\
        0 & 0 & 0 & 1 & 0              & T \\
        0 & 0 & 0 & 0 & \beta          & 0 \\
        0 & 0 & 0 & 0 & 0              & \beta \\
    \end{bmatrix}}_{\mathbf{F}_{k}}
    \times
    \underbrace{\begin{bmatrix}
        x(k)\\
        y(k)\\
        \dot{x}(k)\\
        \dot{y}(k)\\
        \ddot{x}(k)\\
        \ddot{y}(k)\\
    \end{bmatrix}}_{\mathbf{X}(k)}
    + 
    \underbrace{\begin{bmatrix}
        0 & 0 \\
        0 & 0 \\
        0 & 0 \\
        0 & 0 \\
        1 & 0 \\
        0 & 1 \\
    \end{bmatrix}}_{\mathbf{G}_{k}}
    \times
    \underbrace{\begin{bmatrix}
        w_{x}(k)\\
        w_{y}(k)
    \end{bmatrix}}_{\mathbf{u}(k)}
\end{equation}
On note que le système d'équations peut être represente par l'équation matricielle suivante:
\begin{equation}
    \boxed{\mathbf{X}(k+1) = \mathbf{F}_{k} \times \mathbf{X}(k) + \mathbf{G}_{k} \times \mathbf{u}(k)}
\end{equation}
Pour considérer la gravité on va la ajouter sur les conditions initialles du problême.


\subsection{Part 3}
\paragraph{Proposition}Exprimer $D_k$ et $\alpha_k$ en fonction des positions de la cible et du dispositif d'observation.

\paragraph{Résolution}On considere la representation du système \ref{fig:systeme} et on déduire les équations suivantes avec la trigonométrie:
\begin{equation}
    \boxed{D_{k} = \sqrt[2]{(x(t) - x_{0})^2 + (y(t) - y_{0})^2}}
    \qquad
    \boxed{\alpha_{k} = \arctan\left( \frac{(y(t) - y_{0})}{(x(t) - x_{0})} \right)}
\end{equation}
On note que ces équations ne considerent pas les bruits et incertitudes des mesures mas fournir une information cartesienne pour comprendre le movement de la cible.

\subsection{Part 4}
\paragraph{Proposition}On considère ici une transformation des mesures permettant de les transformer sous une forme plus de l'état que l'on cherche à estimer. De cette façon on considère les pseudo mesures suivantes:
\begin{align*}
    \tilde{x}(k)^{(b)} &= D_{k}^{(b)} \cdot \cos(\alpha_{k}^{(b)})\\
    \tilde{y}(k)^{(b)} &= D_{k}^{(b)} \cdot \sin(\alpha_{k}^{(b)})
\end{align*}
Où $D_{k}^{(b)}$ et $\alpha_{k}^{(b)}$ sont considères, par hypothèse, modèleses avec un bruit gaussien.

\paragraph{Résolution}D'abbord on considère que le bruit sera modélise pour les equations suivantes:
\begin{align*}
    D_{k}^{(b)} &= D_{k} + w_{D_{k}}(k)\\
    \alpha_{k}^{(b)} &= \alpha_{k} + w_{\alpha_{k}(k)}
\end{align*}
Où $w_{D_{k}}(k)$ et $w_{\alpha_{k}}(k)$ sont les representations du bruit gaussien. Ainsi on aura:
\begin{align*}
    \tilde{x}(k)^{(b)} &= (D_{k} + w_{D_{k}}(k)) \cdot \cos(\alpha_{k} + w_{\alpha_{k}(k)})\\
    &= D_{k} \cdot \cos(\alpha_{k} + w_{\alpha_{k}(k)}) + w_{D_{k}}(k) \cdot \cos(\alpha_{k} + w_{\alpha_{k}(k)})\\
    &= D_{k} \cdot \cos(\alpha_{k})\cos(w_{\alpha_{k}(k)}) - D_{k} \cdot \sin(\alpha_{k})\sin(w_{\alpha_{k}(k)}) + w_{D_{k}}(k) \cdot \cos(\alpha_{k} + w_{\alpha_{k}(k)})\\
    &= \boxed{\tilde{x}(k) \cdot \cos(w_{\alpha_{k}(k)})} - \boxed{\tilde{x}(k)\sin(w_{\alpha_{k}(k)})} + w_{D_{k}}(k) \cdot \cos(\alpha_{k} + w_{\alpha_{k}(k)})
\end{align*}
On rappelle que:
\begin{align*}
    \cos(a + b) &= \cos(a)\cos(b) - \sin(a)\sin(b)\\
    \sin(a + b) &= \sin(a)\cos(b) + \sin(b)\cos(a)
\end{align*}
Ainsi on a considere que:
\begin{align*}
    \tilde{x}(k) &= D_{k} \cdot \cos(\alpha_{k})\\
    \tilde{y}(k) &= D_{k} \cdot \sin(\alpha_{k})
\end{align*}
Représentent les mesures sans bruits associes à $x$ et $y$. On note que après les substitutions d'avant on aura $\cos(w_{\alpha_{k}}(k))$ qui represente une opération pas possible, le cossenus d'une fonctionne gaussienne. Ça démontre que la consideration de bruit gaussien c'est pas possible.


\section{Exercice 2}
\subsection{Part 1}
\paragraph{Proposition}Écrire les équations du filtre de Kalman simple correspondant aux équations dynamiques de l'équation d'observation correspondant aux pseudomesures.

\paragraph{Résolution}Un filtre de Kalman Simple, \cite{kalmanFilterWiki}, \cite{kalmanFilterMATLAB} et \cite{kalmanFilterMonashUniversity}, est mis en œuvre avec les équations:
\begin{equation*}
    \begin{cases}
        \mathbf{X}_{k} &= \mathbf{F}_{k} \times \mathbf{X}_{k-1} + \mathbf{G}_{k} \times \mathbf{u}_{k} + \mathbf{w}_{k}\\
        \mathbf{Z}_{k} &= \mathbf{H}_{k} \times \mathbf{X}_{k} + \mathbf{v}_{k}
    \end{cases}
\end{equation*}
Où:
\begin{enumerate}
    \item \textbf{Équation d'Évolution du Système}:
    \begin{enumerate}[noitemsep]
        \item $\mathbf{X}_{k}$, \texttt{Vecteur de Mesures} à l'instant $k$;
        \item $\mathbf{F}_{k}$, \texttt{Matrice d'Évolution} ou Matrice de Transition;
        \item $\mathbf{G}_{k}$, \texttt{Matrice de Controle};
        \item $\mathbf{u}_{k}$, \texttt{Vecteur de Entrée};
        \item $\mathbf{w}_{k}$, \texttt{Vecteur de Bruit d'Évolution}: supposé gaussien, centré et de matrice de covariance $\mathbf{Q}_{k}$;
    \end{enumerate}
    \item \textbf{Équation d'Observation du Système}:
    \begin{enumerate}[noitemsep]
        \item $\mathbf{Z}_{k}$, \texttt{Vecteur de Messures};
        \item $\mathbf{H}_{k}$, \texttt{Matrice d'Observations};
        \item $\mathbf{v}_{k}$, \texttt{Vecteur de Bruit de Mesures}: supposé gaussien, centré et de matrice de covariance $\mathbf{R}_{k}$;
    \end{enumerate}
\end{enumerate}
Où chaque matrice qui a une souscrit $k$ represente une matrice qui varie pour chaque $k$. Dans le cas de ce exercice on ira considérer les matrices suivantes pour décrire le système:
\begin{equation}
    \mathbf{X}_{k} = 
    \begin{bmatrix}
        x(k)\\
        y(k)\\
        \dot{x}(k)\\
        \dot{y}(k)\\
        \ddot{x}(k)\\
        \ddot{y}(k)\\
    \end{bmatrix}
    \quad
    \mathbf{u}_{k}
    \begin{bmatrix}
        w_{x}(k)\\
        w_{y}(k)\\
    \end{bmatrix}
    \quad
    \mathbf{F}_{k} = 
    \begin{bmatrix}
        1 & 0 & T & 0 & \frac{T^2}{2}& 0\\
        0 & 1 & 0 & T & 0 & \frac{T^2}{2}\\
        0 & 0 & 1 & 0 & T & 0\\
        0 & 0 & 0 & 1 & 0 & T\\
        0 & 0 & 0 & 0 & \beta & 0\\
        0 & 0 & 0 & 0 & 0 & \beta\\
    \end{bmatrix}
    \quad
    \mathbf{G}_{k} = 
    \begin{bmatrix}
        0 & 0\\
        0 & 0\\
        0 & 0\\
        0 & 0\\
        1 & 0\\
        0 & 1\\
    \end{bmatrix}
    \quad
    \mathbf{H}_{k}^{t} = 
    \begin{bmatrix}
        1 & 0\\
        0 & 1\\
        0 & 0\\
        0 & 0\\
        0 & 0\\
        0 & 0\\
    \end{bmatrix}
\end{equation}
\begin{equation}
    \mathbf{w}_{k} = T \times 
    \begin{bmatrix}
        \sigma_{D_{k}}^2 & 0 & 0 & 0 & 0 & 0\\
        0 & \sigma_{D_{k}}^2 & 0 & 0 & 0 & 0\\
        0 & 0 & \sigma_{\alpha_{k}}^2 & 0 & 0 & 0\\
        0 & 0 & 0 & \sigma_{\alpha_{k}}^2 & 0 & 0\\
        0 & 0 & 0 & 0 & \sigma_{Z_{k}}^2 & 0\\
        0 & 0 & 0 & 0 & 0 & \sigma_{Z_{k}}^2\\
    \end{bmatrix}
    \quad
    \mathbf{v}_{k} = 
    \begin{bmatrix}
        \sigma_{Z}^2 & 0\\
        0 & \sigma_{Z}^2\\
    \end{bmatrix}
\end{equation}

\subsection{Part 2}
\paragraph{Proposition}Suggérer un moyen de déterminer les matrices de covariances associées aux pseudomesures.

\paragraph{Résolution}La Matrice de Covariance \cite{jacobianMatrix} \cite{jacobianMatrixComputation} du vecteur $\mathbf{Z}_{k}$ peut être estimer avec l'utilisation du Jacobian de la façon suivante:

\subsection{Part 3}
\paragraph{Proposition}Implémenter l'algorithme résultant sous MATLAB.

\paragraph{Résolution}Codes en MATLAB:
\begin{scriptsize}\mycode
    % \lstinputlisting[language=MATLAB, linerange={1-5}]{TD/2022_09_14/Counter.hpp}
    \lstinputlisting[language=MATLAB]{exercice1.m}
\end{scriptsize}

\begin{scriptsize}\mycode
    \lstinputlisting[language=MATLAB]{kalmanCalculation.m}
\end{scriptsize}

\begin{scriptsize}\mycode
    \lstinputlisting[language=MATLAB]{kalmanFilter.m}
\end{scriptsize}


\section{Exercice 3}
\subsection{Part 1}
\paragraph{Proposition}Écrire l'équation d'observation linéarisée autour de $\hat{\mathbf{X}}(k | k-1)$ sous la forme $\hat{\mathbf{X}}(k) = \mathbf{H}_{k} \times \mathbf{X}(k) + \mathbf{b}_{k}$ en exprimant 

\paragraph{Résolution}


\subsection{Part 2}
\paragraph{Proposition}

\paragraph{Résolution}


\subsection{Part 3}
\paragraph{Proposition}

\paragraph{Résolution}


\subsection{Part 4}
\paragraph{Proposition}

\paragraph{Résolution}


\subsection{Part 5}
\paragraph{Proposition}

\paragraph{Résolution}



\section{Exercice 4}
\subsection{Part 1}
\paragraph{Proposition}

\paragraph{Résolution}


\subsection{Part 2}
\paragraph{Proposition}

\paragraph{Résolution}


\subsection{Part 3}
\paragraph{Proposition}

\paragraph{Résolution}



\section{Exercice 5}
\subsection{Part 1}
\paragraph{Proposition}

\paragraph{Résolution}


\subsection{Part 2}
\paragraph{Proposition}

\paragraph{Résolution}


\subsection{Part 3}
\paragraph{Proposition}

\paragraph{Résolution}



Ces sont les matrices définis dans l'Exercice 2. \cite{cellMATLAB} À la suite on a:
Pour définir les bruits on utilise les fonctions du MATLAB \cite{awgnMATLAB} \cite{wgnMATLAB} \cite{randnMATLAB} qui donnent des bruits gaussiens centrées et bruits qui suivent une Loi Normale. \cite{bruitBlancMATLAB} \cite{bruitBlancMATLABrand} \cite{bruitBlancMATLABwgn}\\

\bibliography{ref.bib}
\bibliographystyle{ieeetr}

\end{document}