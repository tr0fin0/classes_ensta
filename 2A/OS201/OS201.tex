\documentclass{article}
\usepackage{C:/Users/guitr/Documents/git_repositories/tpack/tpack}
% \usepackage{C:/Users/Admin-PC/Documents/git_repository/tpack/tpack}
% \usepackage{/home/tr0fin0/git_repositories/tpack/tpack}


\title{OI201 - Systèmes d'Exploitation}
\project{Résumé Théorique}
\author{Guilherme Nunes Trofino}
\authorRA{2022-2024}


\makeatletter
\begin{document}\selectlanguage{french}
\maketitle


\newpage\tableofcontents

\section{Introduction}
\subfile{C:/Users/guitr/Documents/git_repositories/classes_ensta/intro.tex}
% \subfile{C:/Users/Admin-PC/Documents/git_repository/classes_ensta/intro.tex}
% \subfile{/home/tr0fin0/git_repositories/classes_ensta/intro.tex}


\subsection{Information Matier}
\paragraph{Référence}


\section{Travail Dirigé}
\subsection{Séance 07/11/2022}
\paragraph{Définition}

\subsection{Séance 14/11/2022}
\paragraph{Définition}16 kilo octets = 16 kilo bytes = 16 000 bytes
dans le td il faut faire la plut part hardcoded
char *heap c'est un tableau de pointeurs et char heap c'est un tableau de char
memalloc doit reçu

\end{document}