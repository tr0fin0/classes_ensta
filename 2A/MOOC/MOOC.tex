\documentclass{article}
\usepackage{C:/Users/Admin-PC/Documents/git_repository/tpack/tpack}
% \usepackage{/home/tr0fin0/git_repositories/tpack/tpack}


\title{MOOC - Gestion de Projet}
\project{Résumé Théorique}
\author{Guilherme Nunes Trofino}
\authorRA{217276}


\makeatletter
\begin{document}\selectlanguage{french}
\maketitle

\newpage\tableofcontents

\section{Introduction}
\subfile{C:/Users/Admin-PC/Documents/git_repository/classes_ensta/intro.tex}
% \subfile{/home/tr0fin0/git_repositories/classes_ensta/intro.tex}

\subsection{Informations Matière}
\paragraph{Présentation}Ce cours sera présenter par MOOC qui a pour but d'étudier la Gestion de Projet avec des informations dans \href{https://drive.google.com/drive/folders/0B2LN5eYzM2xhNlZORkVCQjBpQTA?resourcekey=0-hcgRqPSj5pIJUyyMe8y2cA}{Google Drive}.

\section{Semaine 1}
\subsection{Qu'est-ce qu'un projet?}
\paragraph{Définition}Selon le Project Management Institute:
\begin{phrase}
    Action temporaire entreprise dans le but de créer un produit, un service ou un résultat unique
\end{phrase}
En générale il va avoir des conditions suivantes:
\begin{enumerate}
    \item \textbf{Temporaire}: un début et une fin;
    \begin{enumerate}[noitemsep]
        \item \texttt{Qualité};
        \item \texttt{Coût};
        \item \texttt{Délais};
    \end{enumerate}
    \item \textbf{Prévoir}: anticiper les actions dans le temps;
    \item \textbf{Résultat}: un livrable à la fin;
\end{enumerate}
\paragraph{Paradoxe de Management de Projet}Au début on aura beaucoup d'options pour decider le projet mais peu de connaissance. À la fin on aura peu d'options pour decider, car le projet est déjà pensé, mais beaucoup de connaissance.

\subsubsection{Questions}


\subsection{Les Organigrammes Projets}
\paragraph{Définition}Classification de projets en ordre croissante de taille:
\begin{enumerate}
    \item \textbf{Projet Local}: Même service internes;
    \item \textbf{Projet Transversale}: Plusieurs services internes:
    \begin{enumerate}[noitemsep]
        \item \texttt{Matricielle Fonctionnelle}: pas de chef de projet pour division, départements se coordonnent directement;
        \item \texttt{Matricielle Faible}: le Coordination de Projet de division n'a pas plus de pouvoir;
        \item \texttt{Matricielle Forte}: le Coordination de Projet de division a plus de pouvoir;
        \item \texttt{Sorti}: collaborateurs font une équipe independantes;
    \end{enumerate}
    \item \textbf{Projet Sortis}: Plusieurs services internes avec des intervenants détaches;
    \item \textbf{Joint-Venture}: Projet avec d'autres entreprises; 
\end{enumerate}

\paragraph{Gouvernance}Ensemble des systèmes mis en oeuvre pour assure la circulation efficace de l'information et de la prise de décisions.

\subsubsection{Questions}


\subsection{Les Profils de Projet}
\begin{enumerate}
    \item \textbf{Enjeu}:
    \item \textbf{Innovation}:
    \item \textbf{Autonomie}:
    \item \textbf{Budget}:
\end{enumerate}

\subsubsection{Questions}


\subsection{La Dualité Projet-Opérations}
\paragraph{Définition}Opération est répétitif.

\subsubsection{Questions}


\subsection{Coût Global et Investissement}
\begin{enumerate}[]
    \item \textbf{Études de Marketing}: Définir les besoin a satisfaire et le marché visé;
    \item \textbf{Business Case}: Évaluer sa pertinence et en quoi il es plus intéressant que d'autres projets possibles; 
\end{enumerate}

\section{Semaine 2}
parties prenantes stakeholders
    negocier les objectifs
    triangule qualité coûts delais
        scope / cost / time in english
    carte des parties prenantes
        enjeux pour le projet
        enjeux contre le projet

matrice SWOT
    Strengths Qualites
    Weakeness Defauts
    Opportunities Competences
    Threats Lacunes
    \begin{table}[H]
        \centering\begin{tabular}{r|c|c}
            x       & Positif       & Négatif  \\\hline
            Interne & Strengths     & Weakeness\\\hline
            Externe & Opportunities & Threats  \\\hline
        \end{tabular}
        \caption{Matrice SWOT}
        \label{tab:SWOT}
    \end{table}
    quand: avant commencer
    comment: impliquez

PDCA
    Plan
    Do
    Check
    Act

réunion technique
    but: travail sur des points précis
    participants: uniquement personnes concernées
    séquence:
        ordre du jour
        point per point
        objectifs
    peut dépasser 1 heure

réunion de chantier
    but: presenter l'essentiel: avancement / retards sur les livrables
    participants: comité de pilotage, chef de projet, responsables opérationnels
    séquence: standard
    durée: courte: jamais plus d'une heure. aller à l'essentiel

réunion d'avancement
    but: suivre déroulement, détecter/traiter les blocages
    participants: membres de l'équipe projet
    séquence standard
    durée: moins d'une heure si possibles

réunion stand-up meeting
    principe: rester debou
    but: point rapide entre membres de l'equipe
    participants: ceux qui sont disponibles
    séquence:
        qu'ai-je fait depuis dernière réunion
        ce que j'ai à faire
        difficultés rencontrées
    durée: vite

compte-rendu: ata de reunião

les objetifs smarts
    specifique
    mesurable
    ambitieux
    realiste
    temporellement defini

todo list
    pilote: qui
    action: fait quoi
    délai: pour quand
    
    tache pilote échéance charge de travail livrables priorisation qui valide comment d'avancement mise à jour retard cause

l'effet tunnel

\section{Couch}
16 personalites
ikiban
les objetifs smarts
    specifique
    mesurable
    ambitieux
    realiste
    temporellement defini
les objectif smack
    specifique
    motivan
    ambitieux
    coherent
    Ki c le patron
SWOT
    Strengths Qualites
    Weakeness Defauts
    Opportunities Competences
    Threats Lacunes
    \begin{table}[H]
        \centering\begin{tabular}{r|c|c}
            x       & Positif       & Négatif  \\\hline
            Interne & Strengths     & Weakeness\\\hline
            Externe & Opportunities & Threats  \\\hline
        \end{tabular}
        \caption{Matrice SWOT pour Personnalité}
        \label{tab:SWOTcouch}
    \end{table}
le PDCA
    Plan
    Do
    Check
    Act
le graphe radar
    Autonomie
    compréhension besoin
    créer la confiance
    analyse du besoin
    conseiller
    gérer les risques
    rigueur
le qqoqcp
    quoi
    qui
    ou
    quand
    comment
    pourquoi
post mortem
    bilan
    funcionne, pas funcionne

\end{document}