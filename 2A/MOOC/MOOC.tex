\documentclass{article}
\usepackage{C:/Users/Admin-PC/Documents/git_repository/tpack/tpack}
% \usepackage{/home/tr0fin0/git_repositories/tpack/tpack}


\title{MOOC - Gestion de Projet}
\project{Résumé Théorique}
\author{Guilherme Nunes Trofino}
\authorRA{217276}


\makeatletter
\begin{document}\selectlanguage{french}
\maketitle

\newpage\tableofcontents

\section{Introduction}
\subfile{C:/Users/Admin-PC/Documents/git_repository/classes_ensta/intro.tex}
% \subfile{/home/tr0fin0/git_repositories/classes_ensta/intro.tex}

\subsection{Informations Matière}
\paragraph{Présentation}Ce cours sera présenter par MOOC qui a pour but d'étudier la Gestion de Projet avec des informations dans \href{https://drive.google.com/drive/folders/0B2LN5eYzM2xhNlZORkVCQjBpQTA?resourcekey=0-hcgRqPSj5pIJUyyMe8y2cA}{Google Drive}.

\section{Semaine 1}
\subsection{Qu'est-ce qu'un projet?}
\paragraph{Définition}Selon le Project Management Institute:
\begin{phrase}
    Action temporaire entreprise dans le but de créer un produit, un service ou un résultat unique
\end{phrase}
En générale il va avoir des conditions suivantes:
\begin{enumerate}
    \item \textbf{Temporaire}: un début et une fin;
    \begin{enumerate}[noitemsep]
        \item \texttt{Qualité};
        \item \texttt{Coût};
        \item \texttt{Délais};
    \end{enumerate}
    \item \textbf{Prévoir}: anticiper les actions dans le temps;
    \item \textbf{Résultat}: un livrable à la fin;
\end{enumerate}
\paragraph{Paradoxe de Management de Projet}Au début on aura beaucoup d'options pour decider le projet mais peu de connaissance. À la fin on aura peu d'options pour decider, car le projet est déjà pensé, mais beaucoup de connaissance.

\subsubsection{Questions}


\subsection{Les Organigrammes Projets}
\paragraph{Définition}Classification de projets en ordre croissante de taille:
\begin{enumerate}
    \item \textbf{Projet Local}: Même service internes;
    \item \textbf{Projet Transversale}: Plusieurs services internes:
    \begin{enumerate}[noitemsep]
        \item \texttt{Matricielle Fonctionnelle}: pas de chef de projet pour division, départements se coordonnent directement;
        \item \texttt{Matricielle Faible}: le Coordination de Projet de division n'a pas plus de pouvoir;
        \item \texttt{Matricielle Forte}: le Coordination de Projet de division a plus de pouvoir;
        \item \texttt{Sorti}: collaborateurs font une équipe independantes;
    \end{enumerate}
    \item \textbf{Projet Sortis}: Plusieurs services internes avec des intervenants détaches;
    \item \textbf{Joint-Venture}: Projet avec d'autres entreprises; 
\end{enumerate}

\paragraph{Gouvernance}Ensemble des systèmes mis en oeuvre pour assure la circulation efficace de l'information et de la prise de décisions.

\subsubsection{Questions}


\subsection{Les Profils de Projet}
\begin{enumerate}
    \item \textbf{Enjeu}:
    \item \textbf{Innovation}:
    \item \textbf{Autonomie}:
    \item \textbf{Budget}:
\end{enumerate}

\subsubsection{Questions}


\subsection{La Dualité Projet-Opérations}
\paragraph{Définition}Opération est répétitif.

\subsubsection{Questions}


\subsection{Coût Global et Investissement}
\begin{enumerate}[]
    \item \textbf{Études de Marketing}: Définir les besoin a satisfaire et le marché visé;
    \item \textbf{Business Case}: Évaluer sa pertinence et en quoi il es plus intéressant que d'autres projets possibles; 
\end{enumerate}

\end{document}