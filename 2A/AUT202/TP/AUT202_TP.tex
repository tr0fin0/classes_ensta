\documentclass{article}
\usepackage{C:/Users/guitr/Documents/git_repositories/tpack/tpack}
% \usepackage{C:/Users/Admin-PC/Documents/git_repository/tpack/tpack}
% \usepackage{/home/tr0fin0/Documents/git_repositories/tpack/tpack}
% \usepackage{/home/Documents/git_repositories/tpack/tpack}
\usetikzlibrary{decorations.pathreplacing,calligraphy}

\title{AUT202 - Dynamique et Contrôle des Systèmes}
\project{Travail Pratique}
\author{Guilherme Nunes Trofino}
\authorRA{2022-2024}


\makeatletter
\begin{document}\selectlanguage{french}
\maketitle
\setlength{\parindent}{0pt}
% \sisetup{inter-unit-product =$\cdot$}
% \sisetup{inter-unit-product =\ensuremath{\cdot}}

\newpage\tableofcontents

\section{Introduction}
\subfile{C:/Users/guitr/Documents/git_repositories/classes_ensta/intro.tex}
% \subfile{C:/Users/Admin-PC/Documents/git_repository/classes_ensta/intro.tex}
% \subfile{/home/tr0fin0/Documents/git_repositories/classes_ensta/intro.tex}
% \subfile{/home/Documents/git_repositories/classes_ensta/intro.tex}

\subsection{Information Matier}
\paragraph{Référence}Dans cette matière le but sera de comprendre comment une Système d'Exploitation marche.


\subfile{sections/AUT202_TP_p1.tex}
\subfile{sections/AUT202_TP_p2.tex}
\subfile{sections/AUT202_TP_p3.tex}
\subfile{sections/AUT202_TP_p4.tex}
\subfile{sections/AUT202_TP_p5.tex}

\end{document}